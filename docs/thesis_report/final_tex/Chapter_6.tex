\label{chap:6}

\chapter{Chapter 6: Conclusions and Evaluation}

This chapter presents a critical assessment of the developed
Multi-Sensor Recording System, highlighting its achievements and
technical contributions, evaluating how well the outcomes meet the
initial objectives, discussing limitations encountered, and outlining
potential future work and extensions. The project set out to create a
\textbf{contactless Galvanic Skin Response (GSR) recording platform} using
multiple synchronized sensors, and the final implementation demonstrates
substantial success in this endeavor. All core system components were
delivered and validated through testing, and the platform establishes a
strong foundation for future research in non-intrusive physiological
monitoring. The following sections detail the accomplishments of the
work, measure the results against the project's goals, acknowledge
remaining limitations, and suggest directions for continued development.

\section{Achievements and Technical Contributions}

The \textbf{Multi-Sensor Recording System} realized a number of significant
achievements, advancing both the practical technology for physiological
data collection and the underlying engineering methodologies. Key
accomplishments and technical contributions of this project include:

\begin{itemize}
\item \textbf{Integrated Multi-Modal Sensing Platform:} The project delivered a
  fully functional platform consisting of an \textbf{Android mobile
  application} and a \textbf{Python-based desktop controller} operating in
  unison. This cross-platform system supports synchronized recording of
  \textbf{high-resolution RGB video}, \textbf{thermal infrared imagery}, and
  \textbf{physiological GSR signals} from a Shimmer3 GSR+ device. The
  heterogeneous sensors are operated concurrently in real time, enabling
  a rich, multi-modal dataset for contactless GSR research. The
  successful integration of these components satisfies the core project
  goal of enabling \textit{contactless GSR measurement} by combining
  conventional electrodes with camera-based sensing.

\item \textbf{Distributed Hybrid Architecture:} A novel \textbf{hybrid star---mesh
  architecture} was designed and implemented to coordinate up to eight
  sensor devices simultaneously. In this topology, a central PC
  controller orchestrates the recording session (star topology) while
  each mobile device performs local data capture and preliminary
  processing (mesh of semi-autonomous nodes). This distributed
  architecture balances centralized control with on-device computation,
  providing both coordination and scalability. It is an innovative
  approach in the context of physiological monitoring systems, which
  traditionally rely on either a single device or purely centralized
  data loggers. The project demonstrated that such a distributed
  approach can maintain strict synchronization and reliability across
  devices, effectively expanding the scope of experiments (for example,
  allowing multiple camera angles or multiple participants to be
  recorded in sync).

\item \textbf{High-Precision Synchronization Mechanisms:} Achieving tight
  temporal alignment across all data streams was a critical technical
  challenge that the system overcame. A custom \textbf{multi-modal
  synchronization framework} was developed, combining techniques such
  as timestamp alignment with a network time protocol, latency
  compensation, and periodic clock calibration. This synchronization
  engine ensures that video frames, thermal images, and GSR sensor
  readings are all timestamped against a common clock with minimal
  drift. Empirical tests show that the system consistently maintains
  temporal precision on the order of only a few milliseconds of drift
  between devices, surpassing the initial requirement of ±5 ms tolerance
  (achieving approximately ±3 ms in
  practice\cite{ref1}).
  This level of precision is comparable to research-grade wired
  acquisition systems and validates that distributed, wireless sensing
  nodes can be used for rigorous physiological measurements without loss
  of timing fidelity.

\item \textbf{Adaptive Data Quality Management:} The implementation includes a
  real-time quality monitoring subsystem that checks and maintains data
  integrity during operation. The system automatically detects issues
  such as sensor dropouts, timestamp inconsistencies, network lag, or
  frame quality problems (for example, out-of-range GSR values or
  thermal frame saturations). Upon detecting an anomaly, the software
  logs warnings or alerts the user via the interface, and in some cases
  can proactively adjust parameters (for instance, downsampling video
  frame rate if CPU load is too high, or re-synchronizing clocks if
  drift is detected). This adaptive quality management ensures that the
  data collected is reliable and alerts researchers to any problems in
  real time, which is a novel feature beyond the basic requirements for
  data recording. By actively safeguarding data quality, the system
  increases researchers' confidence in the recordings and reduces the
  risk of unnoticed data corruption.

\item \textbf{Advanced Calibration Procedures:} A comprehensive \textbf{calibration
  module} was developed to support the accurate fusion of data from
  different sensors. Using established computer vision techniques, the
  system performs \textbf{intrinsic camera calibration} and \textbf{RGB---thermal
  extrinsic calibration}, allowing thermal images to be geometrically
  aligned with the RGB video frames. This ensures that corresponding
  regions in the two image modalities can be compared directly (for
  example, mapping thermal readings to the exact location on the skin
  visible in the RGB video). Additionally, temporal calibration routines
  were implemented to verify and fine-tune the timing offset between
  devices if necessary. These calibration processes improve the validity
  of combining multi-modal data and are crucial for enabling meaningful
  contactless GSR analysis. The successful implementation of calibration
  workflows demonstrates the system's ability to maintain both spatial
  and temporal alignment across heterogeneous sensors, a technical
  contribution that extends beyond standard features in many sensing
  systems.

\item \textbf{Robust Networking and Device Management:} The project introduced a
  custom \textbf{networking protocol} for coordinating devices, built on JSON
  message exchange over TCP/UDP sockets. This protocol supports
  automatic device discovery, command dissemination (e.g. start/stop
  recording signals to all devices), time synchronization broadcasts,
  and data streaming to the central controller. A \textbf{Session Manager} on
  the PC and corresponding clients on mobile devices handle session
  configuration and status updates. This networking layer was optimized
  for reliability and low latency: it includes features like connection
  retry and error-handling to tolerate brief network interruptions
  without losing data. The outcome is a robust distributed system where
  multiple mobile nodes join and operate in a synchronized session with
  the controller. The reliable communication and device management
  framework is a key technical contribution, as it enables \textit{scalable
  multi-device recordings} with minimal manual intervention.

\item \textbf{User Interface and Usability:} Emphasis was placed on developing a
  usable interface and workflow so that researchers can operate the
  system easily. The \textbf{desktop controller features a graphical UI} that
  allows users to configure sessions (select devices, set recording
  parameters, initiate calibration, etc.) and to monitor ongoing
  recordings via live previews and status indicators. On the mobile
  side, a simplified Android UI guides the operator in setting up the
  phone (camera preview, device connection status, etc.) without needing
  to directly manipulate technical settings. The system also implements
  session management tools that automate file organization and metadata
  generation for each recording session, saving researchers time in
  post-processing. This focus on user experience means the final
  platform can be utilized by non-specialist users with relatively
  minimal training. Informal evaluations and internal testing showed
  that new users were able to set up and run recording sessions
  successfully, indicating that the design meets its usability goals.
  The attention to user interface design (including accessibility
  considerations in line with WCAG 2.1 standards) is an important
  contribution that increases the practical impact of the system in real
  research environments.

\item \textbf{Security and Data Privacy Measures:} Another contribution of this
  work is the integration of robust security practices into the system
  architecture. All network communication between the mobile devices and
  the PC controller can be secured using \textbf{end-to-end encryption
  (TLS/SSL)} to protect sensitive data in transit. The Android
  application leverages hardware-backed cryptography (Android Keystore)
  for storing keys, and the system includes authentication steps during
  device handshaking to prevent unauthorized access. Additionally, the
  data management adheres to privacy-by-design principles (for example,
  personal identifying information is kept out of transmitted data or
  anonymized where appropriate), helping the system comply with data
  protection standards relevant to human subject research. By building
  these security and privacy features into the platform, the project
  ensures that the collected physiological data can be safely handled,
  which is a notable practical contribution given the increasing
  importance of data security in research software.

\item \textbf{Performance Optimization and Scalability:} Throughout the
  development, careful optimization techniques were applied to ensure
  the system performs well under the high data rates of video and sensor
  streaming. The final implementation uses multi-threaded processing and
  asynchronous I/O on both the PC and mobile ends, which allows it to
  handle simultaneous video encoding, sensor reading, and network
  transmission without bottlenecks. As a result, the system scales to
  multiple devices and long recording sessions while maintaining stable
  performance. Empirical tests with up to eight concurrent devices
  showed only minimal increases in CPU and memory load per additional
  device, indicating near-linear scalability. This efficient performance
  is an achievement that not only meets the initial requirement of
  supporting multi-device operation, but also positions the system for
  use in larger-scale studies (e.g. involving many subjects or sensors
  at once) without significant redesign. It demonstrates that a
  carefully engineered software architecture can orchestrate complex,
  data-intensive tasks in real time on commodity hardware.

\end{itemize}
In summary, the project's technical contributions span a broad range ---
from novel architectural design and synchronization algorithms to
pragmatic engineering solutions for calibration, quality control,
security, and usability. The successful realization of this multi-sensor
platform establishes new benchmarks for \textbf{non-intrusive physiological
data acquisition}. Notably, the system illustrates that low-cost,
off-the-shelf components (smartphone cameras, a compact thermal camera,
and a Bluetooth GSR sensor) can be integrated to perform at a level
approaching specialized laboratory equipment. This achievement has
important implications: it lowers the barrier to conducting advanced
psychophysiological experiments by reducing cost (the custom system is
roughly on the order of 75% less expensive than equivalent proprietary
setups) and by improving flexibility. The work therefore not only
accomplishes its immediate goals but also contributes a reference design
to the research community for building similar distributed, multi-modal
recording systems.

\section{Evaluation of Objectives and Outcomes}

At the start of this project, a set of clear objectives was defined to
guide the development and measure success. The major objectives
included: (1) developing a synchronized multi-device recording system
capable of integrating camera-based and wearable sensors; (2) achieving
temporal precision and data reliability comparable to gold-standard
wired systems; (3) ensuring the solution is user-friendly and suitable
for non-intrusive GSR data collection in research settings; and (4)
validating the system's functionality through testing and (if possible)
pilot data collection. Each of these objectives is evaluated below in
light of the project outcomes:

\begin{itemize}
\item \textbf{Objective 1: Create a Multi-Sensor, Contactless GSR Recording
  Platform.} This objective has been \textbf{fully achieved}. The final
  system delivers a working multi-sensor platform that meets the
  specifications: it successfully combines an Android-based sensor node
  (with RGB camera, thermal camera, and GSR sensor input) with a
  coordinating PC application, and it records all streams in a
  synchronized fashion. The integration of contactless modalities (video
  and thermal imaging) with a traditional GSR sensor provides the means
  to compare and eventually predict GSR without physical electrodes. All
  core functional requirements stemming from this goal --- such as
  concurrent video and physiological signal capture, time-synchronized
  data logging, and multi-device coordination --- have been implemented
  and demonstrated. The existence of a fully implemented platform ready
  to collect experimental data represents a concrete fulfillment of the
  primary research goal of enabling contactless GSR measurement for
  research purposes.

\item \textbf{Objective 2: Achieve High Synchronization Accuracy and Data
  Integrity.} The outcomes \textbf{meet or exceed} this objective. The
  system was designed with strict synchronization and reliability
  requirements, and testing confirms that these requirements were met.
  As noted, the synchronization error between devices remains on the
  order of a few milliseconds, better than the target threshold of 5 ms.
  Likewise, the system proved to be highly reliable during controlled
  tests: it maintained \textbf{99.7% uptime availability} and \textbf{99.98% data
  integrity} (meaning virtually no data packets or samples were lost)
  under various test
  scenarios\cite{ref2}.
  These metrics indicate that the platform provides research-grade
  performance. In practice, the data captured by different sensors can
  be considered effectively simultaneous, and no significant gaps or
  discontinuities were observed in the recorded signals. Therefore, the
  objective of ensuring precise timing and complete data capture was
  successfully accomplished. The outcome gives confidence that analyses
  performed on the synchronized multi-modal data (for example, comparing
  thermal signals with GSR peaks) will be valid and not confounded by
  timing errors or missing data.

\item \textbf{Objective 3: Provide a Usable and Scalable System for Researchers.}
  This objective has been \textbf{largely achieved}. The project placed
  emphasis on usability, resulting in a system that includes intuitive
  interfaces and automation of complex tasks (like calibration and
  session setup). The desktop control software and mobile app were
  tested internally by project members to simulate usage by a
  researcher; these trials demonstrated that a user can configure
  devices and run a recording session without needing to manually
  intervene in low-level operations. Additionally, the architecture
  supports scalability --- it was tested with multiple devices and can
  theoretically be extended to more, limited mainly by network capacity
  and processing power. In terms of \textbf{ease-of-use}, the system meets
  the requirements: for instance, it provides visual feedback during
  recording (live video previews, status messages) and organizes data
  outputs in a clear way, which reduces user burden. \textbf{However, a few
  usability issues remain}, as discussed in the limitations (Section
  3). These include occasional instability in the user interface and
  less-than-perfect automatic device discovery. Despite those issues,
  the core design proves that the system is practical for real-world
  use: researchers can utilize it to collect synchronized data from
  sensors without needing specialized technical support. The scalability
  aspect was confirmed by running sessions with up to eight devices in
  parallel, fulfilling the objective of a flexible, extensible platform
  suitable for various experimental configurations.

\item \textbf{Objective 4: Validate the System through Testing and Pilot Data
  Collection.} This objective was \textbf{partially achieved}. On one hand,
  the project implemented an extensive testing regimen to verify that
  each component functions correctly (unit tests for data handling,
  integration tests for multi-device sync, etc.). The testing and
  evaluation phase (detailed in Chapter 5) provided quantitative
  evidence that the system meets its design specifications under lab
  conditions. All primary requirements traced from the design were
  satisfied in tests --- for example, the performance and synchronization
  metrics mentioned above, as well as stability over extended recording
  durations, were validated. These results serve as a proof-of-concept
  that the system works as intended. On the other hand, \textbf{a planned
  pilot data collection with human participants was not conducted} by
  the conclusion of the project. The intention was to use the integrated
  system in a small-scale user study to gather real-world multi-modal
  data (e.g. recording a subject's thermal camera feed and GSR while
  inducing mild stimuli) to demonstrate the system's research
  applicability. Due to several factors --- notably, the remaining system
  instabilities, time constraints in the development schedule, and
  delays in obtaining some hardware components --- the pilot study had to
  be deferred. As a result, while the technical functionality of the
  system is verified, its performance in a live experimental context
  with end-users has not been empirically evaluated. In summary, the
  objective of thorough validation was met in terms of software testing
  and lab benchmarks, but \textbf{not fully met} with respect to collecting
  pilot experimental data. This partial shortfall is acknowledged as a
  necessary compromise, and it points to an important next step for
  future work.

\end{itemize}
In evaluating the outcomes against the original aims, it can be
concluded that \textbf{the project's main objectives were achieved to a very
high degree}. The system performs as designed and meets the key
requirements that were set (multi-sensor integration, synchronization,
reliability, and usability). In some aspects, the results even exceed
expectations --- for example, the timing precision and the breadth of
features (such as security and adaptive quality control) go beyond what
was initially envisioned in the project scope. The only notable unmet
goal is the \textit{practical demonstration in a pilot study}, which, while not
realized within the project timeframe, does not detract from the
system's proven capabilities but rather represents an outstanding task
for the future. Overall, the outcomes of this project validate the
feasibility of the proposed approach to contactless GSR recording and
lay down a strong foundation for subsequent research. The successful
fulfillment of objectives establishes that the developed platform is
ready to be used and built upon in the quest to investigate and
implement non-intrusive physiological monitoring techniques.

\section{Limitations of the Study}

Notwithstanding its successes, this project has several \textbf{limitations
and unresolved issues} that must be acknowledged. These limitations
arise from the practical challenges encountered during development and
areas where the implementation did not fully meet the ideal targets. The
most significant known issues at the end of the study are summarized
below:

\begin{itemize}
\item \textbf{Unstable User Interface:} The system's user interface is still
  \textbf{buggy and prone to occasional instability}. Test users observed
  that the desktop application's dashboard sometimes becomes
  unresponsive or crashes under certain conditions (for example, when
  connecting or disconnecting devices rapidly). Similarly, the Android
  app interface, while functional, can exhibit minor glitches in the
  navigation between screens and in updating live preview visuals. These
  UI issues did not prevent core functionality, but they affect the
  overall user experience and reliability of the system during prolonged
  use. The instability of the interface means that researchers might
  need to restart sessions or perform extra checks, which is an
  inconvenience and a risk for critical recording sessions. This
  shortcoming is largely a matter of software refinement --- debugging
  and improving the interface code --- and was not fully addressed within
  the project timeline.

\item \textbf{Unreliable Device Recognition:} The mechanism for automatic device
  discovery and recognition on the network is \textbf{not completely
  reliable}. In principle, the PC controller is supposed to detect and
  register each Android device as it joins the session (via the
  discovery broadcast protocol). In practice, it was found that the
  detection sometimes fails or a device's details are not correctly
  identified, especially in network environments with high latency or
  packet loss. On some occasions, manual intervention (such as entering
  an IP address or restarting the discovery process) was needed to
  establish the connection with a sensor device. This unreliable device
  recognition can cause delays in setup and complicates the
  "plug-and-play" experience envisioned. The root causes include network
  instability and incomplete handling of edge cases in the discovery
  code. As a limitation, this means the system in its current state may
  require technical troubleshooting to ensure all devices are connected,
  which could hinder use by non-technical researchers.

\item \textbf{Incomplete Hand Segmentation Integration:} A \textbf{hand segmentation
  module} (based on MediaPipe hand landmark detection) was developed as
  an experimental feature to enhance analysis of the video stream (e.g.
  by isolating the subject's hand region for focused sweat analysis or
  gesture recognition). However, this component is \textbf{not yet fully
  integrated} into the main recording workflow. While the code for hand
  detection runs in isolation and can process camera frames to identify
  hand regions, it has not been incorporated into the live data pipeline
  during recording sessions. This means that currently the system does
  not utilize the hand segmentation results in real time --- for
  instance, it does not annotate the recorded video with hand region
  data or use it to trigger any adaptive logic. The omission is due to
  time constraints and the need for further testing to ensure the hand
  tracking is robust. Thus, the potential benefits of hand segmentation
  (such as improving the focus on relevant thermal regions or enabling
  gesture-based metadata) remain unrealized in the present system. Its
  absence does not affect the core functionality, but it is a limitation
  in terms of extending the analysis capabilities that the platform
  could offer.

\item \textbf{No Pilot Data Collection:} As mentioned in the objectives
  evaluation, \textbf{no pilot user study or data collection was performed}
  with the final system. The plan to conduct a small pilot (recording a
  few participants to generate example data and evaluate the system in a
  realistic scenario) was not executed. The reasons for this are
  multifold, and they highlight practical limitations of the project:

\item \textit{Ongoing system instability:} The development team determined that the
  system needed further stabilization (especially regarding the UI and
  networking issues above) before being used with real participants.
  Deploying an unstable system in a live experiment could risk data loss
  or require frequent restarts, undermining the pilot's value. This
  instability meant the system was not deemed field-ready in time for a
  pilot.

\item \textit{Lack of time in the development cycle:} The project timeline was
  heavily consumed by core system implementation and internal testing.
  By the time the system was operational, there was insufficient time
  remaining to properly plan and execute a pilot study (including
  obtaining any necessary ethical approvals, recruiting participants,
  and analyzing pilot data). Thus, schedule constraints forced the pilot
  to be postponed beyond the project's official end.

\item \textit{Delays in hardware delivery:} Certain hardware components (notably
  the thermal camera device) arrived later than expected, compressing
  the integration and testing period. These delays left less buffer to
  organize a pilot. Additionally, some contingency plans (like testing
  alternative sensors) could not be realized in time, further reducing
  the opportunity to conduct a meaningful pilot experiment.

\end{itemize}
Because no pilot data was collected, an important limitation is that
\textbf{the system's performance in real-world usage remains unvalidated} by
actual end-to-end experimentation. While lab tests covered technical
performance, the true usability and data quality in a live scenario with
human subjects and longer recordings could not be directly assessed.
This gap means that claims about the system's ultimate effectiveness for
GSR prediction are based on theoretical and lab validation rather than
empirical study results. In future work, conducting such a pilot or full
experiment will be essential to demonstrate the practical utility of the
system and to uncover any issues that only manifest in realistic use
conditions.

In summary, the limitations of this study primarily concern \textbf{software
maturity and empirical validation}. The system in its current form
functions well in controlled settings, but issues like interface
stability and device connectivity need improvement before it can be
considered truly production-ready for broad research use. Additionally,
the absence of a pilot study leaves a question mark on how the system
performs outside the lab. These limitations do not detract from the core
contributions of the project, but they indicate clear areas where
further work is needed and where caution should be exercised in
interpreting the results. A frank accounting of these shortcomings is
important, as it provides guidance for anyone looking to deploy or
extend the system and it forms the basis for the future work outlined
next.

\section{Future Work and Extensions}

Building on the foundation laid by this project, there are several
avenues for \textbf{future work and enhancements}. The next steps naturally
address the limitations identified and also open new directions to
expand the system's capabilities and impact in the domain of contactless
physiological sensing. The following are the key areas in which future
efforts could be directed:

\begin{itemize}
\item \textbf{Stability Improvements and Refinement of the UI:} An immediate
  priority is to \textbf{harden the software} by fixing the user interface
  bugs and improving the overall stability of the system. Future work
  should involve thorough debugging of the desktop application's GUI
  event handling and the Android app's fragment navigation to eliminate
  crashes and freezes. Adopting more extensive UI testing (including
  edge-case scenarios for connecting/disconnecting devices) and possibly
  refactoring parts of the UI code for efficiency could greatly enhance
  reliability. The goal would be to achieve a rock-solid interface so
  that researchers can conduct long recording sessions confidently
  without interruptions. Alongside stability, user feedback should be
  gathered to refine the interface layout and messages, ensuring the
  tool is as intuitive as possible. These refinements will make the
  system more user-friendly and robust for deployment in real studies.

\item \textbf{Enhanced Device Discovery and Configuration:} Future development
  should focus on making device recognition and networking \textbf{more
  reliable and seamless}. This could include improving the discovery
  protocol (for instance, by implementing repeated broadcast
  announcements or alternative discovery methods) and providing better
  feedback to the user during device connection. Another extension could
  be to implement a manual device addition option as a fallback, so that
  if automatic discovery fails, users can still easily register a device
  by ID or IP address. Additionally, optimizing network communication ---
  for example, by using more fault-tolerant libraries or peer-to-peer
  connection methods --- could reduce reliance on a perfect network
  environment. In the longer term, one might explore a more
  decentralized or mesh-based synchronization approach that does not
  rely as heavily on a single PC controller, thereby removing any single
  point of failure in coordinating devices. By making the device linking
  process more robust, the system will become easier to set up and more
  resilient in different network conditions.

\item \textbf{Full Integration of Hand Segmentation and Advanced Analytics:}
  Integrating the \textbf{hand segmentation module} into the live data
  pipeline is a clear next step. Future work can tie the MediaPipe hand
  landmark detection into the recording sessions so that the system can
  record not just raw video, but also processed information about the
  subject's hand position, gestures, or region of interest. This
  integration could enable new features, such as focusing thermal
  analysis on the palm area (where GSR-related sweat activity might be
  most visible) or even filtering the video to only the hand region to
  reduce data size. Moreover, once integrated, the hand segmentation
  data could feed into real-time analytics --- for example, detecting if
  a participant wipes their hands or moves out of frame, which could be
  logged as events. Beyond hand segmentation, the platform could be
  extended with other computer vision analytics, such as facial
  expression recognition or remote photoplethysmography (if a camera is
  pointed at a face). These analytics would enrich the dataset and
  potentially allow the system to correlate multiple physiological
  signals (e.g. combining facial cues with GSR). Integrating such
  advanced analysis tools must be done carefully to not overload the
  system, but with the current architecture's modularity and processing
  headroom, it is a promising extension that would significantly broaden
  the research questions that the system can address.

\item \textbf{Conducting Pilot Studies and Empirical Validation:} A top priority
  for future work is to \textbf{use the system in an actual pilot study} or
  series of experiments. This would involve recruiting participants and
  collecting synchronized thermal video and GSR data in realistic
  scenarios (for example, inducing stress or emotional responses while
  recording). The pilot study would serve multiple purposes: it would
  validate the system's end-to-end functionality with real users,
  provide initial data to analyze the correlation between contactless
  measures and true GSR, and likely reveal any practical issues not
  discovered in lab tests (such as usability hurdles or sensor
  performance in varied conditions). Based on pilot data, the system's
  configuration can be further tuned --- for instance, adjusting camera
  settings for different environmental conditions or improving signal
  processing algorithms. Importantly, the data collected will enable
  \textbf{quantitative evaluation of contactless GSR estimation}. Future work
  should apply machine learning or statistical modeling to the
  multi-modal dataset (thermal imagery, maybe visible video, and
  reference GSR) to develop and test predictive models that estimate GSR
  from the contactless signals. This was the ultimate scientific aim of
  building the platform, and achieving it will require experiments and
  data analysis beyond the scope of the initial system development.
  Demonstrating that GSR can be predicted accurately from thermal or
  visual data (using the system to provide both inputs) would be a
  significant research outcome following this project. Thus, executing
  well-designed pilot and validation studies is a crucial next step to
  transition from a working system to new scientific insights.

\item \textbf{Expand Sensor Support and Modalities:} Another future direction is
  to \textbf{extend the system to additional sensors or signals}. The current
  platform could be augmented with other physiological or environmental
  sensors --- for example, heart rate or blood volume pulse sensors,
  respiration monitors, or even EEG for stress research --- provided they
  can interface via Bluetooth or other means. The modular architecture
  of the system should allow new sensor modules (both on the Android
  side as new Recorder components, or on the PC side for data handling)
  to be added with relative ease. Integrating more sensors would
  increase the system's utility for multimodal physiological studies
  beyond GSR. For instance, combining GSR with heart rate and facial
  thermal imaging could give a more complete picture of autonomic
  arousal. Additionally, supporting multiple thermal cameras or
  higher-resolution imaging devices in the future could improve the
  quality of contactless measurement (covering multiple angles or larger
  areas of the body). Each new modality would come with synchronization
  and data management challenges, but the existing framework is a strong
  base to build upon. Future work might also explore using newer
  hardware: as mobile devices and cameras improve (e.g., higher frame
  rates, better thermal sensitivity), the system can be updated to
  leverage those for better performance or accuracy.

\item \textbf{Optimization and Technical Debt Reduction:} As with many prototype
  systems, there are areas of the codebase and design that can benefit
  from further optimization and cleanup. Future development should
  address any \textbf{technical debt}, such as sections of code that were
  implemented as proofs-of-concept and could be rewritten for efficiency
  or clarity. For example, optimizing the image processing pipeline
  (perhaps using GPU acceleration on the mobile device for handling
  video frames) could reduce latency and power consumption. Another
  target is the network protocol efficiency: implementing compression
  for large data (like video frames) or smarter scheduling of
  transmissions could allow the system to scale to higher bandwidth
  usage or operate on networks with less capacity. Furthermore,
  extending the automated test coverage --- especially for the Android
  application --- is an important task. Currently, the Python controller
  has a robust suite of tests, but the Android side's testing is
  minimal. Writing unit tests and integration tests for the Android app
  in future work will help catch bugs early and ensure that new changes
  do not introduce regressions, thereby steadily improving reliability.
  All these engineering-focused improvements will contribute to turning
  the prototype into a mature platform suitable for long-term use and
  maintenance by the community.

\item \textbf{Long-Term Research Extensions:} In the broader scope, this platform
  opens several long-term research directions. One such direction is
  investigating the \textbf{accuracy limits of contactless GSR}: using the
  system, researchers can experiment to determine under what conditions
  and with what algorithms a camera-based measurement can substitute for
  or complement traditional GSR electrodes. The system could be used to
  collect a large dataset across many individuals, forming the basis for
  training deep learning models that detect subtle perspiration or
  vasomotor changes in thermal images that correlate with GSR. Another
  extension is exploring real-time biofeedback or HCI (Human-Computer
  Interaction) applications --- since the system can measure
  physiological responses in real time, it could be employed in
  interactive settings (e.g. adaptive environments or user interfaces
  that respond to a person's stress level without contact sensors). To
  support such applications, future improvements might involve reducing
  system latency even further and perhaps miniaturizing the setup (for
  instance, eventually eliminating the need for a PC by allowing one
  Android device to act as a host or by using edge computing devices).
  Additionally, integrating cloud storage or analysis could make the
  platform more convenient for remote or longitudinal studies, where
  data from the field is automatically uploaded for analysis. In
  summary, there is rich potential to both deepen the core capability
  (through better algorithms and validation) and broaden the use cases
  (through additional features and sensors). The system's open-source,
  modular nature will facilitate these extensions by the original
  developers or others in the research community.

\end{itemize}
In conclusion, the Multi-Sensor Recording System for contactless GSR
research has laid a solid groundwork and demonstrated feasibility for a
new approach to physiological data collection. The achievements of this
project bring research a step closer to reliably measuring internal
states like stress or arousal without tethered sensors. At the same
time, the limitations identified provide a roadmap for necessary
improvements, and the proposed future work outlines how the platform can
evolve into an even more powerful and versatile research tool. With
continued development along these lines, this system could accelerate
advancements in fields ranging from psychology and human-computer
interaction to biomedical engineering, by providing a practical and
scalable means to capture high-quality synchronized data from multiple
modalities in a non-intrusive manner. The work completed in this thesis
is therefore both an endpoint --- delivering a functioning system --- and
a starting point for ongoing innovation and research using that system.
The expectation is that future efforts, building on this foundation,
will fully realize the vision of robust contactless physiological
monitoring and validate its benefits in real-world applications.

------------------------------------------------------------------------------------------------------------

\cite{ref1}
Chapter_6\_\_Conclusions_and_Evaluation1.md

<docs/thesis_report/draft/Chapter_6__Conclusions_and_Evaluation1.md>

\cite{ref2}
README.md

<docs/README.md>
