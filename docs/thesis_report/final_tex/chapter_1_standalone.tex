\documentclass[11pt,a4paper]{report}

% Essential packages
\usepackage[utf8]{inputenc}
\usepackage[T1]{fontenc}
\usepackage[english]{babel}
\usepackage{cite}
\usepackage{graphicx}
\usepackage{amsmath}
\usepackage{amsfonts}
\usepackage{amssymb}
\usepackage{hyperref}
\usepackage{geometry}
\usepackage{setspace}

% Page geometry and spacing
\geometry{margin=1in}
\onehalfspacing

\begin{document}

% Chapter 1 content
\label{chap:1}
\chapter{Chapter 1: Introduction}

\section{Background and Motivation}

Stress is a common physiological and psychological response that significantly
impacts human-computer interaction (HCI), health monitoring, and emotion recognition.
Measuring a user's stress level reliably and unobtrusively is valuable in contexts
like adaptive user interfaces and mental health assessment.  Galvanic Skin Response
(GSR), or electrodermal activity, is a recognized index of stress and arousal,
reflecting sweat gland activity through skin conductance measurements
\cite{Boucsein2012}.  Traditional GSR monitoring techniques, however, rely on
attaching electrodes to the skin (typically on the fingers or palm) to sense minute
electrical conductance changes \cite{Fowles1981}.  While effective in controlled
settings, this contact-based approach has significant drawbacks: sensors can be
intrusive and uncomfortable, often altering users' natural behaviour and emotional
state \cite{Cacioppo2007}.  In other words, the very act of measuring stress via
contact sensors may itself induce stress or otherwise confound the measurements,
raising concerns about ecological validity in HCI and ambulatory health scenarios
\cite{Wilhelm2010}.  Moreover, contact sensors tether participants to devices,
limiting mobility and making longitudinal or real-world monitoring cumbersome.

These limitations motivate the pursuit of contactless stress measurement methods that
can capture stress-related signals without any physical attachments, thereby
preserving natural behaviour and comfort.  Recent advances in sensing and computer
vision suggest that it may be feasible to infer physiological stress responses using
ordinary cameras and imaging devices, completely bypassing the need for electrode
contact \cite{Picard2001}.  Previous work in affective and physiological computing
shows that visual cues—facial expressions, skin pallor, perspiration, and subtle
movements—correlate with emotional arousal and stress levels \cite{Healey2005}.  For
example, thermal infrared imaging of the face can detect temperature changes linked
to blood flow variations under stress, such as cooling of the nose tip due to
vasoconstriction, in a non-contact manner.  Likewise, high-resolution RGB video can
monitor heart rate or breathing rate through subtle skin colour fluctuations and
movements, as demonstrated in emerging remote photoplethysmography techniques
\cite{Poh2010}.

\section{Research Problem and Objectives}

\subsection{Problem Context and Significance}

Recent advancements in contactless sensing pose a crucial question at the
intersection of computer vision and psychophysiology: Can GSR-based stress
measurements be approximated using only RGB camera video data?  Specifically, can a
simple video recording provide enough information to estimate physiological stress
responses, eliminating the need for skin contact sensors?  An affirmative answer
could have significant implications.  It would enable accessible stress monitoring
through smartphones or laptop cameras, seamlessly integrating stress detection into
everyday interactions and health monitoring, without the need for wearables or
electrodes.

\subsection{Aim and Specific Objectives}

To investigate this research question, this project aims to develop and validate a
system for contactless stress measurement.  The primary objectives are to develop a
multimodal data acquisition platform, utilise it to conduct a controlled
stress-induction experiment, and analyse the resulting data to model GSR based on
video features.

To achieve this, we first developed a multi-sensor data acquisition platform, named
\textit{bucika\_gsr}.  The system architecture spans two tightly integrated
components: a custom Android mobile application and a desktop PC application.  The
Android app operates on a modern smartphone equipped with an attachable thermal
camera module, simultaneously capturing both thermal and standard high-definition RGB
video.  Complementing this, the desktop PC application serves as the master
controller, connecting via Bluetooth to a Shimmer3 GSR+ sensor to record the
participant's ground-truth skin conductance in real-time.

A key objective was to ensure precise temporal alignment.  The Android and PC
components communicate over a wireless network, following a master--slave
synchronisation protocol.  The PC controller orchestrates the timing of recordings
across all devices, achieving timestamp synchronisation on the order of a few
milliseconds.  Such tight synchronisation is crucial for our research, as it enables
frame-by-frame correlation of physiological signals with visual cues captured on
video \cite{Gravina2017}.  The development of this platform involved several
technical contributions, including a real-time multi-device synchronisation
mechanism, an integrated framework for capturing heterogeneous data, and an
extensible user interface architecture.

A further objective was to gather a high-quality dataset.  Using the
\textit{bucika\_gsr} platform, we conducted a controlled experiment where human
participants underwent a standardised stress induction protocol (e.g., a
time-pressured mental arithmetic task).  Throughout each session, the system logged
three synchronised data streams: continuous GSR signals, thermal video, and RGB video.
The ultimate objective is to analyse this multimodal dataset.  By examining the
time-synchronised recordings, we can directly compare the GSR readings with visual
data to determine what correlates of stress are present in the videos and quantify
the limits of video-only stress assessment.

\section{Thesis Structure and Scope}

The document is organised as follows: Chapter~\ref{chap:2} reviews the multimodal
physiological data collection platform, related work, the psychophysiology of stress
responses, and recent advances in contactless physiological monitoring.
Chapter~\ref{chap:3} defines the requirements and system analysis.
Chapter~\ref{chap:4} details the design and implementation of the
\textit{bucika\_gsr} platform.  Chapter~\ref{chap:5} covers the evaluation and
testing methodology and data analysis.  Finally, Chapter~\ref{chap:6} concludes the
thesis, discussing the findings concerning the research question, the limitations of
the current approach, and potential directions for future research.

% Bibliography (minimal for standalone)
\bibliographystyle{ieeetr}
\bibliography{references}

\end{document}
