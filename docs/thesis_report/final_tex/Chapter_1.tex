\label{chap:1}
\chapter{Chapter 1: Introduction}

\section{Background and Motivation}

Stress is a ubiquitous physiological and psychological response with profound implications for human-computer interaction (HCI), health monitoring, and emotion recognition. In contexts ranging from adaptive user interfaces to mental health assessment, the ability to measure a user's stress level reliably and unobtrusively is highly valuable. Galvanic Skin Response (GSR), also known as electrodermal activity, is a well-established index of stress and arousal, reflecting changes in sweat gland activity via skin conductance measurements \cite{Boucsein2012}. Traditional GSR monitoring techniques, however, rely on attaching electrodes to the skin (typically on the fingers or palm) to sense minute electrical conductance changes \cite{Fowles1981}. While effective in controlled laboratory settings, this contact-based approach carries significant drawbacks: the physical sensors can be obtrusive and uncomfortable, often altering natural user behavior and emotional state \cite{Cacioppo2007}. In other words, the very act of measuring stress via contact sensors may itself induce stress or otherwise confound the measurements, raising concerns about ecological validity in HCI and ambulatory health scenarios \cite{Wilhelm2010}. Moreover, contact sensors tether participants to devices, limiting mobility and making longitudinal or real-world monitoring cumbersome.

These limitations motivate the pursuit of contactless stress measurement methods that can capture stress-related signals without any physical attachments, thereby preserving natural behavior and comfort. Recent advances in sensing and computer vision suggest that it may be feasible to infer physiological stress responses using ordinary cameras and imaging devices, completely bypassing the need for electrode contact \cite{Picard2001}. Prior work in affective computing and physiological computing has demonstrated that various visual cues---facial expressions, skin pallor, perspiration, subtle head or body movements---can correlate with emotional arousal and stress levels \cite{Healey2005}. Thermal infrared imaging of the face, for instance, can reveal temperature changes associated with blood flow variations under stress (e.g., cooling of the nose tip due to vasoconstriction) in a fully non-contact manner. Likewise, high-resolution RGB video can capture heart rate or breathing rate through imperceptible skin color fluctuations and movements, as shown in emerging remote photoplethysmography techniques \cite{Poh2010}.

\section{Research Problem and Objectives}

\subsection{Problem Context and Significance}

The developments in contactless sensing raise a critical research question at the intersection of computer vision and psychophysiology: Can we approximate or even predict a person's GSR-based stress measurements using only contactless video data from an RGB camera? In other words, does a simple video recording of an individual contain sufficient information to estimate their physiological stress response, obviating the need for dedicated skin contact sensors? Answering this question affirmatively would have far-reaching implications. It would enable widely accessible stress monitoring using ubiquitous smartphone or laptop cameras and allow for the seamless integration of stress detection into everyday human-computer interactions and health monitoring applications, all without the burden of wearables or electrodes.

\subsection{Aim and Specific Objectives}

To investigate this research question, this project aims to develop and validate a system for contactless stress measurement. The primary objectives are to build a multi-modal data acquisition platform, use it to conduct a controlled stress-induction experiment, and analyze the resulting data to model GSR from video features.

To achieve this, we first developed a multi-sensor data acquisition platform, named \textit{bucika\_gsr}. The system architecture spans two tightly integrated components: a custom Android mobile application and a desktop PC application. The Android app operates on a modern smartphone equipped with an attachable thermal camera module and simultaneously captures both thermal and standard high-definition RGB video. Complementing this, the desktop PC application functions as the master controller, connecting via Bluetooth to a Shimmer3 GSR+ sensor to record the participant's ground-truth skin conductance in real time.

A key objective was to ensure precise temporal alignment. The Android and PC components communicate over a wireless network, following a master--slave synchronization protocol. The PC controller orchestrates the timing of recordings across all devices, achieving timestamp synchronization on the order of a few milliseconds. Such tight synchronization is crucial for our research, as it enables frame-by-frame correlation of physiological signals with visual cues captured on video \cite{Gravina2017}. The development of this platform involved several technical contributions, including a real-time multi-device synchronization mechanism, an integrated framework for capturing heterogeneous data, and an extensible user interface architecture.

A further objective was to gather a high-quality dataset. Using the \textit{bucika\_gsr} platform, we conducted a controlled experiment where human participants underwent a standardized stress induction protocol (e.g., a time-pressured mental arithmetic task). Throughout each session, the system logged three synchronized data streams: continuous GSR signals, thermal video, and RGB video. The final objective is to analyze this multi-modal dataset. By examining the time-synchronized recordings, we can directly compare the GSR readings with visual data to determine what correlates of stress are present in the videos and quantify the limits of video-only stress assessment.

\section{Thesis Structure and Scope}

This thesis addresses a critical gap in physiological computing by exploring a contactless approach to stress measurement. The remainder of this document is organized as follows: Chapter~\ref{chap:2} reviews the multi-modal physiological data collection platform and related work, including the psychophysiology of stress responses and recent advances in contactless physiological monitoring. Chapter~\ref{chap:3} defines the requirements and system analysis. Chapter~\ref{chap:4} details the design and implementation of the \textit{bucika\_gsr} platform. Chapter~\ref{chap:5} covers the evaluation and testing methodology and data analysis. Finally, Chapter~\ref{chap:6} concludes the thesis, discussing the findings with respect to the research question, the limitations of the current approach, and potential directions for future research.
