\label{chap:1}

\chapter{Chapter 1: Introduction}

\section{1.1 Motivation and Research Context}

In recent years, there has been growing interest in \textbf{physiological
computing} --- the use of bodily signals to infer a person's internal
states for health monitoring, affective computing, and human-computer
interaction. One physiological signal that has proven especially
valuable is the \textbf{Galvanic Skin Response (GSR)} (also known as
electrodermal activity or skin conductance). GSR measures subtle changes
in the skin's electrical conductance caused by sweat gland activity,
which is directly modulated by the sympathetic nervous
system\cite{Boucsein2012}.
Because these changes are involuntary and reflect emotional arousal and
stress, GSR is widely regarded as a reliable indicator of autonomic
nervous system
activity\cite{Boucsein2012}.
Applications of GSR span clinical psychology (e.g. biofeedback therapy
and polygraph testing) and user experience research, where it can reveal
unconscious stress or emotional responses. Even consumer technology has
begun to leverage skin conductance: modern wearable devices (e.g. recent
smartwatches by Apple and Samsung) incorporate sensors for continuous
stress monitoring based on GSR or related
metrics\cite{AppleHealthWatch2019}\cite{SamsungHealth2020}.
This surge of interest underscores the \textit{motivation} to harness
physiological signals like GSR in everyday contexts.

Despite its value, traditional GSR measurement requires skin-contact
electrodes (typically attached to fingers or palms with conductive
gel)\cite{Fowles1981}.
This method is inherently obtrusive --- the wires and electrodes can
restrict natural movement and comfort, and long-term use may cause
discomfort or skin
irritation\cite{Healey2005}\cite{Picard2001}.
These practical limitations make it difficult to use GSR in natural,
real-world settings outside the lab. Consequently, \textbf{contactless
measurement techniques} for GSR have become an appealing research
direction\cite{DriverStressThermal2020}.
The idea is to infer GSR (or the underlying psychophysiological arousal)
using remote sensors that do not require physical contact with the user.
For example, thermal infrared cameras can detect subtle temperature
changes on the skin surface due to blood flow and perspiration, offering
a proxy for stress-induced
responses\cite{GSRFacialThermal2021}.
Facial infrared imaging has shown promise as a complementary measure in
emotion research, capitalizing on the fact that stress and
thermoregulation are linked (e.g. perspiration causes evaporative
cooling)\cite{StressDefinitionHH}.
Similarly, high-resolution RGB cameras with advanced computer vision
algorithms can non-invasively capture other physiological signals ---
prior work has demonstrated heart rate and breathing can be measured
from video of a person's face or
body\cite{CortisolStressIndicator2020}\cite{WHOStressDefinition}.
These developments suggest that \textit{multi-modal sensing}, combining
traditional biosensors with imaging, could enable \textbf{contactless
physiological monitoring} in the future. Research in affective
computing increasingly points to the benefit of fusing multiple
modalities (e.g. GSR, heart rate, facial thermal signals) to more
robustly capture emotional or stress
states\cite{Boucsein2012}\cite{CortisolStressIndicator2020}.

However, realizing such a vision requires overcoming significant
challenges. A key \textit{research gap} is the lack of an integrated platform
to collect and synchronize these diverse data streams. Most prior
studies have tackled contactless GSR estimation in isolation or under
highly controlled conditions, often using separate devices that are not
synchronized in real
time\cite{ElectrodermalActivityWiki}\cite{DeviceServer}.
For instance, thermal cameras and wearable GSR sensors have typically
been used independently, with any fusion of their data done post hoc.
This piecemeal approach complicates the development of machine learning
models, which require well-aligned datasets of inputs (e.g.
video/thermal data) and ground truth outputs (measured GSR). There is a
clear need for a \textbf{multi-modal data collection platform} that can
simultaneously record GSR signals alongside other sensor modalities in a
\textit{synchronized} manner. Such a platform would enable researchers to
gather rich, time-aligned datasets --- for example, thermal video of a
participant's face recorded in lockstep with their GSR signal --- thereby
laying the groundwork for training and validating predictive models that
infer GSR from alternative sensors. \textbf{The primary contribution of this
thesis is the development of precisely such a platform:} a modular,
multi-sensor system for synchronized physiological data acquisition
geared toward future GSR prediction research. In summary, the motivation
behind this work stems from recent trends in physiological computing and
multimodal sensing, and the recognized need for robust, synchronized
datasets to advance \textit{contactless} GSR measurement.

\section{1.2 Research Problem and Objectives}

Given the above context, the \textbf{research problem} can be stated as
follows: \textit{there is currently no readily available system that enables
synchronized collection of GSR signals together with complementary data
streams (such as thermal and visual data) in naturalistic settings,
which hinders the development of machine learning models for contactless
GSR prediction}. While traditional GSR sensors provide reliable
ground-truth measurements, they are intrusive for real-world use, and
purely contactless approaches remain unvalidated or
imprecise\cite{ElectrodermalActivityWiki}.
To bridge this gap, researchers require a platform that can record
\textbf{multiple modalities simultaneously} --- for example, capturing a
person's skin conductance with a wearable sensor while concurrently
recording thermal camera footage and standard video. Crucially, all data
must be time-synchronized with high precision to allow meaningful
correlation and learning. The absence of such an integrated system forms
the core problem that this thesis addresses.

The \textit{objective} of this research, therefore, is to design and implement
a \textbf{multi-modal physiological data collection platform} that enables
the creation of a synchronized dataset for future GSR prediction models.
Unlike end-user applications or final predictive systems, the focus here
is on the data acquisition infrastructure --- in other words, building
the \textit{foundation} upon which real-time GSR inference algorithms can later
be developed. It is important to clarify that \textbf{real-time GSR prediction
is not within the scope of this thesis}. Instead, the aim is to
facilitate future machine learning by providing a robust means to gather
ground-truth GSR and candidate predictor signals in unison. The
following specific objectives have been defined to achieve this aim:

\begin{itemize}
\item \textbf{Objective 1: Multi-Modal Platform Development.} \textit{Design and develop
  a modular data acquisition system capable of recording synchronized
  physiological and imaging data.} This involves integrating a
  \textbf{wearable GSR sensor} and \textbf{camera-based sensors} into one
  platform. In practice, the system will use a research-grade Shimmer3
  GSR+ device for ground-truth skin conductance
  measurement\cite{GSRPPGMachineLearning2024},
  a thermal infrared camera (Topdon TC001) attached to a smartphone for
  capturing thermal
  video\cite{GSRPPGMachineLearning2024},
  and the smartphone's own RGB camera for high-resolution video. A
  \textbf{smartphone-based sensor node} will be coordinated with a \textbf{desktop
  controller} application to start/stop recordings in unison and
  timestamp data streams consistently. The architecture should ensure
  that all modalities can be recorded \textbf{simultaneously} with
  millisecond-level timestamp alignment.

\item \textbf{Objective 2: Synchronized Data Acquisition and Management.}
  \textit{Implement methods for precise time synchronization and data handling
  across devices.} A custom \textbf{control and synchronization layer}
  (developed in Python) will coordinate the sensor node(s) and ensure
  that GSR readings, thermal frames, and RGB frames are logged with
  synchronized timestamps. This objective includes establishing a
  reliable communication protocol between the smartphone and the PC
  controller to transmit control commands and streaming
  data\cite{SimulatorValidityPhysiological2025}.
  It also involves data management aspects: storing the multi-modal data
  with appropriate formats and metadata so that they can be easily
  combined for analysis. By the end, the platform should produce a
  well-synchronized dataset (e.g. timestamps of physiological samples
  aligned with video frame times) that can serve as a training corpus
  for machine learning.

\item \textbf{Objective 3: System Validation through Pilot Data Collection.}
  \textit{Evaluate the integrated platform's performance and data integrity in
  a real recording scenario.} To verify that the system meets
  research-grade requirements, a series of test recording sessions will
  be conducted. For example, pilot experiments might involve human
  participants performing tasks designed to elicit varying GSR responses
  (stress, stimuli, etc.) while the platform records all modalities. The
  \textbf{validation} will focus on checking temporal synchronization
  accuracy (e.g. confirming that events are correctly aligned across
  sensor streams) and the quality of the recorded signals (such as
  signal-to-noise ratio of GSR, resolution of thermal data, etc.). We
  will analyze the collected data to ensure that the GSR signals and the
  corresponding thermal/RGB data show the expected correlations or
  time-locked changes. Successful validation will demonstrate that the
  platform can reliably capture synchronized multi-modal data suitable
  for subsequent machine learning analysis. (Developing the predictive
  model itself is left for future work; here we concentrate on
  validating the \textit{data pipeline} that would feed such a model.)

\end{itemize}
By accomplishing these objectives, the thesis will deliver a proven
multi-sensor data collection platform that fills the current
technological gap. This platform will enable researchers to build
\textbf{multimodal datasets} for GSR prediction, accelerating progress toward
truly contactless and real-time stress monitoring systems. The emphasis
is on creating a flexible, extensible setup --- a \textbf{modular sensing
system} --- that not only integrates the specific devices in this
project (GSR sensor and thermal/RGB cameras) but can be extended to
additional modalities in the future. Ultimately, this work lays the
groundwork for future studies to train and test machine learning
algorithms that estimate GSR from camera data, by first solving the
critical challenge of \textit{acquiring synchronized ground-truth data}.

\section{1.3 Thesis Outline}

This thesis is organized into six chapters, following a logical
progression from background concepts through system development to
evaluation:

\begin{itemize}
\item \textbf{Chapter 2 --- Background and Research Context:} This chapter reviews
  the relevant literature and technical background underpinning the
  project. It discusses physiological computing and emotion recognition,
  the significance of GSR in stress research, and prior approaches to
  contactless physiological measurement. Key related works in
  \textbf{multimodal data collection} and sensor fusion are examined to
  highlight the state of the art and the gap that this research
  addresses. The chapter also introduces the rationale behind the
  selected sensors (Shimmer3 GSR+ and Topdon thermal camera) and the
  expected advantages of a multimodal approach.

\item \textbf{Chapter 3 --- Requirements Analysis:} In this chapter, the specific
  requirements for the data collection platform are defined. The
  research problem is analyzed in detail to derive both \textbf{functional
  requirements} (such as the ability to record multiple streams
  concurrently, synchronization accuracy, user interface needs for the
  recording system) and \textbf{non-functional requirements} (such as system
  reliability, timing precision, and data storage considerations).
  Use-case scenarios and user stories are presented to ground the
  requirements in practical research situations. By the end of this
  chapter, the scope of the system and the criteria for success are
  clearly established.

\item \textbf{Chapter 4 --- System Design and Architecture:} This chapter
  describes the design of the proposed multi-modal recording system. It
  presents the overall \textbf{architecture}, detailing how hardware
  components and software modules interact. Key design decisions are
  discussed, such as the choice of a distributed setup with an Android
  smartphone as a sensor hub and a PC as a central
  controller\cite{SimulatorValidityPhysiological2025}.
  The chapter covers how the \textbf{hardware integration} is achieved
  (mounting and connecting the thermal camera to the phone, Bluetooth
  pairing with the GSR sensor, etc.) and how the software is structured
  into modules for camera capture, sensor communication, network
  synchronization, and data logging. Diagrams are provided to illustrate
  the flow of data and control commands between the Android app and the
  Python desktop application. The design ensures modularity, so that
  each sensing component (thermal, RGB, GSR) can operate in sync under
  the coordination of the central controller. Important considerations
  like timestamp synchronization protocols, latency handling, and error
  recovery mechanisms are also described here.

\item \textbf{Chapter 5 --- Implementation Testing and Validation:} In this
  chapter, the focus is on evaluating the implemented platform and
  demonstrating that it meets the thesis objectives. The \textbf{evaluation
  methodology} is first outlined, including the test setup and metrics
  for assessing synchronization and data quality. Results from pilot
  recordings are then presented: for example, timing logs verifying that
  the disparity between camera frame timestamps and GSR signal
  timestamps is within acceptable bounds (on the order of milliseconds),
  and qualitative examples of data (such as parallel plots of GSR peaks
  alongside thermal video frames during a stress event). The chapter
  discusses any challenges encountered during testing --- for instance,
  connectivity issues or drift in clocks --- and how they were resolved
  or mitigated. We interpret the results to confirm that the system can
  reliably produce synchronized multi-modal datasets. This validation
  demonstrates the platform's capability to serve as a data collection
  tool for future GSR prediction research. Any limitations observed
  (such as minor synchronization offsets or sensor noise issues) are
  also noted to inform future improvements.

\item \textbf{Chapter 6 --- Conclusion and Future Work:} The final chapter
  summarizes the contributions of the thesis and reflects on the extent
  to which the objectives were achieved. The \textbf{achievements} of
  developing a working multi-modal physiological data collection
  platform are highlighted, and the significance of this platform for
  the research community is discussed. The chapter also candidly
  addresses the \textbf{limitations} of the current system (for example, if
  real-time analysis was not implemented or if certain environments were
  not tested). Finally, it outlines \textbf{future work} and recommendations
  --- including the next steps of using the collected data to train
  machine learning models for GSR prediction, improving the platform's
  real-time capabilities, and possibly extending the system with
  additional sensors (such as heart rate or respiration sensors) to
  broaden its application. By charting these future directions, the
  thesis concludes with a roadmap for transitioning from this data
  collection foundation to full-fledged \textbf{real-time GSR inference} in
  forthcoming research.

\end{itemize}
Overall, \textbf{Chapter 1} (this introduction) has set the stage by
identifying the motivation and research problem, and the subsequent
chapters proceed to address that problem through systematic development
and evaluation of the multi-modal GSR data collection platform.
Together, these chapters document the journey from concept to
realization of a synchronized sensing system that will enable advanced
research into predicting GSR from multiple sensor modalities. The
outcome is a valuable tool and dataset for the community, marking a step
toward more ubiquitous and contact-free physiological monitoring in the
future.
