\label{chap:2}

\chapter{Chapter 2. Multi-Modal Physiological Data Collection Platform for Future GSR Prediction}

\section{2.1 Emotion Analysis Applications}

Emotion recognition and stress monitoring are increasingly important across many fields, often using physiological signals like Galvanic Skin Response (GSR) to gain insight into human internal states. GSR, in particular, has a long history in psychophysiological research. By the early 1970s, over 1,500 scientific articles on GSR had been published, and it remains one of the most popular methods for investigating human emotional arousal\cite{Boucsein2012}. GSR-driven emotion analysis has broad applicability across several domains, including:

\textbf{Psychological and Clinical Research:} Psychologists use GSR to quantify emotional reactions to stimuli and to study conditions like phobias or PTSD. High GSR readings can indicate fear or stress, so therapists often monitor GSR during exposure or relaxation therapy to gauge a patient's progress\cite{AppleHealthWatch2019}\cite{SamsungHealth2020}. For example, an anxious patient might exhibit elevated GSR when facing a feared stimulus. Over the course of therapy, a reduction in that GSR response would signal desensitization and recovery progress.

\textbf{Marketing and Media Testing:} In consumer neuroscience and marketing, GSR provides an objective measure of subtle differences in product appeal or advertisement impact. Marketers track GSR to determine which ads evoke arousal and engagement, pinpointing moments that resonate or fall flat\cite{Fowles1981}\cite{Healey2005}. Similarly, media producers use GSR to test audience reactions to scenes in films or games. Spikes in GSR reveal where excitement or stress occurs at key moments, informing creative decisions.

\textbf{Human–Computer Interaction and UX:} In human-computer interaction and user experience research, GSR helps detect user frustration or cognitive load during usability testing. When a user struggles with a confusing interface or encounters an error, their stress level rises, which is reflected in an increased skin conductance reading\cite{Picard2001}. Designers leverage these insights to pinpoint problematic interface elements. In adaptive systems, real-time GSR feedback can even trigger interface adjustments to reduce user stress, resulting in more responsive and empathetic technology.

\textbf{Human–Computer Interaction and UX:} In human-computer interaction and user experience research, GSR helps detect user frustration or cognitive load during usability testing. When a user struggles with a confusing interface or encounters an error, their stress level rises, which is reflected in an increased skin conductance reading\cite{Picard2001}. Designers leverage these insights to pinpoint problematic interface elements. In adaptive systems, real-time GSR feedback can even trigger interface adjustments to reduce user stress, resulting in more responsive and empathetic technology.

These application areas underscore the critical importance of reliably detecting emotional states. They create strong motivation to develop robust data collection platforms that fuel machine learning models for stress and emotion recognition. A multimodal approach – combining \textbf{physiological signals} like GSR with \textbf{behavioral cues} such as facial expressions or thermal signatures – promises to provide richer data for these applications. The ultimate goal is to enable models to detect or predict stress accurately in natural settings. Achieving this requires complete, high-quality datasets. By capturing synchronized multimodal data, the proposed platform aims to provide the ground truth needed to train and validate advanced affective computing systems.

\section{2.2 Rationale for Contactless Physiological Measurement}

Traditional emotion detection often uses wearable sensors attached to the body to measure signals like heart rate or skin conductance. While effective, these contact-based methods can be obtrusive and may alter the user's behavior or comfort. As a result, there is strong rationale for using contactless physiological measurement techniques in stress and emotion research. A contactless approach gathers data without encumbering the subject, allowing more natural interactions and broader applicability (for example, in scenarios where wearing sensors is impractical). For instance, in automotive research, monitoring driver stress with cameras is preferable to wiring a driver with electrodes. A recent study demonstrated a non-invasive driver stress monitoring system using only thermal infrared imaging, and validated its output against traditional ECG-based stress indices\cite{DriverStressThermal2020}. The ability to assess stress through a camera without any physical contact proved to be feasible and accurate. This finding is promising for real-world driver assistance systems.

Contactless measurement is also advantageous for continuous mental health monitoring in daily life. For example, modern smartphones equipped with optical (camera) and thermal sensors can passively gauge physiological signals. Researchers have even combined a smartphone's camera-based photoplethysmography (to capture heart pulse) with a small thermal camera, enabling quick and convenient daily stress measurements\cite{GSRFacialThermal2021}. These systems demonstrate that cameras – both standard RGB and infrared – can capture proxies for vital signs (such as subtle changes in facial blood flow or skin temperature) without any body attachments. This \textbf{unobtrusiveness} reduces the burden on participants, making long-term stress tracking more acceptable and scalable.

In light of these benefits, our platform prioritizes contactless modalities alongside traditional sensors. By integrating a thermal camera (and optionally the device's own RGB camera), we obtain physiological data (such as heat patterns or heart-rate-related signals) with no additional contact points beyond a simple finger GSR sensor. This approach supports data collection in more natural environments (e.g., at work, while driving, or in everyday settings) where people might not tolerate multiple wired sensors. In summary, including contactless measurements in a multimodal platform broadens the contexts in which \textbf{high-quality stress data} can be collected. This approach ensures the platform can be used comfortably in real-world settings and easily extended for future \textbf{stress inference} applications.

\section{2.3 Definitions of "Stress" (Scientific vs. Colloquial)}

The term "stress" has distinctly different meanings in scientific contexts versus everyday conversation. \textbf{Scientifically}, stress is often defined as the body's physiological response to demands or threats. Hans Selye's classic definition frames stress as "the non-specific response of the body to any demand"\cite{StressDefinitionHH}. In this view, any physical or emotional challenge triggers a cascade of biological reactions—activation of the sympathetic nervous system and the hypothalamic-pituitary-adrenal axis—that prepare the organism to adapt. In scientific terms, a distinction is made between the \textit{stressor} (the challenging stimulus) and the \textit{stress response} (the body's reaction). Key aspects of the scientific concept include measurable changes like elevated adrenaline, \textbf{cortisol} secretion, increased heart rate, and heightened GSR—all resulting from sympathetic activation\cite{CortisolStressIndicator2020}. These changes are objectively observable indicators that an organism is under strain. Notably, stress can be positive ("eustress," which may enhance performance) or negative ("distress," which can be harmful). In both cases, it entails a departure from homeostasis and triggers the body's coping mechanisms.

\textbf{Colloquially}, however, "stress" usually refers to a subjective feeling of pressure, tension, or anxiety. In everyday usage, when someone says "I feel stressed," they typically mean they are experiencing mental or emotional strain. This informal definition aligns with, for example, the World Health Organization's description, which defines stress as a state of worry or mental tension in response to a difficult situation\cite{WHOStressDefinition}. In colloquial use, "stress" often serves as an umbrella term encompassing both the sources of stress ("I have a stressful job") and the feelings evoked ("I'm stressed out"). This everyday concept may ignore precise physiological mechanisms, focusing instead on the perceived burden or discomfort. For example, a tight deadline at work might be called "stressful" whether or not it triggers significant biological stress responses, simply because the individual \textit{feels} under pressure.

When reconciling these definitions, it is important for our research to clarify which aspect of "stress" we intend to measure. Our project is concerned with \textbf{physiological stress responses}, meaning the objective signals that accompany the stress state (such as changes in GSR, heart rate variability, and thermal variations). These signals serve as ground truth data for building predictive models. However, we also acknowledge that the \textbf{perception of stress} (the colloquial understanding) is relevant, since ultimately any automated GSR prediction system should correspond to a person's experienced stress. By aligning scientific measurements with everyday notions of stress (for example, by validating that high GSR coincides with self-reported stress levels), our platform and future models can bridge the gap between the scientific perspective and the colloquial understanding of stress.

\section{2.4 Cortisol vs. GSR as Stress Indicators}

\textbf{Cortisol} and \textbf{Galvanic Skin Response (GSR)} are both widely used indicators of stress, yet they operate via very different physiological pathways and timescales. Cortisol is a hormone released by the adrenal cortex as the end product of HPA (hypothalamic-pituitary-adrenal) axis activation during stress. It is often regarded as a "gold-standard" biochemical marker of stress, reflecting the body's hormonal stress response. For instance, acute stressors (like the Trier Social Stress Test) reliably cause a spike in cortisol roughly 20–30 minutes after the stressful event\cite{CortisolStressIndicator2020}. This delay occurs because cortisol release and distribution are relatively slow processes. Research confirms that psychological stress triggers almost immediate sympathetic reactions, whereas cortisol peaks only after a considerable lag\cite{CortisolStressIndicator2020}. Cortisol measurement (typically via saliva samples) thus provides a \textit{delayed} but specific index of stress level. High cortisol levels indicate activation of the HPA axis, which is associated with sustained stress and can have downstream effects on various organs and cognitive functions.

In contrast, \textbf{GSR responds almost instantaneously to stress} via the sympathetic nervous system. GSR (also called electrodermal activity) is governed by sweat gland activity in the skin, which increases under sympathetic activation. When an individual encounters a stressor (for example, a sudden scare or a mental challenge), the sympathetic nervous system fires within seconds, causing heart rate and sweat secretion to rise as part of the fight-or-flight response\cite{CortisolStressIndicator2020}. As a result, skin conductance begins to climb almost immediately, often within 1–3 seconds of the stimulus\cite{ElectrodermalActivityWiki}. This makes GSR an excellent \textit{real-time} indicator of arousal. For example, during a stressful task, distinctive GSR peaks can be observed at moments of heightened stress or excitement — long before any cortisol changes are measurable\cite{CortisolStressIndicator2020}. Because of this immediacy, GSR is invaluable for capturing the dynamic pattern of stress responses on a second-by-second basis.

However, there are important differences and complementary aspects between these two measures. \textbf{Cortisol} represents a downstream, cumulative stress effect—it reflects the intensity of stress exposure over minutes and is relatively \textit{specific} to true stress (since the HPA axis is activated mainly by stressors significant enough to provoke a hormonal response). It is less sensitive to brief, transient arousal that a person might not even perceive as "stressful." \textbf{GSR}, on the other hand, is a direct readout of sympathetic nervous system arousal. It is extremely sensitive, registering any kind of emotional or physical arousal (e.g., surprise, anxiety, excitement) even if those responses are mild or short-lived\cite{GSRPPGMachineLearning2024}. Thus, GSR can sometimes register false positives for "stress." For instance, excitement or startle responses produce GSR changes but would not be considered stress in the everyday sense. In summary, GSR is more of a \textit{situational marker} of arousal, whereas cortisol is a \textit{hormonal marker} of systemic stress load\cite{SimulatorValidityPhysiological2025}.

In our context of building a prediction platform, we primarily use GSR as the \textbf{ground-truth stress signal} because of its high temporal resolution and directness. The near-instantaneous changes in GSR allow synchronization with other modalities (like video frames or thermal readings) on a fine timescale. Cortisol, while not practical for real-time data collection (due to the need for sampling bodily fluids and its delayed response), provides valuable scientific validation. Indeed, one study modeled a \textit{cortisol-equivalent stress indicator} from GSR peaks and found a significant correlation with measured salivary cortisol\cite{CortisolStressIndicator2020}, suggesting that carefully processed GSR data can approximate the hormonal stress profile. This finding reinforces that GSR, despite its limitations, is a powerful proxy for stress when collected properly. In summary, cortisol and GSR each have distinct roles: cortisol underscores the biological significance of stress, whereas GSR offers an accessible, immediate window into the sympathetic activation that accompanies stress. We therefore leverage GSR as the primary stress indicator in our platform, with the understanding that it captures the rapid dynamics of stress responses that future models will aim to predict.

\section{2.5 GSR Physiology and Measurement Limitations}

\textbf{Physiology of GSR:} Galvanic Skin Response originates from the activity of eccrine sweat glands and the skin's electrical properties. When the sympathetic branch of the autonomic nervous system is aroused (for example, during stress or strong emotion), it drives the sweat glands — particularly on the palms and soles — to produce sweat\cite{ElectrodermalActivityWiki}. Even imperceptible amounts of sweat on the skin change the skin's conductivity, lowering its electrical resistance. GSR sensors typically apply a tiny constant voltage across two skin contacts and measure the conductance. An increase in conductance indicates greater sweat gland activity and thus higher sympathetic arousal\cite{GSRGuideIMotions}. This makes GSR a direct readout of physiological arousal levels. It is \textbf{entirely involuntary} — unlike facial expressions or heart rate, a person cannot consciously suppress or modulate their skin conductance. As a result, GSR is highly valued in psychophysiology because it provides an "honest" signal of emotional arousal that is not under cognitive control. Numerous studies and reviews point to electrodermal activity as a primary indicator of stress and arousal\cite{GSRPPGMachineLearning2024}. In summary, GSR's physiological basis (sweat secretion under sympathetic control) ties it closely to the body's fight-or-flight machinery, which is exactly what we seek to monitor in stress research.

\textbf{Limitations of GSR measurements:} Despite its value, GSR is not a perfect signal and comes with several important limitations and challenges\cite{ElectrodermalActivityWiki}:

\textbf{Environmental and Individual Factors:} External conditions like ambient \textbf{temperature and humidity} can significantly affect skin conductance readings\cite{ElectrodermalActivityWiki}. Heat can increase baseline skin moisture, elevating GSR even in the absence of emotional stimuli, whereas cold, dry air might suppress the sweat response. Likewise, individual physiological factors (such as a person's hydration level or certain medications like beta-blockers or SSRIs) can alter skin conductance responsiveness\cite{ElectrodermalActivityWiki}. As a result, the same stimulus might produce different GSR magnitudes across different conditions or individuals, reducing consistency. Proper experimental control or normalization is needed to account for these variables. Additionally, GSR can drift over time (as the skin gradually becomes more sweaty or drier), so interpreting absolute values requires caution.

\textbf{Sensor Placement and Response Variability:} The classic assumption is that GSR reflects a uniform "whole-body" arousal, but in reality it varies by measurement location. Measurements at different body sites (fingertip, wrist, foot, etc.) can yield different response patterns, partly because sweat glands in different regions are regulated by distinct sympathetic nerves\cite{ElectrodermalActivityWiki}. For example, the left and right hands can exhibit non-identical GSR responses to the same stimulus\cite{ElectrodermalActivityWiki}. This spatial variability means the placement of electrodes must be chosen carefully (fingers are standard since they have high sweat gland density and responsiveness). Moreover, GSR changes do not happen instantly; there is an inherent \textbf{lag of about 1–3 seconds} between a stimulus (e.g., a sudden stressor) and the rise of the GSR signal\cite{ElectrodermalActivityWiki}. This delay, caused by physiological and electrochemical processes in the skin, complicates precise alignment with fast events. It requires any data collection platform to synchronize stimulus or event timestamps with GSR data while accounting for this latency. Finally, obtaining high-quality GSR data can depend on the \textbf{skill of the operator}\cite{ElectrodermalActivityWiki} — proper skin preparation, electrode attachment, and calibration are needed to avoid motion artifacts or poor contact, which can introduce noise.

These limitations underscore why a \textbf{multimodal approach} is beneficial. By combining GSR with other signals (such as heart rate or thermal imaging), we can cross-validate and compensate in cases where GSR alone might be ambiguous or affected by external factors. In our platform, careful attention is given to data quality: we use high-grade GSR sensors for stable readings, ensure consistent placement (e.g., finger straps on the same hand for all sessions), and log environmental conditions if necessary. We also design the data acquisition with synchronization and timing in mind, so that the known GSR lag can be corrected during analysis. By recognizing GSR's limitations, we can design a data collection system (and later predictive models) that are more robust and interpretable. GSR will still serve as a core ground truth for "stress" in our dataset, but it will be interpreted alongside the other modalities to build a reliable inference model.

\section{2.6 Thermal Cues of Stress in Humans}

Beyond electrical signals like GSR, \textbf{thermal imaging} offers a contactless window into the physiological changes that occur under stress. When a person experiences stress, the autonomic nervous system not only triggers sweating, but also redistributes blood flow as part of the fight-or-flight response. One observable consequence is peripheral \textbf{vasoconstriction} — blood vessels in the face and extremities constrict, leading to cooler skin temperatures in those regions. Thermal cameras can detect these subtle temperature shifts. Research has consistently found that acute stress or fear is accompanied by a measurable drop in temperature at the tip of the nose and across parts of the face\cite{ContactlessStressThermal2022}. For example, in controlled studies where participants underwent a stress task (such as the Stroop test or public speaking), infrared thermal cameras recorded that the participants' nose-tip temperature dropped significantly during stress and then rebounded as they recovered\cite{ContactlessStressThermal2022}. This "cold nose" effect is considered a hallmark thermal signature of stress, and it is attributed to sympathetic vasoconstriction diverting blood to core organs.

In addition to these cooling effects, thermal imaging can capture signs of \textbf{stress-induced perspiration} and associated heat dissipation. According to Pavlidis et al., stress activates sweat glands especially in the \textbf{periorbital (around the eyes) and nasal regions}. This causes increased evaporation and cooling that a thermal camera can pick up as temperature fluctuations\cite{DriverStressThermal2020}. Their system, often dubbed a "StressCam," demonstrated that facial heat patterns—particularly warming due to blood flow and cooling due to evaporative sweat—correlate strongly with psychological stress levels\cite{DriverStressThermal2020}. For instance, during a sudden stress event, a transient warming can appear in the forehead (from a quick blood pressure rise) while cooling occurs around the nose and mouth (from the evaporative cooling of sweat). These patterns are \textbf{sympathetically driven}, meaning they stem from the same nervous activation that causes GSR changes\cite{DriverStressThermal2020}. Thus, they provide a complementary view of the stress response.

Thermal cues have even been used to detect concealed stress or deceit. A well-known application is lie detection, where a thermal camera can spot the "heat signature" of stress around the eyes (from blood vessel dilation) or a cooling of the nose when someone is anxious while lying. Recent advances in higher-resolution thermal imaging and computer vision have expanded analysis to multiple facial regions. Rather than relying only on the nose tip, researchers define multiple regions of interest (ROIs) across the face (forehead, cheeks, nose, periorbital area, etc.) and track how each ROI's temperature changes under stress\cite{ContactlessStressThermal2022}. This approach has revealed a complex picture. For example, one study noted that during a cognitive stress task, not only did the nose and periorbital regions cool, but the cheeks actually showed a slight increase in temperature (perhaps due to blushing or muscle activity)\cite{ContactlessStressThermal2022}. Such findings suggest that a multi-region thermal analysis can yield a rich feature set for machine learning—essentially a "thermal signature" of stress encompassing several physiological processes.

For our platform, including a thermal camera is motivated by these known thermal cues of stress. By recording thermal video of participants' faces or hands during data collection, we capture signals like nose-tip cooling and perinasal perspiration remotely, in sync with GSR. These thermal features will serve as valuable predictors of stress in future models. Importantly, they are \textbf{contactless} and non-invasive, aligning with our rationale (Section 2.2) of making data collection as natural as possible. Thermal imaging thereby provides a bridge between purely internal signals (like GSR or cortisol) and external observations. It visualizes autonomic changes on the surface of the skin, giving our multimodal dataset another dimension of ground truth for stress that can be leveraged by machine learning algorithms.

\section{2.7 RGB vs. Thermal Imaging (Machine Learning Hypothesis)}

In designing our multimodal platform for stress data, we consider both \textbf{visible spectrum (RGB) imaging} and \textbf{thermal infrared imaging} as complementary modalities. Each type of camera offers unique information. An RGB camera (like a standard smartphone camera) captures fine details of facial expressions, skin color changes, and movements, while a thermal camera captures the invisible heat patterns related to blood flow and sweat. Our central hypothesis for future \textbf{machine learning} models is that combining RGB and thermal data will yield more accurate and robust predictions of stress (or GSR levels) than either modality alone. This hypothesis is grounded in the idea that stress manifests in multiple observable ways—some best seen in the visible domain (e.g., a furrowed brow, a pale face from reduced blood flow, subtle tremors) and others detectable only in the thermal domain (e.g., a drop in skin temperature, increased heat from breath or perspiration). By fusing these, an AI model can develop a holistic picture of a person's state.

Prior work supports the advantages of this multimodal approach. For instance, researchers have built \textbf{dual-camera systems} pairing a regular RGB camera with a thermal sensor. They found that this combination dramatically increases the richness of physiological measurements available\cite{InstantStressSmartphone2019}. In one smartphone-based study, an integrated approach using the phone's camera (to measure blood volume pulse) together with an attached thermal camera (for nose-tip temperature) could quickly detect stress and yielded better classification accuracy than using a single sensor\cite{InstantStressSmartphone2019}. In that study, using both modalities improved stress inference accuracy to approximately 78\%, compared to about 68\% using only the photoplethysmography (RGB-based) data or 59\% using only thermal data\cite{InstantStressSmartphone2019}. This demonstrates a \textbf{synergy}: the errors of one modality may be compensated by the other. For example, if visible facial cues are ambiguous (say the person maintains a neutral expression), thermal cues might still reveal physiological stress. Conversely, if a thermal signal is unclear due to an external heat source, the RGB camera might capture a telltale anxious fidget or a change in complexion.

From a machine learning perspective, RGB and thermal images together provide a multi-channel input that can enable more robust feature extraction. \textbf{RGB video} frames can be processed to extract heart rate (via subtle color changes in the face), breathing rate (via chest movements), and facial action units (muscle movements indicating emotion). \textbf{Thermal video} frames can be processed to extract temperature-based features like the nose-to-face temperature gradient, the rate of thermal change, or the presence of cool spots from sweating. Our hypothesis is that a model trained on a well-synchronized dataset containing both types of data alongside ground-truth GSR will learn latent patterns that correlate with stress more strongly than either modality alone. For instance, a sudden stress event might cause a combination of cues: a facial expression change (widened eyes) together with a thermal drop in nose temperature. A multimodal model could learn this joint signature, whereas a unimodal model might catch only one of these cues and be less certain.

To facilitate this, our data collection platform is designed to record \textbf{synchronized RGB and thermal streams}. By capturing both, we ensure that for every moment in time we have aligned data: a thermal image and a corresponding RGB image (and of course the concurrent physiological readings like GSR). This alignment is crucial for training algorithms to exploit cross-modal features. It also allows us to test our hypothesis: we can train machine learning models on just RGB data, just thermal data, and then on both together, to quantitatively evaluate the benefit of multimodal integration. Based on the literature and our understanding, we anticipate that the fused model will outperform each single-modality model because the RGB and thermal modalities are not redundant but rather complementary. Ultimately, this approach aims to pave the way for \textbf{contactless stress inference}. If a model can reliably predict GSR (or stress levels) using only cameras, it could enable real-time stress monitoring with everyday devices. Thus, this section underlines the theoretical foundation for including both imaging modalities in the platform and guides our plan for future machine learning experiments using the collected dataset.

\section{2.8 Sensor Device Selection Rationale (Shimmer GSR Sensor and Topdon Thermal Camera)}

To realize the above goals, we carefully selected the hardware components for our multimodal data collection platform. Our sensor selection was based on signal quality, compatibility, and the ability to provide \textbf{synchronized, high-resolution data}. The platform's current configuration centers on two primary devices: the \textbf{Shimmer 3 GSR+ sensor} for electrodermal activity and the \textbf{Topdon TC001 thermal camera} for infrared imaging. We detail the rationale for each device:

\textbf{Shimmer 3 GSR+ (Galvanic Skin Response sensor):} The Shimmer GSR unit is a research-grade wearable sensor widely used in academic and clinical studies for EDA/GSR measurements. We selected Shimmer over consumer fitness devices (like smartwatches) to ensure \textbf{data accuracy and flexibility}. The Shimmer 3 GSR+ provides raw skin conductance data with high resolution and sampling rates (up to 128 Hz)\cite{GSRPPGMachineLearning2024}, far exceeding the 4–10 Hz sampling typical of wristband trackers. This high sampling rate means we capture fast phasic changes in GSR without aliasing, which is crucial for precise synchronization with video frames. Moreover, Shimmer's reliability has been demonstrated in comparative evaluations—studies comparing the Shimmer GSR sensor to popular devices (e.g., the Empatica E4 wristband or Fitbit Sense) found Shimmer data to be consistently robust and trustworthy for stress research\cite{GSRPPGMachineLearning2024}. The sensor uses Ag/AgCl electrodes attached to the fingers, providing a low-noise conductance measurement, and it interfaces via Bluetooth, streaming data in real time for synchronization. The Shimmer was also chosen for its \textbf{extensibility}: it includes additional channels (such as a photoplethysmograph (PPG) and an accelerometer) which we can utilize to collect heart rate or motion data in the future without adding another device\cite{GSRPPGMachineLearning2024}. The decision to use Shimmer ensures that our "ground truth" GSR signal is of the highest possible quality, serving as a dependable reference for training machine learning models.

\textbf{Topdon TC001 Thermal Camera (USB, Android-compatible):} For the thermal imaging component, we needed a camera that is \textbf{portable}, offers \textbf{high infrared resolution}, and can integrate with a mobile platform for synchronized recording. We selected the Topdon TC001 thermal camera after evaluating several options. It features an \textbf{IR sensor resolution of 256×192 pixels} (which can be enhanced via image processing to an equivalent 512×384 resolution)\cite{TopdonTC001}, substantially higher than many consumer thermal cameras. For comparison, the popular FLIR One Pro has a native resolution of only 160×120. This higher pixel count allows finer discrimination of small temperature differences on the face or skin, improving the fidelity of stress-related thermal features. The TC001 is designed to connect directly to an Android smartphone via USB-C, essentially turning the phone into a thermal camera display and recorder. This plug-and-play compatibility was a major reason for our choice—it allowed us to integrate the thermal feed into our \textbf{Android-based data collection app} with relative ease. The camera comes with an open SDK and uses standard UVC (USB Video Class) protocols, meaning we can programmatically control it and capture frames in sync with other sensors. Additionally, the Topdon camera operates at a decent frame rate (up to approximately 25–30 frames per second), enabling us to capture fluid thermal video of physiological changes. We also considered the device's \textbf{calibration and accuracy}: the TC001 has an optimized temperature accuracy and includes a calibration shutter, which helps maintain accurate absolute temperature readings across sessions (useful if we need actual temperature values for analysis, not just relative changes). Practically speaking, the Topdon offered the best trade-off between cost and performance for our academic project—it is more affordable than high-end FLIR cameras but still delivers high-quality data. By using this camera, we ensure that our platform's thermal channel is rich enough for detailed analysis of stress patterns (like those discussed in Section 2.6).

\textbf{Synchronization and Integration:} A critical aspect of using these devices together is achieving precise time alignment. The Shimmer GSR sensor provides timestamps for each data point and the Android device hosting the thermal camera can timestamp each frame; we implemented a synchronization mechanism (a master clock in the recording app) to align the streams. This way, we can correlate each GSR peak with the exact thermal image frames (and any RGB frames, if using the phone camera) around that moment. The importance of synchronization cannot be overstated—misaligned data could lead to incorrect labeling (e.g., attributing a GSR surge to the wrong facial expression). Our platform uses a \textbf{common time base} and logging system to ensure all modalities (GSR, thermal, and any others) are recorded in lockstep. Early development included calibration routines where we trigger known events (like a LED flash visible in both thermal and RGB, or a manual signal causing a GSR spike) to measure and correct any offsets between sensors.

\textbf{Extensibility:} The chosen sensor set is meant to be extensible. The \textbf{Android smartphone} that acts as the hub can also record \textbf{RGB video} from its built-in camera simultaneously, adding an additional modality (this was discussed in Section 2.7). We can enable or disable this as needed. The hardware and software design allows adding further sensors such as a heart rate chest strap or a respiration belt, as long as they can interface via Bluetooth or USB to the same system—future researchers or developers can plug in new data streams and have them synchronized with the existing ones. The Shimmer's modular nature (it can be fitted with other sensing modules like ECG or EMG) and the Android platform's connectivity mean the \textbf{multi-modal platform can grow} to incorporate new physiological signals or environmental sensors with minimal changes. This extensibility supports our central motivation: to create a \textbf{synchronized, high-quality multimodal dataset} that is \textit{future-proof} for various machine learning modeling efforts. Whether the goal is to predict GSR from thermal images, to classify stress vs. no-stress from all modalities, or to explore new physiological correlations, the platform provides a flexible foundation.

In conclusion, the combination of the Shimmer GSR sensor and the Topdon thermal camera was deliberate to ensure we capture \textbf{ground-truth stress signals (GSR) alongside rich, contactless indicators (thermal imagery)}. By using research-grade and high-resolution devices, we maximize data quality. By focusing on synchronization and extensibility, we ensure the data is \textbf{machine-learning ready}—correctly aligned and scalable. Every section in this chapter has underscored that our aim is not real-time inference for its own sake, but rather the \textbf{collection of robust, ground-truth aligned multimodal data}. The rationale behind each component choice is ultimately to serve that aim, yielding a platform capable of underpinning advanced GSR prediction models in the future. The next steps will involve deploying this platform in experimental settings, collecting a complete dataset, and then utilizing it to train and evaluate the machine learning models that motivated its creation.


