\label{chap:2}
\chapter{Chapter 2: Multi-Modal Physiological Data Collection Platform for Future GSR Prediction}

\section{2.1 Emotion Analysis Applications}

Emotion recognition and stress monitoring leverage physiological signals such as \textbf{GSR} for insight into human states. GSR is extensively used in psychophysiological research, with applications across diverse fields:

\begin{itemize}
\item \textbf{Psychological and Clinical Research:} Psychologists use GSR to quantify emotional reactions and understand conditions like phobias or PTSD. Therapists monitor GSR during therapy to gauge treatment progress.

\item \textbf{Marketing and Media Testing:} Marketers track GSR to measure advertisement impact and audience engagement.

\item \textbf{Human-Computer Interaction and UX:} GSR detects user frustration or cognitive load in usability studies. Real-time GSR feedback can trigger interface adjustments to reduce user stress.
\end{itemize}

These applications underscore the need for robust data collection platforms to fuel machine learning models for stress recognition. A \textbf{multi-modal approach} combining physiological signals with behavioral cues promises richer data for accurate stress detection in natural settings.

\section{2.2 Rationale for Contactless Physiological Measurement}

Traditional emotion detection often relies on wearable sensors (for heart rate, skin conductance, etc.) attached to the user.

While effective, these contact-based methods can be obtrusive and may alter the user's behavior or comfort.

There is a strong rationale for \textbf{contactless physiological measurement}
 techniques in stress and emotion research.

A contactless approach allows data to be gathered without encumbering the subject, enabling more natural interactions and broader applicability (e.g. in scenarios where wearing sensors is impractical).

For instance, in automotive research, monitoring driver stress with cameras is preferable to wiring the driver with electrodes.

A recent study demonstrated a non-invasive driver stress monitoring system using only thermal infrared imaging, validating its output against traditional ECG-based stress indices.

The ability to assess stress state through a camera, without any physical contact, was shown to be feasible and accurate, which is promising for real-world driver assistance systems.

Contactless measurement is also advantageous for continuous mental health monitoring in daily life.

Modern smartphones equipped with optical and thermal sensors can passively gauge physiological signals.

Researchers have combined smartphone camera photoplethysmography (for heart pulse) with a small thermal camera to quickly detect stress responses, aiming for quick and convenient daily measurements.

Such systems highlight that cameras (both regular RGB and infrared) can capture proxies for vital signs --- for example, subtle changes in facial blood flow or temperature --- without any attachments on the body.

This \textbf{unobtrusiveness}
 reduces the burden on participants and makes long-term stress tracking more acceptable and scalable.

Given these benefits, our platform prioritizes contactless modalities alongside traditional sensors.

By integrating a thermal camera and optionally the device's own RGB camera, we obtain physiological data (like heat patterns or heart-rate-related signals) without additional contact points beyond a simple finger GSR sensor.

This approach supports data collection in more natural environments (e.g. workplace, driving, or everyday settings) where people might not tolerate multiple wired sensors.

In summary, the rationale for including contactless measurement in a multimodal platform is to broaden the contexts in which \textbf{high-quality stress data}
 can be collected, ensuring the platform can be used comfortably and extensibly for future real-world \textbf{stress inference}
 applications.

\section{2.3 Definitions of "Stress" (Scientific vs. Colloquial)}

The term "stress" carries distinct meanings in scientific literature versus everyday conversation.

\textbf{Scientifically}
, stress is often defined in terms of the body's physiological response to demands or threats.

Hans Selye's classic definition frames stress as "the non-specific response of the body to any demand".

In this view, any challenge --- whether physical or emotional --- triggers a cascade of biological reactions (activation of the sympathetic nervous system and the hypothalamic-pituitary-adrenal axis) that prepare the organism to adapt.

Scientific discussions of stress distinguish the \textit{stressor} (the challenging stimulus) from the \textit{stress response} (the body's reaction).

Key aspects of the scientific concept include measurable changes such as elevated adrenaline, \textbf{cortisol}
 secretion, increased heart rate, and heightened GSR due to sympathetic activation.

These are objectively observable indicators that an organism is under strain.

Notably, stress can be positive (eustress, which can enhance performance) or negative (distress, which can be harmful), but in both cases it involves a departure from homeostasis and activation of coping mechanisms.

\textbf{Colloquially}
, however, "stress" usually refers to a subjective feeling of pressure, tension, or anxiety.

In everyday usage, someone saying "I feel stressed" typically means they are experiencing mental or emotional strain.

This informal definition aligns with descriptions like that of the World Health Organization, which defines stress as a state of worry or mental tension in response to a difficult situation.

Colloquial stress is often used as an umbrella term encompassing both the sources of stress ("I have a stressful job") and the feelings evoked ("I'm stressed out").

It may ignore the precise physiological mechanisms, focusing instead on the perceived burden or discomfort.

For example, a tight deadline at work might be called "stressful" whether or not it triggers significant biological stress responses, because the individual \textit{feels} under pressure.

In reconciling these definitions, it is important for our research to clarify what aspect of "stress" we aim to measure.

Our project is concerned with \textbf{physiological stress responses}
 --- the objective signals such as GSR changes, heart rate variability, and thermal variations that accompany the stress state.

These signals provide ground truth data for building predictive models.

However, we also acknowledge that the \textbf{perception of stress}
 (as understood colloquially) is relevant, since ultimately any automated GSR prediction system should correlate with a person's experienced stress.

By aligning scientific measurements with everyday notions (e.g. validating that high GSR coincides with self-reported stress levels), our platform and future models can bridge the gap between the scientific and colloquial understanding of stress.

\section{2.4 Cortisol vs. GSR as Stress Indicators}

 \textbf{Cortisol}
 and \textbf{Galvanic Skin Response (GSR)}
 are both widely used indicators of stress, but they represent very different physiological pathways and timescales.

Cortisol is a hormone released by the adrenal cortex as the end product of the hypothalamic-pituitary-adrenal (HPA) axis activation during stress.

It is often regarded as a "gold-standard" biochemical marker of stress, reflecting the body's hormonal stress response.

For instance, acute stressors (like the Trier Social Stress Test) reliably cause a spike in cortisol about 20---30 minutes after the stressful event.

This delay occurs because cortisol release and distribution are relatively slow processes: research confirms that psychological stress triggers almost immediate sympathetic reactions, whereas cortisol peaks only after a considerable lag.

Cortisol measurement (typically via saliva samples) thus provides a \textit{delayed} but specific index of stress level.

High cortisol levels indicate activation of the HPA axis, which is associated with sustained stress and can have downstream effects on various organs and cognitive functions.

In contrast, \textbf{GSR responds almost instantaneously to stress}
 via the sympathetic nervous system.

GSR (or electrodermal activity) is controlled by sweat gland activity in the skin, which increases under sympathetic drive.

The moment an individual encounters a stressor --- e.g. a sudden scare or mental challenge --- their sympathetic nervous system fires within seconds, causing heart rate and sweat secretion to rise as part of the fight-or-flight response.

As a result, skin conductance begins to climb almost immediately, often within 1---3 seconds of a stimulus.

This makes GSR an excellent \textit{real-time} indicator of arousal.

For example, during a stressful task, one can observe distinctive GSR peaks corresponding to moments of heightened stress or excitement, long before any cortisol changes would be measurable.

Because of this immediacy, GSR is invaluable for capturing the dynamic pattern of stress responses on a second-by-second basis.

However, there are important distinctions and complementary aspects between these two measures.

\textbf{Cortisol}
 represents a downstream, cumulative stress effect --- it reflects the intensity of stress exposure over minutes and is relatively \textit{specific} to true stress (since the HPA axis is chiefly activated by stressors threatening enough to warrant a hormonal response).

It is less sensitive to brief, transient arousal that might not be subjectively perceived as "stressful." \textbf{GSR}
, on the other hand, is a direct readout of sympathetic nervous system arousal.

It is extremely sensitive, registering any kind of emotional or physical arousal (e.g. surprise, anxiety, excitement) even if those responses are mild or short-lived(see implementation details in Appendix~F).

Thus, GSR can sometimes register false positives for "stress" (for instance, excitement or startle responses produce GSR changes but might not be considered stress in the colloquial sense).

GSR is more of a \textit{situational marker} of arousal, while cortisol is a \textit{hormonal marker} of systemic stress load.

In our context of building a prediction platform, we primarily use GSR as the \textbf{ground-truth stress signal}
 due to its high temporal resolution and directness.

The near-instantaneous changes in GSR allow synchronization with other modalities (like video frames or thermal readings) on a fine timescale.

Cortisol, while not practical for real-time data collection (given the need for sampling bodily fluids and the latency of response), provides valuable scientific validation.

Indeed, one study modeled a \textit{cortisol-equivalent stress indicator} from GSR peaks and found significant correlation with measured salivary cortisol, suggesting that carefully processed GSR data can approximate the hormonal stress profile.

This reinforces that GSR, despite its limitations, is a powerful proxy for stress when collected properly.

In summary, cortisol and GSR each have roles: cortisol underscores the biological significance of stress, whereas GSR offers an accessible, immediate window into the sympathetic activation that accompanies stress.

Our platform leverages GSR as the primary stress indicator, with the understanding that it captures the fast dynamics of stress responses which future models will aim to predict.

\section{2.5 GSR Physiology and Measurement Limitations}

 \textbf{Physiology of GSR:}

Galvanic Skin Response is rooted in the activity of eccrine sweat glands and the skin's electrical properties.

When the sympathetic branch of the autonomic nervous system is aroused (for example, during stress or strong emotion), it drives the sweat glands----particularly on the palms and soles----to produce sweat(see implementation details in Appendix~F).

Even imperceptible amounts of sweat in the skin change the skin's conductivity (lowering its electrical resistance).

GSR sensors typically apply a tiny constant voltage across two skin contacts and measure the conductance; an increase in conductance indicates greater sweat gland activity and thus higher sympathetic arousal.

This makes GSR a direct readout of physiological arousal levels.

It is \textbf{entirely involuntary}
 --- unlike facial expressions or heart rate, one cannot consciously suppress or modulate their skin conductance.

This is why GSR is prized in psychophysiology: it offers an "honest" signal of emotional arousal that is not under cognitive control.

Numerous studies and reviews acknowledge electrodermal activity as a primary indicator of stress and arousal.

In summary, GSR's physiological basis (sweat secretion under sympathetic control) ties it closely to the fight-or-flight machinery of the body, which is exactly what we seek to monitor in stress research.

\textbf{Limitations of GSR measurements:}

Despite its value, GSR is not a perfect signal and comes with several important limitations and challenges: \begin{itemize}
 
\item \textbf{Environmental and Individual Factors:}

External conditions like ambient \textbf{temperature and humidity}
 can significantly affect skin conductance readings.

Heat can increase baseline skin moisture, elevating GSR even without emotional stimuli, while cold dry air might suppress sweat response.

Likewise, individual physiological factors --- such as a person's level of \textbf{hydration}
, or if they are on certain \textbf{medications}
 (e.g. beta-blockers or SSRIs) --- can alter skin conductance responsiveness.

This means the same stimulus might produce different GSR magnitudes in different conditions or people, reducing consistency.

Proper experimental control or normalization is needed to account for these influences.

Additionally, GSR can drift over time (skin becomes gradually sweatier or drier), so interpreting absolute values requires caution.

\item \textbf{Sensor Placement and Response Variability:}

The classic assumption is that GSR reflects a uniform "whole-body" arousal, but in reality it \textbf{varies by location}
.

Measurements on different body sites (fingertip, wrist, foot, etc.) can yield different response patterns, partly because different regions' sweat glands are regulated by different sympathetic nerves.

For example, the left and right hands can show non-identical GSR responses to the same stimulus.

This spatial variability means placement of electrodes must be chosen carefully (fingers are standard due to high sweat gland density and responsiveness).

Moreover, GSR changes do not happen instantaneously; there is an inherent \textbf{lag of about 1---3 seconds}
 between a stimulus (e.g. a sudden stressor) and the rise of the GSR signal.

This delay, due to physiological and electrochemical processes in the skin, complicates precise alignment with fast events.

It requires any data collection platform to synchronize stimulus/event timestamps with GSR data while accounting for this latency.

Finally, obtaining high-quality GSR data can depend on the \textbf{skill of the operator}
 --- proper skin preparation, electrode attachment, and calibration are needed to avoid motion artifacts or poor contact, which can introduce noise.

\end{itemize}

These limitations underscore why a \textbf{multimodal approach}
 is beneficial.

By combining GSR with other signals (such as heart rate or thermal imaging), we can cross-validate and compensate for cases when GSR alone might be ambiguous or affected by external factors.

In our platform, careful attention is given to data quality: we use high-grade GSR sensors (for stable readings), ensure consistent placement (finger straps on the same hand for all sessions), and log environmental conditions if necessary.

We also design the data acquisition with synchronization and timing in mind, so the known GSR lag can be corrected in analysis.

Recognizing GSR's limitations allows us to design a collection system --- and later, predictive models --- that are more robust and interpretable.

GSR will serve as a core ground truth for "stress" in the dataset, but it will be interpreted in context with the other modalities to build a reliable inference model.

\section{2.6 Thermal Cues of Stress in Humans}

Beyond electrical signals like GSR, \textbf{thermal imaging}
 offers a contactless window into physiological changes under stress.

When a person experiences stress, the autonomic nervous system not only triggers sweating but also redistributes blood flow as part of the fight-or-flight response.

One observable consequence is peripheral \textbf{vasoconstriction}
 --- blood vessels in the face and extremities may constrict, leading to cooler skin temperatures in those regions.

Thermal cameras can detect these subtle temperature shifts.

Research has consistently found that acute stress or fear is accompanied by a measurable drop in temperature at the tip of the nose and across parts of the face.

For example, in controlled studies where participants underwent a stress task (like the Stroop test or public speaking), infrared thermal cameras recorded that the participants' nose tip temperature decreased significantly during stress, then rebounded as they recovered.

This "cold nose" effect is considered a hallmark thermal signature of stress and is attributed to sympathetic vasoconstriction diverting blood to core organs.

In addition to cooling effects, thermal imaging can capture signs of \textbf{stress-induced perspiration}
 and related heat dissipation.

A prominent finding by Pavlidis et al. is that stress activates sweat glands especially in the \textbf{periorbital (around the eyes) and nasal regions}
, leading to increased evaporation and cooling that a thermal camera can pick up as temperature fluctuations.

Their system, often dubbed a "StressCam," showed that the heat patterns on the face --- particularly the warming from blood flow and the cooling from evaporative sweat --- correlate strongly with psychological stress levels.

For instance, during a sudden stress event, one might observe a transient warming in the forehead (from a quick blood pressure rise) but a cooling around the nose and mouth (from evaporative cooling of sweat).

These patterns are \textbf{sympathetically driven}
, meaning they stem from the same nervous activation that causes GSR changes.

Thus, they provide a complementary view of the stress response.

Thermal cues have been used to detect concealed stress or even deceit; a well-known application is lie detection, where a thermal camera can spot the "heat signature" of stress around the eyes (from blood vessel dilation) or the cooling of the nose when a person is under the anxiety of lying.

Recent advances in higher-resolution thermal imaging and computer vision have expanded the analysis to multiple facial regions.

Rather than relying only on the nose tip, researchers define regions of interest (ROIs) across the face (forehead, cheeks, nose, periorbital area, etc.) and track how each ROI's temperature changes under stress.

This approach has revealed a complex picture: for instance, one study noted that during a cognitive stress task, not only did nose and periorbital regions cool, but the cheeks actually showed a slight increase in temperature (perhaps due to blushing or muscle activity).

Such findings suggest that a multi-region thermal analysis can yield a rich feature set for machine learning --- essentially a "thermal signature" of stress encompassing several physiological processes.

For our platform, the inclusion of a thermal camera is driven by these known thermal cues of stress.

By recording thermal video of a participant's face or hands during data collection, we capture signals like nose-tip cooling and perinasal perspiration remotely, in sync with GSR.

These thermal features will serve as valuable predictors for stress in future models.

Importantly, they are \textbf{contactless}
 and non-invasive, aligning with our rationale (Section 2.2) to make the data collection as natural as possible.

Thermal imaging thereby provides a bridge between purely internal signals (like GSR or cortisol) and external observations --- it visualizes the autonomic changes on the surface of the skin, giving our multimodal dataset another dimension of ground truth for stress that can be leveraged by machine learning algorithms.

\section{2.7 RGB vs. Thermal Imaging (Machine Learning Hypothesis)}

In designing a multimodal platform for stress data, we consider both \textbf{visible spectrum (RGB) imaging}
 and \textbf{thermal infrared imaging}
 as complementary modalities.

Each type of camera offers unique information: an RGB camera (like a standard smartphone camera) captures fine details of facial expression, skin color changes, and movements, while a thermal camera captures the invisible heat patterns related to blood flow and sweat.

A central hypothesis for future \textbf{machine learning}
 models is that combining RGB and thermal data will yield more accurate and robust predictions of stress (or GSR levels) than either modality alone.

This is grounded in the idea that stress manifests in multiple observable ways --- some best seen in the visible domain (e.g. a furrowed brow, a pale face due to reduced blood flow, or subtle tremors), and others only detectable thermally (e.g. temperature drop on the skin, increased heat from breath or perspiration).

By fusing these, an AI model can develop a holistic picture of the person's state.

Prior work supports this multimodal advantage.

For instance, researchers have built \textbf{dual-camera systems}
 that pair a regular camera with a thermal sensor and found that the combination dramatically increases the richness of physiological measurements available.

A smartphone-based study reported that an integrated approach (using the phone's camera for imaging blood volume pulse and an attached thermal camera for nose-tip temperature) could quickly detect stress and produced better classification accuracy than single sensors.

In that study, using both modalities improved stress inference accuracy to \~78%, compared to \~68% using only the photoplethysmography (RGB-based) data or \~59% using only thermal data.

This demonstrates \textbf{synergy}
: the errors of one modality may be compensated by the other.

For example, if visible facial cues are ambiguous (person maintains a neutral expression), thermal cues might still reveal physiological stress, and vice versa (if a thermal signal is unclear due to an external heat source, the RGB camera might capture a telltale anxious fidget or change in complexion).

From a machine learning perspective, RGB and thermal images together provide a multi-channel input that can enable more robust feature extraction.

\textbf{RGB video}
 frames can be processed to extract heart rate (via subtle color changes in the face), breathing rate (via chest movements), and facial action units (muscle movements indicating emotion).

\textbf{Thermal video}
 frames can be processed to extract temperature-based features like the nose-facial temperature gradient, rate of thermal change, or the presence of cool spots from sweating.

Our hypothesis is that a model trained on a well-synchronized dataset of both types of data alongside ground-truth GSR will learn latent patterns that correlate with stress more strongly than either alone.

For instance, a sudden stress event might cause a combination of cues: a facial expression change (widened eyes) and a thermal drop in nose temperature.

A multimodal model could learn this joint signature whereas a unimodal model might catch only one and be less certain.

To facilitate this, our data collection platform is designed to record \textbf{synchronized RGB and thermal streams}
.

By capturing both, we ensure that for every moment in time, we have aligned data: a thermal image and a corresponding RGB image (and of course the physiological readings like GSR).

This alignment is crucial for training algorithms to exploit cross-modal features.

It also allows us to test the hypothesis: we can train machine learning models on just RGB data, just thermal data, and the combination, to quantitatively evaluate the benefit of multi-modal integration.

Based on the literature and our understanding, we anticipate the fused model will outperform because the RGB vs.

Thermal modalities are not redundant but rather complementary.

Ultimately, this approach aims to pave the way for \textbf{contactless stress inference}
: if a model can reliably predict GSR (or stress levels) from just cameras, it could enable real-time stress monitoring using everyday devices.

Thus, Section 2.7 underlines the theoretical foundation for including both imaging modalities in the platform and guides our plan for future machine learning experiments using the collected dataset.

\section{2.8 Sensor Device Selection Rationale (Shimmer GSR Sensor and Topdon Thermal Camera)}

To realize the above goals, we carefully chose the hardware components for our multimodal data collection platform.

The selection of sensors was based on their signal quality, compatibility, and ability to provide \textbf{synchronized, high-resolution data}
.

The platform's current configuration centers on two primary devices: the \textbf{Shimmer 3 GSR+ sensor}
 for electrodermal activity and the \textbf{Topdon TC001 thermal camera}
 for infrared imaging.

We detail the rationale for each: \begin{itemize}
 
\item \textbf{Shimmer 3 GSR+ (Galvanic Skin Response sensor):}

The Shimmer GSR unit is a research-grade wearable sensor widely used in academic and clinical studies for EDA/GSR measurement.

We selected Shimmer over consumer fitness devices (like smartwatches) to ensure \textbf{data accuracy and flexibility}
.

The Shimmer 3 GSR+ provides raw skin conductance data with high resolution and sampling rates (up to 128 Hz), far exceeding the 4---10 Hz sampling typical of wristband trackers.

This high sampling rate means we capture the fast phasic changes in GSR without aliasing, which is crucial for precise synchronization with video frames.

Moreover, Shimmer's reliability has been demonstrated in comparative evaluations --- studies comparing the Shimmer GSR sensor to popular devices (e.g., Empatica E4 wristband or Fitbit Sense) found Shimmer data to be consistently robust and trustworthy for stress research.

The sensor uses Ag/AgCl electrodes attached to the fingers, providing a low-noise conductance measurement and it interfaces via Bluetooth, streaming data in real-time for synchronization.

The Shimmer was also chosen for its \textbf{extensibility}
: it includes additional channels (like a photoplethysmograph/PPG and accelerometer), which we can utilize to collect heart rate or motion data in the future without adding another device.

The decision to use Shimmer ensures that our "ground truth" GSR signal is of the highest possible quality, serving as a dependable reference for training machine learning models.

\item \textbf{Topdon TC001 Thermal Camera (USB, Android-compatible):}

For the thermal imaging component, we required a camera that is \textbf{portable}
, offers \textbf{high infrared resolution}
, and can integrate with a mobile platform for synchronized recording.

The Topdon TC001 thermal camera was selected after evaluating several thermal imaging options.

It features an \textbf{IR sensor resolution of 256×192 pixels}
 (which can be enhanced via image processing to an equivalent 512×384 resolution), substantially higher than many consumer thermal cameras (for comparison, the popular FLIR One Pro has a native 160×120 resolution).

This higher pixel count allows finer discrimination of small temperature differences on the face or skin, improving the fidelity of stress-related thermal features.

The TC001 is designed to connect directly to an Android smartphone via USB-C, essentially turning the phone into a thermal camera display and recorder.

This plug-and-play compatibility was a major reason for our choice --- it allowed us to integrate the thermal feed into our \textbf{Android-based data collection app}
 with relative ease.

The camera comes with an open SDK and uses standard UVC (USB Video Class) protocols, meaning we can programmatically control it and capture frames in sync with other sensors.

Additionally, the Topdon camera operates at a decent frame rate (up to \~25---30 frames per second), enabling us to capture fluid thermal video of physiological changes.

We also considered the device's \textbf{calibration and accuracy}
: the TC001 has an optimized temperature accuracy and includes a calibration shutter, which helps maintain accurate absolute temperature readings across sessions (useful if we need actual temperature values for analysis, not just relative changes).

Practically speaking, the Topdon offered the best trade-off between cost and performance for our academic project --- it is more affordable than high-end FLIR cameras but still delivers high-quality data.

By using this camera, we ensure that our platform's thermal channel is rich enough for detailed analysis of stress patterns (like those discussed in Section 2.6).

\end{itemize}

 \textbf{Synchronization and Integration:}
 A critical aspect of using these devices together is achieving precise time alignment.

The Shimmer GSR sensor provides timestamps for each data point and the Android device hosting the thermal camera can timestamp each frame; we implemented a synchronization mechanism (a master clock in the recording app) to align the streams.

This way, we can correlate each GSR peak with the exact thermal image frames (and any RGB frames, if using the phone camera) around that moment.

The importance of synchronization cannot be overstated --- misaligned data could lead to incorrect labeling (e.g., attributing a GSR surge to the wrong facial expression).

Our platform uses a \textbf{common time base}
 and logging system to ensure all modalities (GSR, thermal, and any others) are recorded in lockstep.

Early development included calibration routines where we trigger known events (like a LED flash visible in both thermal and RGB, or a manual signal causing a GSR spike) to measure and correct any offsets between sensors.

\textbf{Extensibility:}

The chosen sensor set is meant to be extensible.

The \textbf{Android smartphone}
 that acts as the hub can also record \textbf{RGB video}
 from its built-in camera simultaneously, adding an additional modality (this was discussed in Section 2.7).

We can enable or disable this as needed.

The hardware and software design allows adding further sensors such as a heart rate chest strap or a respiration belt, as long as they can interface via Bluetooth or USB to the same system --- future researchers or developers can plug in new data streams and have them synchronized with the existing ones.

The Shimmer's modular nature (it can be fitted with other sensing modules like ECG or EMG) and the Android platform's connectivity mean the \textbf{multi-modal platform can grow}
 to incorporate new physiological signals or environmental sensors with minimal changes.

This extensibility supports our central motivation: to create a \textbf{synchronized, high-quality multimodal dataset}
 that is \textit{future-proof} for various machine learning modeling efforts.

Whether the goal is to predict GSR from thermal images, to classify stress vs. no-stress from all modalities, or to explore new physiological correlations, the platform provides a flexible foundation.

In conclusion, the combination of the Shimmer GSR sensor and the Topdon thermal camera was deliberate to ensure we capture \textbf{ground-truth stress signals (GSR) alongside rich, contactless indicators (thermal imagery)}
.

By using research-grade and high-resolution devices, we maximize data quality.

By focusing on synchronization and extensibility, we ensure the data is \textbf{machine-learning ready}
 --- correctly aligned and scalable.

Every section in this chapter has underscored that our aim is not real-time inference for its own sake, but rather the \textbf{collection of robust, ground-truth aligned multimodal data}
.

The rationale behind each component choice is ultimately to serve that aim, yielding a platform capable of underpinning advanced GSR prediction models in the future.

The next steps will involve deploying this platform in experimental settings, collecting a complete dataset, and then utilizing it to train and evaluate the machine learning models that motivated its creation.
