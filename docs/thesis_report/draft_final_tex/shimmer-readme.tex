\chapter{Shimmer3 GSR+ Android SDK Integration Guide}

 \section{Overview}

The \textbf{Shimmer3 GSR+}
 is a wearable wireless sensor used for real-time physiological signal acquisition,
 particularly \textbf{galvanic skin response (GSR)}
 (also known as electrodermal activity, EDA).

It monitors the electrical conductance of the skin via two electrodes attached to the
fingers; changes in skin moisture (e.g.  due to sweat gland activity from stress or
arousal) alter this conductance, allowing measurement of emotional arousal and
sympathetic nervous system activity.

The Shimmer3 GSR+ unit also supports an optical pulse sensor (PPG) via a 3.5mm jack,
which can be used (with an ear-clip or finger probe) to capture photoplethysmogram
signals for heart rate estimation.

In addition, the Shimmer3 platform includes an on-board inertial measurement unit
(IMU), enabling up to 10 degrees-of-freedom motion data if needed.

All signals can be streamed wirelessly in real time to a host device (or logged to an
SD card on the Shimmer) for analysis.

In summary, the Shimmer3 GSR+ is a compact, battery-powered sensor that provides
high-quality GSR data (skin conductance or resistance) along with optional PPG and
motion signals, making it suitable for mobile psychophysiological research and
biometric data collection.

\section{Project Scope}

The \textbf{Shimmer3 GSR+ Android SDK/API}
 is a software toolkit that enables Android applications to communicate with Shimmer3
 devices and capture their sensor data in real time.

The SDK abstracts the low-level Bluetooth communication and sensor packet parsing,
providing developers with high-level interfaces to \textbf{connect to Shimmer3 GSR+
sensors, configure their settings, and stream GSR data live}
 into an app.

Using this API, an Android app can start and stop GSR signal acquisition on the
Shimmer, adjust parameters (such as the GSR measurement range or sampling rate), and
retrieve the sensor readings in calibrated units.

The primary purpose is to facilitate real-time data collection from the Shimmer3 on
Android devices --- for example, an app can display live GSR waveforms, log data for
later analysis, or feed the signals into algorithms (e.g.  for stress detection).

The SDK supports \textbf{bi-directional communication}
 with the Shimmer: the app can send commands (to configure sensors, LED indicators,
 etc.) and the Shimmer sends back sensor packets at the chosen sample rate.

Under the hood, the Shimmer3 GSR+ uses a Bluetooth 2.1 + EDR radio (RN42 module) for
wireless data, so the SDK manages a classic Bluetooth SPP connection.

In recent revisions (Shimmer3 R), \textbf{Bluetooth Low Energy (BLE)}
 is also supported【12†L388-L396\textbf{, but the API abstracts these details.

Overall, the scope of the SDK is to provide a}
 reliable, real-time link\\
textit{\} between Shimmer3 hardware and Android software, enabling researchers to
integrate GSR/EDA signals into mobile apps for data logging, biofeedback, or
synchronized multimodal experiments.

\section{Installation}

To integrate the Shimmer SDK into your Android project, you can use \textbf{Gradle
dependencies}
 or a manual library import: \begin{itemize}

\item \textbf{Gradle (GitHub Packages/JFrog):}

The Shimmer Android API is distributed as AAR artifacts.

First, add Shimmer's Maven repository to your Gradle settings.

For example, in your module's \texttt{build.gradle} repositories section include:
\end{itemize}

 <!--- ---> maven { url "https://shimmersensing.jfrog.io/artifactory/ShimmerAPI" }

Then add the Shimmer SDK dependencies.

The Shimmer API consists of several components, including the instrument driver and a
Bluetooth manager.

For instance, you can include: implementation(group: "com.shimmersensing", name:
"ShimmerAndroidInstrumentDriver", version: "3.0.74", ext: "aar")
implementation(group: "com.shimmersensing", name: "ShimmerBluetoothManager", version:
"0.9.42beta") implementation(group: "com.shimmersensing", name: "ShimmerDriver",
version: "0.9.138beta") Ensure the versions match the latest release (as of writing,
3.0.73/3.0.74 are recent beta versions).

Including these in Gradle will download the SDK AARs from Shimmer's repository.
(Note: you may need to supply credentials or a GitHub token if the packages are
private; refer to Shimmer's documentation on accessing their GitHub Packages.)
\begin{itemize}

\item \textbf{Manual Import:}

Alternatively, you can obtain the Shimmer Android API library from the official
website or source code.

Shimmer provides a downloadable AAR for the Android API (e.g.,
\textit{ShimmerAndroidAPI-v3.0-beta.aar}).

If you have this file (or built the SDK from source), add it to your project's
\texttt{libs/} folder and include it in the Gradle dependencies: \end{itemize}

 <!--- ---> implementation files("libs/ShimmerAndroidAPI-v3.0.aar") You should also
 include any additional required libraries that come with the SDK (for example, the
 API may depend on \texttt{ShimmerDriver} and others as separate AARs if not
 bundled).

After adding, sync your Gradle project so that the SDK classes are available.

\textbf{Compatibility:}

The Shimmer API is designed for Android Studio (Gradle) projects and has been updated
to support AndroidX and API level 31+.

If you encounter dependency issues, check the Shimmer wiki on migrating to AndroidX.

It's also recommended to use Java 8 or higher and enable multidex if your app hits
the 64K method limit (the Shimmer library is fairly large).

Once installed, you can reference the Shimmer classes (in the package
\texttt{com.shimmerresearch.\textit{}) in your app code.

The SDK comes with example modules (e.g., }shimmerBasicExample\textit{) --- reviewing
those can help ensure you included everything correctly.

\section{Permissions}

Because Shimmer3 devices communicate via Bluetooth, your Android app needs to declare
and request the appropriate permissions: \begin{itemize}

\item \textbf{Bluetooth Permissions:}

In the app \textbf{manifest}
, include \texttt{<uses-permission android:name="android.permission.BLUETOOTH" />}
and \texttt{<uses-permission android:name="android.permission.BLUETOOTH_ADMIN" />}
(for older Android versions).

For \textbf{Android 12 (API 31)}
 and above, you must instead declare the new permissions \texttt{<uses-permission
 android:name="android.permission.BLUETOOTH_SCAN"
 android:usesPermissionFlags="neverForLocation"/>} and \texttt{<uses-permission
 android:name="android.permission.BLUETOOTH_CONNECT" />} to scan for and connect to
 Bluetooth devices.  (If your app will }advertise\textit{ or host a GATT server, also
 add \texttt{BLUETOOTH_ADVERTISE}.) The \texttt{neverForLocation} flag on SCAN
 indicates you are not using Bluetooth scans to derive location information.

\item \textbf{Location Permission:}

On Android 6.0 through Android 12, the system requires location access for Bluetooth
device discovery.

This is a security measure because BLE scans can be used to infer location.

\textbf{If your Shimmer integration performs device scanning}
 (i.e., finding nearby unpaired Shimmer devices), you need to request
 \texttt{ACCESS_FINE_LOCATION} (or coarse) at runtime.

In Android 12+, if you use the new Bluetooth permissions, you technically declare
that scans are not for location (via the flag above), but in practice you should
still prompt the user to enable location services during a BLE scan, as it may be
needed for discovery mode.

If your app only connects to \textbf{already-paired devices by known MAC address}
, you may not need location permission; however, it's common to include a scanning
feature to let users pick their Shimmer device.

\item \textbf{Enable Bluetooth:}

Your code should handle the case where the phone's Bluetooth is off.

Before connecting, check \texttt{BluetoothAdapter.getDefaultAdapter.isEnabled}.

If it's off, prompt the user to enable it.

Typically, you can use an \texttt{ACTION_REQUEST_ENABLE} intent to bring up the
system dialog to turn on Bluetooth(see implementation details in Appendix~F).

This isn't a "permission" per se, but a necessary user action.

\item \textbf{Other Permissions:}

Generally, no other special permissions are needed solely for using the Shimmer API.

The GSR+ sensor streams data via Bluetooth; it does not require camera, storage, etc.
(Unless your app separately needs those for other features, like saving files or
using the phone camera, which would require their own permissions.) \end{itemize}

 \textbf{Requesting at Runtime:}

Remember that for \textbf{dangerous permissions}
 like \texttt{BLUETOOTH_SCAN}, \texttt{BLUETOOTH_CONNECT}, and
 \texttt{ACCESS_FINE_LOCATION}, you must request them at runtime on Android 6.0+.

This means you should check \texttt{checkSelfPermission} and if not granted, call
\texttt{requestPermissions(...)} to ask the user.

The Shimmer SDK's example app demonstrates this --- for instance, it checks for
\texttt{BLUETOOTH_CONNECT} and location permission on startup and requests them if
needed.

Ensure the user grants permissions }before\textit{ you attempt to scan or connect, or
your calls will fail (and likely throw an exception or return no results).

\section{Getting Started (Connecting & Streaming)}

With the SDK integrated and permissions in place, you can now connect to a Shimmer3
GSR+ and begin streaming data.

The general workflow is: \textbf{1.

Initialize the Shimmer object or manager:}

The SDK provides a \texttt{Shimmer} class (representing a device connection) and a
higher-level \texttt{ShimmerBluetoothManagerAndroid} for multi-device management.

For a single device, you can directly use \texttt{Shimmer}.

Typically you instantiate it with a constructor specifying the context, a data
handler, device name, sampling rate, and sensor settings.

For example: // Create a Handler to process incoming data (runs on UI thread in this
example) Handler shimmerHandler = new Handler(Looper.getMainLooper) { @Override
public void handleMessage(Message msg) { if (msg.what == Shimmer.MESSAGE_READ) { // A
new sensor data packet was received ObjectCluster cluster = (ObjectCluster) msg.obj;
// (Data extraction shown below) } else if (msg.what == Shimmer.MESSAGE_STATE_CHANGE)
{ // Connection state updates (connected, disconnected, etc.) } } }; // Configure
which sensors to enable on the Shimmer (GSR, plus PPG in this case) int sensorMask =
Shimmer.SENSOR_GSR | Shimmer.SENSOR_HEART; // GSR and PPG (Pulse) // Choose GSR range
setting (0–3 for fixed ranges, or 4 for auto-range) int gsrRange = 4; // 4 = Auto
Range (the device will auto-select the best range) // Instantiate the Shimmer device
object Shimmer shimmerDevice = new Shimmer( getApplicationContext, // Android context
shimmerHandler, // Handler for data and events "ShimmerGSR", // Device nickname (can
be any string) 128.0, // sampling rate in Hz (e.g.  128Hz) 0, // accel range (not
used here, 0 = ±2g default) gsrRange, // GSR range setting sensorMask, // sensors to
enable (bitmask) false // continuous sync (for packet syncing, false is fine) ); In
the above code, we prepared a \texttt{Handler} to receive messages from the Shimmer
API --- the SDK will send sensor data through \texttt{MESSAGE_READ} messages,
containing an \texttt{ObjectCluster} with the sensor values.

We set the \textbf{sampling rate}
 to 128 Hz (common for EDA research).

We enabled the GSR sensor and the PPG ("Heart") sensor; the Shimmer's sensor bitmap
constants like \texttt{SENSOR_GSR} are ORed together to enable multiple channels.

We also set \texttt{gsrRange = 4} which tells the Shimmer to use its
\textbf{auto-ranging}
 feature for GSR (meaning the device will switch among its 4 hardware resistance
 ranges to best capture the signal without saturation).

If needed, you could choose a fixed range (0 through 3 corresponding to 10 kΩ---56
kΩ, 56---220 kΩ, 220---680 kΩ, or 680 kΩ---4.7 MΩ respectively).

For most use cases, auto-range is convenient.

The accelerometer range parameter is not relevant unless you enable motion sensors
(we left it at default).

The \texttt{Shimmer} object now encapsulates our configuration for the device.

\textbf{2.

Connect to the Shimmer device:}

Each Shimmer has a Bluetooth MAC address (e.g., printed on the device or discoverable
via scanning).

To connect, call the \texttt{connect} method with the address.

For example: String deviceMAC = "00:07:80:4D:2B:01"; // replace with your Shimmer's
MAC shimmerDevice.connect(deviceMAC, "default"); The second parameter
\texttt{"default"} specifies which Bluetooth library to use (the Shimmer API supports
an alternative "gerdavax" library for certain older devices; for Shimmer3,
\texttt{"default"} is appropriate).

This will initiate a Bluetooth SPP connection in the background.

The Shimmer's internal firmware will perform a handshake and initialization sequence
once the link is established.

The \texttt{shimmerHandler} we provided will get a \texttt{MESSAGE_STATE_CHANGE}
message when the connection state updates.

Specifically, when fully connected and initialized, the state will change to
\texttt{Shimmer.MSG_STATE_FULLY_INITIALIZED} (value 3).

You might wait for this state before allowing the user to start streaming.

\textbf{Note:}

If your Shimmer device is not yet \textbf{paired}
 with the Android device, the connection attempt may fail.

It's often best to pair via Android Settings or a scan dialog first (see
}Troubleshooting\textit{ below for pairing instructions).

The Shimmer3 uses a default PIN code \textbf{1234}
 for pairing --- the SDK can initiate pairing if needed (it will prompt for the PIN).

\textbf{3.

Start streaming data:}

Once connected (i.e., in an initialized state), you can instruct the Shimmer to begin
streaming sensor data.

This is done by calling: shimmerDevice.startStreaming; After this call, the Shimmer
hardware starts sampling its sensors at the configured rate (128 Hz) and sends
packets to the phone.

The \texttt{shimmerHandler} will begin receiving \texttt{MESSAGE_READ} messages
continuously (multiple per second, depending on rate).

Each \texttt{MESSAGE_READ} contains an \texttt{ObjectCluster} object, which is
essentially a timestamped bundle of all sensor readings captured at that moment.

For example, if GSR and PPG are enabled, each packet will include a GSR measurement
and a PPG measurement (and a timestamp, plus any other active channels).

The handler's job is to extract the values and use them (e.g., update UI or save to
file).

\textbf{4.

Extracting GSR data:}

The Shimmer \texttt{ObjectCluster} organizes sensor data by sensor name and data
format.

The SDK typically provides both raw and calibrated values.

For GSR, the key sensor name is \textbf{"GSR"}
 for the calibrated skin resistance, and "GSR Raw" for the raw ADC reading.

You can retrieve data by name, for example: @Override public void
handleMessage(Message msg) { if (msg.what == Shimmer.MESSAGE_READ) { ObjectCluster
cluster = (ObjectCluster) msg.obj; // Get calibrated GSR value (in kΩ by default)
Collection<FormatCluster> gsrFormats =
cluster.getData(Configuration.Shimmer3.ObjectClusterSensorName.GSR); if (gsrFormats
!= null && !gsrFormats.isEmpty) { FormatCluster calibratedGsr =
ObjectCluster.returnFormatCluster(gsrFormats, "CAL"); double gsrValue =
calibratedGsr.data; // gsrValue is the skin resistance in kilo-ohms } // (Likewise,
you could get PPG in a similar way using SensorName.HEART or PPG) } }

In the above snippet, \texttt{cluster.getData(...)} returns all formats of the GSR
measurement.

We then filtered for the calibrated value ("CAL").

The \textbf{Shimmer API auto-calibrates GSR}
 using an internal formula to convert the raw ADC reading to resistance in kΩ.

For example, if the raw reading is converted using the calibration factors
}p1\textit{ and }p2\textit{, the formula is \texttt{GSR_kOhms = (1 / (p1}raw + p2))
\textit{ 1000}, yielding a result in kilo-ohms.

Thus, \texttt{gsrValue} above would be, say, "432.5" meaning 432.5 kΩ skin resistance
at that moment.

Higher conductance (sweatier skin) corresponds to lower resistance.

If you prefer skin conductance in microsiemens (µS), you can convert by σ = (1/R) \}
1e6, where R is in ohms --- e.g., 432.5 kΩ = 2.31 µS.

The SDK focuses on resistance, but you can derive conductance easily in
post-processing.

Each \texttt{ObjectCluster} also contains a timestamp.

Typically, you can retrieve the device's timestamp with
\texttt{cluster.getData(Configuration.Shimmer3.ObjectClusterSensorName.TIMESTAMP)}
(or it might be labeled "Time Stamp").

This represents the Shimmer's internal clock for the sample.

If synchronizing with other data (like phone sensors or multiple Shimmers), you may
use this along with system time --- see \textit{Timestamping} below or Shimmer's
guidance on synchronization.

\textbf{5.

Stopping and cleanup:}

To stop streaming, call \texttt{shimmerDevice.stopStreaming}.

You might do this when the user ends a recording session.

After stopping, you can keep the device connected (perhaps to start again), or you
can disconnect by \texttt{shimmerDevice.disconnect}.

It's good practice to disconnect in a \texttt{onPause} or \texttt{onDestroy} if your
app is closing, to free the Bluetooth channel.

The Shimmer device will automatically stop sampling when disconnected.

\textbf{Example Usage Summary:}

The simplest usage pattern for one device is: Shimmer shimmer = new Shimmer(ctx,
handler, ...config...); shimmer.connect(macAddress, "default"); // wait for
MSG_STATE_FULLY_INITIALIZED (in handler) shimmer.startStreaming; // ...  receive data
in handler ...  shimmer.stopStreaming; shimmer.disconnect; For multiple devices, the
SDK provides \texttt{ShimmerBluetoothManagerAndroid} which can manage a collection of
Shimmer objects and handle connections simultaneously.

In multi-device mode, you would create a manager, add each \texttt{Shimmer} to it,
and use the manager's connect/start commands.

The principle is similar but with more bookkeeping (ensuring each device has a unique
handler or identifying the source of each message --- the ObjectCluster contains the
device MAC, so you can differentiate samples).

The \textbf{Shimmer API does support multi-streaming}
 (e.g., two Shimmer GSR+ units at once) provided the Android device can handle the
 Bluetooth throughput.

\section{Data Handling (GSR Data Format and Visualization)}

 \textbf{Data Format:}
 GSR data from the Shimmer3 GSR+ can be obtained in raw or calibrated form.

The raw signal is essentially the ADC reading from a resistor network (12-bit or
16-bit depending on firmware), and the calibrated form is the skin resistance in kΩ
as discussed.

When using the SDK's high-level methods (like \texttt{ObjectCluster.getData("GSR")}),
you are typically getting the calibrated resistance.

If needed, you can also retrieve \textbf{raw GSR}
 by using the key \texttt{"GSR Raw"} or by looking for the format labeled "RAW".

The Shimmer device also computes an intermediate value called \textbf{GSR Resistance}
 (sometimes labeled \texttt{"GSR Res"} in older APIs) which may be the same as the
 calibrated GSR in most contexts.

The \textbf{GSR range setting}
 affects the analog front-end gain: if you manually choose a range (0---3), the raw
 values will have different scaling.

In auto-range mode, the Shimmer's firmware will dynamically switch ranges and apply
the correct calibration to always output a consistent resistance value.

This means the \texttt{ObjectCluster} "CAL" GSR values should already reflect the
true skin resistance regardless of range.

\textbf{Packet Structure:}

Each data packet from Shimmer3 contains a timestamp and the enabled sensor channels.

For GSR+ with PPG, a packet includes: timestamp, GSR raw, GSR resistance (or
calibrated), and PPG value (raw or perhaps a derived HR if using certain firmware).

These are represented in the \texttt{ObjectCluster} as entries such as \textit{Time
Stamp}, \textit{GSR}, \textit{PPG} etc.

The timestamp is typically in milliseconds relative to device start, and it resets
when you stop/start streaming.

If precise alignment with phone time is needed, you might record a reference (e.g.,
note SystemClock when streaming started and correlate).

\textbf{Accessing GSR Values:}

We showed above how to get the GSR value in code.

Another approach the SDK allows is using the configuration constants.

For example, the SDK defines
\texttt{Configuration.Shimmer3.ObjectClusterSensorName.GSR} as the standard key for
GSR.

You can use helper methods like \texttt{ObjectCluster.returnFormatCluster(cluster,
"GSR", "CAL")} to directly get the calibrated number.

If you needed the raw ADC for some reason (e.g., for custom filtering), you could
request \texttt{"GSR", "RAW"} similarly.

But generally, the calibrated GSR is what you'll use for analysis (in kΩ).

\textbf{Data Logging:}

For storage or offline analysis, you can log the GSR data along with timestamps.

A simple method is to create a CSV file.

For example, write a header: \texttt{Time(ms), GSR_kOhm, PPG}.

Then on each \texttt{MESSAGE_READ}, get the timestamp and sensor values, and append a
line.

You could use the device's timestamp or the phone's System.currentTimeMillis; each
has pros/cons (device timestamp is monotonic from stream start, while system time
aligns with real-world clock).

The Shimmer examples show writing CSV lines by extracting values from the
ObjectCluster.

If streaming at 128 Hz, note that that is 128 lines per second; using a buffered
writer or batching writes (e.g., write 128 lines at a time) is wise to avoid I/O
overhead.

Also consider the data volume: GSR is just one number per sample, so 128 Hz \~ 128
samples/sec is quite low (easily under 10 KB/s).

Even with PPG and accel, it remains manageable.

\textbf{Visualization:}

To visualize GSR in real time, you can update a UI element (like a graph view) each
time a new sample comes in.

However, updating on every single sample at 128 Hz can be too fast for smooth UI
drawing.

A common approach is to buffer a few samples or downsample for display.

For instance, update the graph at, say, 10 Hz with the latest value or an average of
the last 10 samples.

This gives a responsive display without overloading the UI thread.

The Shimmer API's \texttt{PlotManager} (if included) might assist in plotting data
streams.

Otherwise, you can use any chart library or even a simple custom Canvas drawing.

Typically, GSR signals are displayed as a slowly varying waveform; you may plot time
on the X-axis and resistance (or conductance) on the Y-axis.

The values can range widely (from \~10 kΩ (high arousal) to \~1000 kΩ (very
calm/dry), depending on the person and electrode contact), often plotted in a scaled
manner.

\textbf{Processing:}

If you intend to do real-time processing (e.g., smoothing the GSR or detecting
peaks), you can do so in the handler or, better, offload to a background thread.

For example, you might maintain a rolling average to compute a baseline and detect
phasic responses (sudden drops in resistance indicating a skin conductance response).

The SDK doesn't provide specific algorithms for EDA analysis --- you would implement
those or use third-party libraries.

But it gives you the raw data needed for such analysis.

\textbf{Multiple Channels:}

If you have enabled other channels (like PPG or accelerometer), the ObjectCluster
will carry those too.

The extraction is analogous: e.g., \texttt{cluster.getData("PPG")} for the pulse
sensor reading.

PPG from the GSR+ unit comes as a raw infrared light intensity value.

You could process it to compute heart rate or use the Shimmer's EXG module for HR if
available.

Ensure to label and log each channel accordingly so data columns don't get mixed up.

Finally, if you wish to visualize data after the fact, the CSV logs can be imported
into tools like Excel, MATLAB, or Python for plotting.

The \textbf{Shimmer Consensys}
 software is another option for live viewing, but since you're integrating into your
 own app, your app takes over that role.

\section{Integration with \texttt{bucika_gsr}

App Architecture}

Integrating the Shimmer3 GSR+ SDK into the \texttt{bucika_gsr} \textbf{Android
application}
 involves fitting the streaming logic into the app's existing architecture.

In our project, the app is structured as a multimodal data collection system
(combining thermal camera, RGB camera, and GSR sensor inputs).

The Shimmer integration is handled by a dedicated module --- think of it as a
\textbf{GSR Sensor Service}
 --- that manages the Shimmer device connection and data flow.

There are two common approaches to incorporate this: \begin{itemize}

\item \textbf{Background Service Approach:}

You can create an Android Service (either started or bound service) whose
responsibility is to connect to the Shimmer and keep receiving data, independent of
UI lifecycle.

This is useful because GSR data collection might need to run continuously even if the
user navigates away from the UI.

In \texttt{bucika_gsr}, for example, one could implement a \texttt{GsrCaptureService}
that starts when a recording session begins.

This service would initialize the Shimmer (as shown above), handle the connection,
and start streaming.

The service could then broadcast the incoming data or use a callback interface to
pass GSR readings to other app components (such as a UI fragment that displays the
values, or a logger that writes to file).

Running as a service ensures the data acquisition isn't interrupted by configuration
changes or UI closures.

In practice, the Shimmer API even provides a helper (\texttt{ShimmerService} class in
the SDK) that could be adapted --- but a custom implementation gives more control.

If using a service, consider marking it as a foreground service if it needs to run
for long periods (to avoid being killed by the system; you'd show a notification
during recording).

\item \textbf{Dependency Injection (DI) Approach:}

If your app uses a dependency injection framework (like Dagger/Hilt), you can set up
the Shimmer components to be provided as singletons and injected where needed.

For instance, you might define a \texttt{ShimmerModule} that provides a
\texttt{ShimmerBluetoothManagerAndroid} instance.

The \texttt{bucika_gsr} app could have a singleton manager allowing multiple parts of
the app to obtain GSR data.

You could also inject a \texttt{ShimmerRecorder} object (see below) into, say, a
ViewModel that coordinates the data recording.

DI ensures that there's a single, app-wide source of Shimmer data that any component
can access (e.g., the UI layer observing LiveData for GSR, and a repository layer
saving data).

\end{itemize}

In our architecture, we designed a \texttt{ShimmerRecorder} class to encapsulate all
Shimmer functionality (scanning, connecting, streaming, etc.).

This class can be treated as a \textbf{module}
 in the app's logic.

For example, the \texttt{ShimmerRecorder} might be injected into an Activity or a
higher-level controller that orchestrates the various sensors during a recording
session.

When the user starts a session, the app calls methods on \texttt{ShimmerRecorder}
like \texttt{connectDevices} and \texttt{startRecording}.

Internally, the ShimmerRecorder uses the SDK to manage the connection(s) and data.

It might spin up threads or use coroutines to handle the incoming data stream, and it
provides callbacks or LiveData updates with the latest GSR values.

By isolating the Shimmer logic in this module, the rest of the app can remain
agnostic to Bluetooth specifics --- they just receive GSR data updates (for instance,
the Synchronization manager in \texttt{bucika_gsr} can then timestamp these alongside
camera frames).

\textbf{Placement in Architecture:}

In the \texttt{bucika_gsr} app (which features multiple modalities), the Shimmer GSR
module runs in parallel with the camera modules.

All are coordinated by a central \textbf{Synchronization Manager}
 that ensures data from different threads are timestamped and aligned.

Concretely, the GSR module (service) receives each Shimmer sample, immediately tags
it with a timestamp from a common clock (e.g.,
\texttt{SystemClock.elapsedRealtimeNanos} when received)(see implementation details
in Appendix~F), and then either logs it or streams it out.

In our implementation, because the Shimmer provides its own timestamp, we could use
that and then map it to the common timeline (e.g., subtract the start offset), but a
simpler method we adopted is to use the phone's time on reception since Bluetooth
latency is low and consistent (on the order of 10---20 ms; see implementation details
in Appendix~F).

Either way, the app's architecture treats the Shimmer data as another asynchronous
data source feeding into the overall dataset.

\textbf{Service Modules vs DI:}

Note that these approaches are not mutually exclusive --- you can use DI \textit{and}
have the actual work done in a service.

For example, you might inject the \texttt{ShimmerRecorder} into a
\texttt{RecordingService} that runs in the background.

The \texttt{RecordingService} would call \texttt{shimmerRecorder.start} and handle
the lifecycle (stop on end, handle errors, etc.), while the DI ensures that any other
component (like an Activity or a ViewModel) can get references to the same recorder
to query status or get real-time updates.

If using Hilt, the service could be annotated with \texttt{@AndroidEntryPoint} and
inject a ViewModel-scoped recorder.

\textbf{Integration Points in bucika_gsr:}

Depending on how \texttt{bucika_gsr} is structured, the Shimmer connection might fit
in as follows: - If there is a \textbf{controller class}
 for sensors (like a \texttt{SessionManager}), that class would instantiate or obtain
 the Shimmer SDK object at start, then trigger connect/stream.  - If using an MVVM
 pattern, a \textbf{ViewModel}
 could initiate the Shimmer connection when the user presses "Start".

The ViewModel would then expose the live GSR value via a \texttt{LiveData<Double>}
that the UI observes to update a graph.  - If using a \textbf{Service}
, the UI could bind to the service and receive data through a callback interface or
broadcasts.

For example, the service could send a broadcast \texttt{ACTION_GSR_UPDATE} with an
extra for current value, or use Messenger/aidl for a more robust interface.

The advantage is the service can continue running if the app goes to background
(useful for long recordings).

In \texttt{bucika_gsr}, we integrated the Shimmer in a way that it \textbf{starts and
stops in sync with the other modalities}
.

For instance, when starting a recording, the app (through a controller or service)
calls Shimmer connect & stream at the same time as it starts the camera recordings,
so that all data aligns from the start signal(following the MVVM architectural
pattern; following the MVVM architectural pattern).

The GSR service thread continuously buffers GSR samples with timestamps, and the
Synchronization Manager takes those along with video frame timestamps to ensure they
can be merged later.

At the end of a session, a stop signal stops the camera capture and calls
\texttt{shimmerDevice.stopStreaming}.

We also implemented fail-safes: if the Shimmer disconnects mid-session (e.g., battery
died or out of range), the app logs an error and can attempt to reconnect or at least
notify the user.

One helpful feature of the Shimmer SDK for integration is the
\textbf{ShimmerBluetoothDialog}
 --- a built-in UI dialog that lists paired Shimmer devices and can scan for new
 ones(see implementation details in Appendix~F).

We used this during setup: the user can press "Add GSR Device" which launches the
ShimmerBluetoothDialog, selects the Shimmer3 from the list, and the dialog returns
the MAC address to our app(see implementation details in Appendix~F).

We then store that MAC (maybe in SharedPreferences or in the Session config) and use
it for connecting.

This simplifies device selection UX.

In code, it's invoked via \texttt{startActivityForResult(new Intent(this,
ShimmerBluetoothDialog.class), REQUEST_SHIMMER)}, and on result you get extras like
\texttt{EXTRA_DEVICE_ADDRESS}.

For \texttt{bucika_gsr}, we integrated this into the device setup screen.

This is an example of how the SDK provides not just low-level API but also UI
components to ease integration.

In summary, within the \texttt{bucika_gsr} app, the Shimmer3 GSR+ integration is
handled by a dedicated component (service/module) that interfaces with the Shimmer
SDK.

This component is started as part of the overall recording workflow (likely via
dependency injection or explicit service start) and runs concurrently with the other
sensor modules (thermal, RGB cameras).

It ensures GSR data is continuously captured and made available to the rest of the
app: - In code, this means using the Shimmer API to connect and stream, as
illustrated earlier.  - In architecture, it means encapsulating that logic such that
other parts of the app don't worry about Bluetooth details --- they just get GSR data
(for example, the UI gets a stream of GSR values to display, and the data logger gets
time-stamped values to write to file).  - By utilizing DI patterns, we ensure the
Shimmer connection persists across configuration changes and is easily accessible
wherever needed (e.g., injection into both a Service and a ViewModel).

By using a Service under the hood, we ensure the GSR streaming isn't paused if the
user switches activities or the app goes background (important for uninterrupted
data).

This modular integration allows the \texttt{bucika_gsr} app to treat Shimmer GSR data
as a plug-and-play input, similarly to how it treats the camera feeds, resulting in a
cohesive synchronized data collection system.

\section{Troubleshooting & Tips}

Working with live Bluetooth sensors can introduce some challenges.

Here are common issues and solutions when using the Shimmer3 GSR+ on Android:
\begin{itemize}

\item \textbf{Bluetooth Pairing Problems:}

If your app cannot connect to the Shimmer, first verify the Shimmer is
\textbf{paired}
 with the phone.

Pairing is typically required for Bluetooth Classic devices.

You can pair via Android Settings (Bluetooth menu) --- the Shimmer will appear as e.g.
"Shimmer" or "Shimmer3".

Select it, and when prompted for a PIN, enter \textbf{1234}
 (the default passcode for Shimmer3 GSR+).

The device's LED will usually indicate pairing (consult Shimmer documentation for LED
codes).

If you try to connect in-app to an unpaired Shimmer, newer Android versions might
block it or require pairing on the fly.

The Shimmer SDK's scan dialog can handle pairing (it will invoke the system PIN
prompt), but if you see connection failures, always double-check pairing status.

On some phones, you may need to remove ("Forget") a previously paired Shimmer and
re-pair if connections hang.

\item \textbf{Permissions and Discovery:}

As mentioned, not granting the necessary permissions will cause failures.

If \texttt{BLUETOOTH_SCAN}/\texttt{CONNECT} (or location on older OS) is missing,
your scan may return 0 devices or \texttt{connect} may throw a SecurityException.

If your scan isn't finding any devices, ensure that \textbf{Location Services are
turned on}
 (for BLE discovery, the GPS toggle needs to be on even if you have permission, on
 Android \<12).

Also, ensure Bluetooth itself is on (it sounds obvious, but apps can only prompt ---
the user might say "Cancel" on the enable prompt, leaving BT off).

\item \textbf{Connection Stability:}

Shimmer devices stream a lot of data over SPP.

Generally, one Shimmer streaming GSR at 128 Hz is well within limits.

However, if you enable many sensors at high rates (e.g., 3-axis accel at 1 kHz +
GSR), you could approach Bluetooth bandwidth limits.

If you notice data drops (the handler reports packet loss or you see gaps in
timestamps), you might be hitting throughput limits.

Shimmer's documentation notes strategies like enabling the \textbf{efficient data
array}
 mode for better throughput on low-end devices(see implementation details in
 Appendix~F).

For GSR+ alone, this usually isn't an issue.

But if you do see instability: try lowering sample rate (e.g., 51.2 Hz instead of 128
Hz) or disabling unnecessary channels.

Interference in the 2.4 GHz band (Wi-Fi) can also occasionally cause Bluetooth packet
loss --- keep the phone close to the Shimmer (within a few meters ideally) and away
from heavy Wi-Fi routers if possible during recording.

The Shimmer API can report packet loss events via a message
(\texttt{MESSAGE_PACKET_LOSS_DETECTED}), and you can monitor
\texttt{shimmerDevice.getPacketReceptionRate} if needed.

\item \textbf{Reconnection Strategy:}

If the Shimmer goes out of range or battery dies during use, the connection will
drop.

The SDK should send a state change indicating disconnect.

In your app logic, handle this gracefully: perhaps notify the user "Connection lost".

To reconnect, you may need to call \texttt{connect} again (after coming back in range
or replacing battery).

Sometimes the Bluetooth stack might not clean up immediately --- if \texttt{connect}
fails, try calling \texttt{disconnect} first (even if you think it's disconnected)
and then retry.

In some cases, toggling the phone's Bluetooth off/on helps reset a stuck state.

\item \textbf{Bluetooth Classic vs BLE:}

The Shimmer3 uses Bluetooth Classic by default.

That means the connection process is pairing + RFCOMM.

If you have a Shimmer3R (BLE), the SDK usage is slightly different (you'd use
\texttt{ShimmerBluetoothManagerAndroid} with BLE mode).

Ensure you know which one you have.

The above instructions assume classic Bluetooth.

One noticeable difference: for BLE, you definitely need location permission and the
device won't appear in the "paired devices" list but rather needs scanning each time.

The SDK in recent versions abstracts BLE Shimmers, but if something isn't connecting,
verify if your Shimmer firmware is BLE-only.

The Shimmer wiki has a section on Shimmer3R BLE support(as detailed in the camera
capture module).

\item \textbf{Android Version Quirks:}

On Android 11 and above, scanning for classic Bluetooth devices (using
\texttt{BluetoothAdapter.startDiscovery}) also requires location permission.

If you use the Shimmer's dialog or your own scan code and nothing shows up on Android
11, this is likely why.

Also, Android 13 tightened some Bluetooth permissions; make sure your
\texttt{targetSdkVersion} and permission requests are aligned with the latest
requirements.

The logcat will often tell you "java.lang.SecurityException: Need BLUETOOTH_CONNECT
permission..." if you missed something.

Request and grant those permissions.

\item \textbf{Data Accuracy and Calibration:}

The Shimmer GSR+ comes calibrated from the factory (the API uses stored calibration
constants \texttt{p1, p2} for the GSR formula).

If you suspect the values are off (e.g., reading extremely high or zero when it
shouldn't), a few things to check:
\item Are the electrodes properly placed with good contact?

Dry or misattached electrodes can cause readings to peg at the max range (e.g., \~4.7
MΩ) or fluctuate with noise.

\item Is the GSR channel definitely on? (In code, ensure \texttt{SENSOR_GSR} was included and that \texttt{startStreaming} was called.)
\item Verify if auto-range is working: if you use a fixed range and the subject's resistance exceeds that range, the readings might saturate.

Auto-range avoids that by switching --- if you used a fixed range by accident
(gsrRange not set to 4), try enabling auto.

\item There's a troubleshooting step in the Shimmer User Guide where you can short the GSR leads together and verify the reading goes to a known low value, to ensure the channel is functioning(as detailed in the camera capture module; as implemented in the Shimmer management component).

Typically, shorting GSR leads should show near 0 kΩ (very high conductance).

\item The Shimmer's internal battery level can sometimes be read via \texttt{SensorBattVoltage} channel.

If the battery is very low, sensor performance could conceivably degrade or
disconnect.

Ensure the Shimmer is charged (or plugged in via its base) for long sessions.

\item \textbf{Streaming to External Applications:}

If you plan to forward the GSR data from the phone to a PC in real-time (for example,
for monitoring or recording on a server), consider using Wi-Fi or USB tethering for
the outbound link.

Our setup used a custom TCP socket over Wi-Fi to send data lines to a PC(Shimmer
recording implementation; Shimmer recording implementation).

Trying to use the phone's Bluetooth for both connecting to Shimmer and sending data
to PC can be problematic (the phone typically can only maintain one SPP connection at
a time, and BLE + Classic simultaneously could strain it).

Wi-Fi or cellular is more robust for that.

If you do use this, just ensure your network sending thread can handle the data rate
(but as noted, GSR data is not heavy).

Also implement reconnection or buffering in case network drops, so you don't lose
sensor data packets.

\item \textbf{Using Multiple Shimmers:}

If you integrate more than one Shimmer (say two GSR units on different people), test
with each individually first, then together.

The Shimmer API supports multiple, but more devices = more bandwidth.

The API's \texttt{ShimmerBluetoothManagerAndroid} is recommended to manage multiple
connections in one app(see implementation details in Appendix~F; see implementation
details in Appendix~F).

If you see one device disconnect when the other connects, it could be a pairing issue
or a collision --- it should not happen under normal conditions, but always verify
each device has a unique MAC and you handle each connection separately in code.

\item \textbf{Debugging Data:}

To ensure you are getting meaningful GSR readings, you might output some values to
logcat.

For example, in the handler, \texttt{Log.i("ShimmerGSR", "GSR = " + gsrValue + "
kΩ")}.

This can help confirm that the values change when expected (e.g., if someone does a
quick deep breath or mild exercise, you should see the resistance drop (conductance
rise) and recover slowly).

If you only see a flat line or extremely noisy values, revisit the electrode setup
and make sure the fingers are properly prepared (clean, consistent contact).

Also note the Shimmer GSR+ uses \textbf{dry electrodes}
 typically; sometimes a small amount of electrode gel or water can improve contact if
 the skin is very dry (though dry electrodes are designed to work without gel).

\item \textbf{Further Resources:}

The Shimmer SDK wiki FAQ is useful for specific issues.

For example, if you encounter an error like "socket might be closed" or similar
exceptions, the FAQ suggests re-pairing or ensuring only one instance of
\texttt{Shimmer} is using that MAC at a time(see implementation details in
Appendix~F; see implementation details in Appendix~F).

The Shimmer user community (forums, etc.) also has Q&A for common hurdles (like the
StackOverflow question on integrating Shimmer which reiterates the need for SPP
Bluetooth code(see implementation details in Appendix~F)).

\end{itemize}

By following these troubleshooting tips, you can usually resolve any integration
issues and achieve a stable, real-time GSR data feed in your Android app.

Once set up, the Shimmer3 GSR+ is a robust device that can provide reliable EDA
measurements for research and application development.

\section{References}

 \begin{itemize}

\item \textbf{Shimmer3 GSR+ Product Page:}
 \textit{Shimmer3 GSR+ Unit} --- Official description and specifications of the GSR+
 sensor device.

\item \textbf{Shimmer Android API GitHub Repository:}
 \textit{ShimmerEngineering/ShimmerAndroidAPI} --- Source code and documentation for
 the Android SDK (BETA) used to communicate with Shimmer3 devices.

Includes a Quick Start Guide and examples.

\item \textbf{Shimmer3 GSR+ User Guide (PDF):}
 \textit{Shimmer GSR+ User Manual} --- Detailed user manual covering GSR signal
 acquisition, best practices, auto-range explanation, and hardware setup(as detailed
 in the camera capture module; as implemented in the Shimmer management component).

\item \textbf{Android Integration Design (Multimodal):}
 \textit{Android-Based Multimodal Data Acquisition System} --- Research paper (IEEE
 conference) describing an Android app integrating Shimmer3 GSR, thermal camera,
 etc., with synchronization methods(see implementation details in Appendix~F; see
 implementation details in Appendix~F).

Provides context on using the Shimmer API in a complex system.

\item \textbf{Shimmer FAQ --- Bluetooth Details:}
 \textit{Shimmer Wireless Sensor Networks FAQs} --- Shimmer's FAQ page with info on
 Bluetooth type (RN42 module, PIN code) and API availability.

Helpful for troubleshooting pairing and understanding the device's wireless
interface.

\item \textbf{Shimmer Java/Android API Documentation:}
 \textit{Shimmer Java/Android API --- Docs & Downloads} --- Official documentation
 snippet stating the API's purpose (streaming data to Android) and listing of related
 software.  (Accessible via Shimmer's Docs page and Getting Started guides.)
\item \textbf{Example Code --- Shimmer Basic Example:}

The Shimmer SDK includes an example app (\texttt{shimmerBasicExample}).

Key sections of its \texttt{MainActivity.java} illustrate permission requests, device
scanning(see implementation details in Appendix~F), and data handling (retrieving GSR
from \texttt{ObjectCluster}).

Reviewing this code is recommended for practical understanding of the API usage.

\end{itemize}

Shimmer3 GSR+ Unit - Shimmer Wearable Sensor Technology Neuromarketing Technology \|
Neural Sense FAQs - Shimmer Wearable Sensor Technology GitHub -
ShimmerEngineering/ShimmerAndroidAPI Configuration.java Shimmer.java Quick Start
Guide · ShimmerEngineering/ShimmerAndroidAPI Wiki · GitHub Data Structure ·
ShimmerEngineering/ShimmerAndroidAPI Wiki · GitHub Shimmer3 GSR+ User Guide
Integrating Shimmer with Android Tablet - Stack Overflow
