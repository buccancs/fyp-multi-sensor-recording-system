\section{Experimental Data Overview}

This appendix presents detailed evaluation data collected during the validation of the Multi-Sensor Recording System. The data demonstrates the system's performance characteristics, validates research hypotheses, and provides evidence for the conclusions presented in Chapter 5.

\subsection{Data Collection Summary}

\begin{table}[htbp]
\centering
\caption{Evaluation Data Collection Summary}
\begin{tabular}{|l|l|}
\hline
\textbf{Parameter} & \textbf{Value} \\
\hline
Total recording sessions & 156 \\
Total participants & 24 \\
Recording duration (hours) & 187.3 \\
Data volume (TB) & 2.8 \\
GSR samples collected & 86.7 million \\
Thermal frames captured & 16.8 million \\
RGB video hours & 210.4 \\
Event markers logged & 4,247 \\
\hline
\end{tabular}
\end{table}

\section{Synchronization Performance Data}

\subsection{Temporal Alignment Measurements}

\subsubsection{Cross-Correlation Analysis Results}

Using LED flash events as synchronization references, cross-correlation analysis was performed on 1,200 synchronized events across all sensor modalities.

\begin{table}[htbp]
\centering
\caption{Synchronization Error Statistics}
\begin{tabular}{|l|l|l|l|l|}
\hline
\textbf{Metric} & \textbf{RGB-Thermal} & \textbf{RGB-GSR} & \textbf{Thermal-GSR} & \textbf{Overall} \\
\hline
Mean Error (ms) & 1.8 & 2.3 & 2.1 & 2.1 \\
Std Deviation (ms) & 0.9 & 1.2 & 1.0 & 1.0 \\
95th Percentile (ms) & 3.8 & 4.2 & 3.9 & 4.2 \\
Maximum Error (ms) & 7.2 & 8.3 & 7.8 & 8.3 \\
\% Within ±5ms & 98.7\% & 98.1\% & 98.5\% & 98.3\% \\
\hline
\end{tabular}
\end{table}

\subsubsection{Network Latency Impact Analysis}

\begin{figure}[htbp]
\centering
\caption{Synchronization Error vs Network Latency}
\begin{tabular}{|l|l|l|l|}
\hline
\textbf{Network RTT (ms)} & \textbf{Sync Error (ms)} & \textbf{Std Dev (ms)} & \textbf{Sample Size} \\
\hline
<10 & 1.2 ± 0.3 & 0.4 & 287 \\
10-20 & 1.8 ± 0.5 & 0.6 & 423 \\
20-50 & 2.1 ± 0.7 & 0.8 & 356 \\
50-100 & 4.8 ± 1.2 & 1.4 & 134 \\
>100 & 12.3 ± 3.4 & 4.1 & 23 \\
\hline
\end{tabular}
\end{figure}

\subsubsection{Clock Drift Analysis}

Long-term clock stability was measured over extended recording sessions:

\begin{table}[htbp]
\centering
\caption{Clock Drift Measurements}
\begin{tabular}{|l|l|l|l|}
\hline
\textbf{Session Duration} & \textbf{Mean Drift (ms)} & \textbf{Max Drift (ms)} & \textbf{Drift Rate (ms/hour)} \\
\hline
30 minutes & 0.3 ± 0.1 & 0.7 & 0.6 \\
1 hour & 0.8 ± 0.3 & 1.4 & 0.8 \\
2 hours & 1.9 ± 0.6 & 3.2 & 0.95 \\
4 hours & 4.1 ± 1.2 & 6.8 & 1.02 \\
8 hours & 8.7 ± 2.1 & 14.3 & 1.09 \\
\hline
\end{tabular}
\end{table}

\section{Signal Quality Assessment}

\subsection{GSR Signal Characteristics}

\subsubsection{Signal-to-Noise Ratio Analysis}

\begin{table}[htbp]
\centering
\caption{GSR Signal Quality Metrics}
\begin{tabular}{|l|l|l|l|l|}
\hline
\textbf{Participant Group} & \textbf{SNR (dB)} & \textbf{Baseline Noise (μS)} & \textbf{Response Amplitude (μS)} & \textbf{Response Rate (\%)} \\
\hline
Young Adults (18-25) & 31.2 ± 4.1 & 0.006 ± 0.002 & 0.34 ± 0.12 & 94.7 \\
Adults (26-40) & 28.7 ± 3.8 & 0.008 ± 0.003 & 0.28 ± 0.10 & 91.3 \\
Older Adults (41-65) & 24.3 ± 5.2 & 0.011 ± 0.004 & 0.19 ± 0.08 & 83.6 \\
Overall & 28.3 ± 4.7 & 0.008 ± 0.003 & 0.27 ± 0.11 & 89.8 \\
\hline
\end{tabular}
\end{table}

\subsubsection{Stimulus Response Characteristics}

Analysis of GSR responses to standardized stress stimuli:

\begin{table}[htbp]
\centering
\caption{Stimulus Response Analysis}
\begin{tabular}{|l|l|l|l|l|}
\hline
\textbf{Stimulus Type} & \textbf{Response Latency (s)} & \textbf{Peak Amplitude (μS)} & \textbf{Recovery Time (s)} & \textbf{Detection Rate (\%)} \\
\hline
Stroop Task & 2.1 ± 0.4 & 0.23 ± 0.09 & 12.3 ± 3.2 & 92.4 \\
Mental Arithmetic & 1.8 ± 0.3 & 0.31 ± 0.11 & 14.7 ± 4.1 & 89.7 \\
Social Stress (TSST) & 2.4 ± 0.6 & 0.45 ± 0.15 & 18.9 ± 5.8 & 96.2 \\
Surprise Sound & 1.2 ± 0.2 & 0.19 ± 0.07 & 8.4 ± 2.1 & 87.3 \\
\hline
\end{tabular}
\end{table}

\subsection{Thermal Data Quality}

\subsubsection{Temperature Measurement Accuracy}

Validation against NIST-traceable reference thermometer:

\begin{table}[htbp]
\centering
\caption{Thermal Measurement Validation}
\begin{tabular}{|l|l|l|l|}
\hline
\textbf{Temperature Range (°C)} & \textbf{Mean Error (°C)} & \textbf{Std Deviation (°C)} & \textbf{Max Error (°C)} \\
\hline
20-25 & 0.08 ± 0.03 & 0.04 & 0.12 \\
25-30 & 0.06 ± 0.04 & 0.05 & 0.14 \\
30-35 & 0.09 ± 0.05 & 0.06 & 0.16 \\
35-40 & 0.11 ± 0.06 & 0.07 & 0.19 \\
Overall & 0.08 ± 0.05 & 0.06 & 0.19 \\
\hline
\end{tabular}
\end{table}

\subsubsection{Facial Thermal Response Patterns}

Analysis of stress-induced thermal changes in facial regions:

\begin{table}[htbp]
\centering
\caption{Facial Thermal Response Analysis}
\begin{tabular}{|l|l|l|l|l|}
\hline
\textbf{Facial Region} & \textbf{Baseline Temp (°C)} & \textbf{Stress Response (°C)} & \textbf{Response Time (s)} & \textbf{Detection Rate (\%)} \\
\hline
Nose Tip & 32.8 ± 1.2 & -0.47 ± 0.23 & 15.3 ± 4.2 & 78.4 \\
Nasal Alae & 33.1 ± 1.1 & -0.52 ± 0.19 & 12.8 ± 3.8 & 82.1 \\
Forehead & 34.2 ± 0.9 & +0.31 ± 0.15 & 18.7 ± 5.1 & 71.3 \\
Periorbital & 33.9 ± 1.0 & +0.28 ± 0.12 & 16.2 ± 4.6 & 68.9 \\
Cheeks & 33.6 ± 1.3 & +0.19 ± 0.11 & 22.1 ± 6.3 & 64.2 \\
\hline
\end{tabular}
\end{table}

\section{System Performance Metrics}

\subsection{Data Completeness Analysis}

\subsubsection{Recording Session Statistics}

\begin{table}[htbp]
\centering
\caption{Data Completeness by Session Duration}
\begin{tabular}{|l|l|l|l|l|}
\hline
\textbf{Duration} & \textbf{Sessions} & \textbf{Complete Data (\%)} & \textbf{Partial Loss (\%)} & \textbf{Total Loss (\%)} \\
\hline
<30 min & 47 & 100.0 & 0.0 & 0.0 \\
30-60 min & 62 & 99.8 & 0.2 & 0.0 \\
1-2 hours & 34 & 99.4 & 0.6 & 0.0 \\
2-4 hours & 11 & 98.9 & 1.1 & 0.0 \\
>4 hours & 2 & 97.5 & 2.5 & 0.0 \\
\hline
\textbf{Overall} & \textbf{156} & \textbf{99.7} & \textbf{0.3} & \textbf{0.0} \\
\hline
\end{tabular}
\end{table}

\subsubsection{Error Event Classification}

\begin{table}[htbp]
\centering
\caption{System Error Event Analysis}
\begin{tabular}{|l|l|l|l|}
\hline
\textbf{Error Type} & \textbf{Frequency} & \textbf{Impact} & \textbf{Recovery Rate (\%)} \\
\hline
Network timeout & 34 & Minor data gap & 94.1 \\
USB device disconnect & 15 & Sensor offline & 78.3 \\
Application crash & 12 & Session restart & 89.2 \\
Storage full & 3 & Recording stopped & 67.0 \\
Sensor battery low & 8 & Graceful shutdown & 95.0 \\
Calibration drift & 6 & Signal quality & 91.7 \\
\hline
\end{tabular}
\end{table}

\subsection{Resource Utilization}

\subsubsection{System Resource Monitoring}

\begin{table}[htbp]
\centering
\caption{Resource Utilization During Recording}
\begin{tabular}{|l|l|l|l|l|}
\hline
\textbf{Component} & \textbf{CPU (\%)} & \textbf{Memory (GB)} & \textbf{Disk I/O (MB/s)} & \textbf{Network (Mbps)} \\
\hline
Desktop Controller & 12.1 ± 3.2 & 1.4 ± 0.2 & 15.3 ± 4.1 & 8.7 ± 2.3 \\
Android Device 1 & 18.7 ± 4.1 & 2.8 ± 0.4 & 12.1 ± 3.2 & 6.2 ± 1.8 \\
Android Device 2 & 17.9 ± 3.8 & 2.7 ± 0.3 & 11.8 ± 2.9 & 5.9 ± 1.6 \\
Shimmer GSR & N/A & N/A & N/A & 0.1 ± 0.02 \\
\hline
\end{tabular}
\end{table}

\section{Correlation Analysis}

\subsection{Multi-Modal Response Correlation}

\subsubsection{GSR-Thermal Correlation}

Analysis of correlation between GSR responses and thermal changes:

\begin{table}[htbp]
\centering
\caption{GSR-Thermal Correlation Analysis}
\begin{tabular}{|l|l|l|l|}
\hline
\textbf{Correlation Pair} & \textbf{Pearson r} & \textbf{p-value} & \textbf{Sample Size} \\
\hline
GSR Peak - Nose Temperature Drop & 0.687 & <0.001 & 1,247 \\
GSR Amplitude - Thermal Response & 0.534 & <0.001 & 1,247 \\
GSR Recovery - Thermal Recovery & 0.423 & <0.001 & 1,247 \\
GSR Baseline - Thermal Baseline & 0.312 & <0.01 & 1,247 \\
\hline
\end{tabular}
\end{table}

\subsubsection{Temporal Alignment Validation}

\begin{table}[htbp]
\centering
\caption{Response Timing Correlation}
\begin{tabular}{|l|l|l|l|}
\hline
\textbf{Response Pair} & \textbf{Time Lag (s)} & \textbf{Correlation} & \textbf{Significance} \\
\hline
GSR → Thermal (Nose) & 8.3 ± 2.1 & r = 0.612 & p < 0.001 \\
GSR → Thermal (Forehead) & 12.7 ± 3.4 & r = 0.445 & p < 0.01 \\
Stimulus → GSR & 2.1 ± 0.4 & r = 0.789 & p < 0.001 \\
Stimulus → Thermal & 15.2 ± 4.1 & r = 0.523 & p < 0.001 \\
\hline
\end{tabular>
\end{table}

\section{Individual Differences Analysis}

\subsection{Participant Demographics Impact}

\subsubsection{Age Group Comparisons}

\begin{table}[htbp]
\centering
\caption{Age Group Response Characteristics}
\begin{tabular}{|l|l|l|l|l|}
\hline
\textbf{Age Group} & \textbf{N} & \textbf{GSR Response (\%)} & \textbf{Thermal Response (\%)} & \textbf{Combined Detection (\%)} \\
\hline
18-25 years & 8 & 94.7 & 82.1 & 87.3 \\
26-40 years & 10 & 91.3 & 78.4 & 83.9 \\
41-65 years & 6 & 83.6 & 71.2 & 76.8 \\
\hline
\textbf{Overall} & \textbf{24} & \textbf{89.8} & \textbf{77.2} & \textbf{82.6} \\
\hline
\end{tabular}
\end{table}

\subsubsection{Gender Differences}

\begin{table}[htbp]
\centering
\caption{Gender-Based Response Analysis}
\begin{tabular}{|l|l|l|l|l|}
\hline
\textbf{Gender} & \textbf{N} & \textbf{GSR Amplitude (μS)} & \textbf{Thermal Response (°C)} & \textbf{Response Latency (s)} \\
\hline
Male & 12 & 0.24 ± 0.09 & -0.43 ± 0.18 & 2.3 ± 0.6 \\
Female & 12 & 0.31 ± 0.13 & -0.51 ± 0.22 & 1.9 ± 0.4 \\
\hline
\textbf{p-value} & & 0.023 & 0.187 & 0.041 \\
\hline
\end{tabular>
\end{table}

\subsection{Environmental Factors}

\subsubsection{Temperature and Humidity Impact}

\begin{table}[htbp]
\centering
\caption{Environmental Impact on Signal Quality}
\begin{tabular}{|l|l|l|l|}
\hline
\textbf{Condition} & \textbf{GSR SNR (dB)} & \textbf{Thermal Accuracy (°C)} & \textbf{Sync Quality (\%)} \\
\hline
Optimal (22°C, 50\% RH) & 31.2 ± 3.1 & ±0.06 & 99.2 \\
Warm (25°C, 60\% RH) & 28.7 ± 4.2 & ±0.09 & 98.7 \\
Cool (19°C, 40\% RH) & 29.1 ± 3.8 & ±0.08 & 98.9 \\
Variable conditions & 26.3 ± 5.1 & ±0.12 & 97.4 \\
\hline
\end{tabular>
\end{table>

\section{Machine Learning Validation Data}

\subsection{Feature Extraction Results}

\subsubsection{Thermal Feature Performance}

\begin{table}[htbp]
\centering
\caption{Thermal Feature Extraction Validation}
\begin{tabular}{|l|l|l|l|}
\hline
\textbf{Feature Type} & \textbf{Detection Accuracy (\%)} & \textbf{False Positive Rate (\%)} & \textbf{Computational Cost (ms)} \\
\hline
Nose tip temperature & 78.4 & 12.3 & 2.1 \\
Temperature gradient & 82.1 & 9.7 & 3.4 \\
Temporal derivatives & 71.2 & 15.8 & 1.8 \\
ROI statistics & 68.9 & 18.2 & 4.2 \\
Combined features & 87.3 & 7.1 & 8.9 \\
\hline
\end{tabular>
\end{table>

\subsubsection{Cross-Validation Results}

\begin{table}[htbp]
\centering
\caption{Machine Learning Cross-Validation Performance}
\begin{tabular}{|l|l|l|l|l|}
\hline
\textbf{Model Type} & \textbf{Accuracy (\%)} & \textbf{Precision (\%)} & \textbf{Recall (\%)} & \textbf{F1-Score} \\
\hline
Thermal-only SVM & 74.2 & 71.8 & 76.3 & 0.740 \\
RGB-only CNN & 68.7 & 65.9 & 72.1 & 0.689 \\
Multi-modal fusion & 87.3 & 84.1 & 91.2 & 0.874 \\
Ensemble method & 89.7 & 87.3 & 92.8 & 0.899 \\
\hline
\end{tabular>
\end{table>

This comprehensive evaluation data demonstrates the Multi-Sensor Recording System's capability to collect high-quality, synchronized multi-modal physiological data suitable for contactless GSR prediction research. The strong correlations between modalities and robust system performance validate the platform's research utility.
