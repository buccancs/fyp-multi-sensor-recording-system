% Chapter 3 converted from 3.md to LaTeX
% TODO: Ensure main preamble loads \usepackage[strings]{underscore} or similar to allow underscores in text/code.
% TODO: Verify compilation and refine list formatting/citations if needed.

\chapter{Requirements}

\section{Problem Statement and Research Context}
The system is developed to support contactless Galvanic Skin Response (GSR) prediction research. Traditional GSR measurement requires contact sensors attached to a person's skin, but this project aims to bridge contact-based and contact-free physiological monitoring. The system allows researchers to collect synchronised multi-modal data by combining wearable GSR readings with contactless signals such as thermal imagery and video. This data supports developing models to predict GSR without direct skin contact. This addresses a key research gap by providing a reliable way to acquire ground-truth GSR data alongside contactless sensor inputs, with all data streams synchronised for analysis [1].

In the context of physiological and affective computing research, the focus is on stress and emotion analysis, where GSR is a common measure of sympathetic nervous system activity. The system integrates thermal cameras, RGB video, and inertial sensors with GSR to build a rich dataset for exploring how observable signals (such as facial thermal patterns or motion) correlate with changes in skin conductance. This multi-sensor platform operates in real-world settings (e.g. labs or field studies) and emphasises temporal precision and data integrity, so that subtle physiological responses are captured and can be later aligned for machine learning [1]. Overall, the system's goal is to enable experiments that simultaneously record a participant's physiological responses alongside visual and thermal cues, laying a foundation for contactless stress detection.

\section{Requirements Engineering Approach}
The system's requirements were derived with an iterative, research-driven approach [12]. Initially, high-level objectives (e.g. ``enable synchronised GSR and video recording'') were identified from the research goals. These were then refined through requirements elicitation involving the researchers' needs and hardware constraints. The project adopted a prototyping methodology: early versions of the system were built and tested, and feedback was used to update the requirements. For example, during development, additional needs like data encryption and device fault tolerance emerged and were added to the requirements (as evident from commit history introducing security checks and recovery features).

Requirements engineering was performed in alignment with IEEE guidelines [11]. Each requirement was documented with a unique ID and categorised (functional vs. non-functional). The implementation closely tracked the requirements -- the repository structure and commit messages show that whenever a new capability was implemented (e.g. a calibration module or time synchronisation service), it corresponded to a defined requirement. Traceability was maintained through the comprehensive matrix presented in Section~\ref{sec:req-traceability-placeholder}. Overall, the approach was incremental and user-focused: starting from the core research use cases and continuously refining the system requirements as technical insights were gained during development.

% TODO: Replace the placeholder reference below if a dedicated traceability section exists.
\label{sec:req-traceability-placeholder}

\section{Functional Requirements Overview}
The system's functional requirements are listed below (FR\#). Each requirement is stated in terms of what the system shall do. These requirements were derived from the system's implemented capabilities and verified via the source code:

\begin{itemize}
  \item FR1: Multi-Device Sensor Integration -- The system shall support connecting and managing multiple sensor devices simultaneously. This includes discovering and pairing Shimmer GSR sensors via direct Bluetooth or through an Android device acting as a bridge. If no real sensors are available, the system shall offer a simulation mode to generate dummy sensor data for testing (see Appendix F.3 for implementation details).

  \item FR2: Synchronised Multi-Modal Recording -- The system shall start and stop data recording synchronously across all connected devices. When a session starts, the PC instructs each Android device to begin recording GSR data, video (RGB camera), and thermal imaging in parallel. At the same time, the PC begins logging data from any directly connected Shimmer sensors. All streams share a common session timestamp to enable later alignment.

  \item FR3: Time Synchronisation Service -- The system shall synchronise clocks across devices to ensure all data is time-aligned. The PC runs a time synchronisation service (e.g. an NTP-like server on the local network) so that each Android device can calibrate its clock to the PC's clock before and during recording. This achieves a sub-millisecond timestamp accuracy between GSR readings and video frames, which is crucial for data integrity.

  \item FR4: Session Management -- The system shall organise recordings into sessions, each with a unique ID or name. It shall allow the researcher to create a new session (automatically timestamped) and later terminate the session when finished. Upon session start, the PC creates a directory for data storage and initialises a session metadata file. When the session ends, the metadata (start/end time, duration, status) is finalised and saved. Only one session can be active at a time, preventing overlap.

  \item FR5: Data Recording and Storage -- For each session, the system shall record: (a) Physiological sensor data from the Shimmer GSR module (including GSR and any other channels such as PPG or accelerometer, sampled at 128 Hz), and (b) video and thermal data from each Android device (with at least 1920×1080 video at 30 FPS). Sensor readings stream to the PC in real time and are written to local CSV files as they arrive to avoid data loss. Each Android device stores its own raw video/thermal files locally during recording and transfers them to the PC after the session (see FR10). The system shall also handle audio recording if enabled (e.g. microphone audio at 44.1 kHz), keeping it synchronised with the other streams.

  \item FR6: User Interface for Monitoring \& Control -- The system shall provide a GUI on the PC for the researcher to control sessions and monitor devices. This interface should list connected devices and their status (e.g. battery level, streaming/recording state), allow the user to start/stop sessions, and display indicators like recording timers and sample counts. The GUI should also show preview feeds or periodic status updates (for example, updating every few seconds with the number of samples received). If a device disconnects or encounters an error, the UI should highlight that device so the researcher can take appropriate action.

  \item FR7: Device Synchronisation and Signals -- The system shall coordinate multiple devices by sending control commands and synchronisation cues. For example, the PC can broadcast a synchronisation signal (such as a screen flash or buzzer) to all Android devices to mark the same moment across all video recordings. These signals (e.g. a visual flash on each device's screen) help with aligning footage during analysis. The system uses a JSON-based command protocol so that the PC can instruct all devices to start/stop recording and perform other actions in unison.

  \item FR8: Fault Tolerance and Recovery -- If a device (Android or sensor) disconnects or fails during a session, the system shall detect the issue and continue the session with the remaining devices. The PC will log a warning and mark that device as offline. When the device reconnects, it should seamlessly rejoin the ongoing session. The system will attempt to recover the device's session state by re-synchronising and executing any commands that were queued while it was offline. This ensures a temporary network drop does not invalidate the entire session.

  \item FR9: Calibration Utilities -- The system shall include tools for calibrating sensors and cameras. In particular, it provides a procedure to align the thermal camera's field of view with the RGB camera (e.g. using a checkerboard pattern). The researcher can perform a calibration session where images are captured and calibration parameters are computed. Calibration settings (such as pattern type, pattern size, and number of images) are adjustable in the system configuration. The resulting calibration data is saved so that recorded thermal and visual data can be accurately merged during analysis. (This requirement is derived from the presence of a calibration module in the code.)

  \item FR10: Data Transfer and Aggregation -- Once a session is stopped, the system shall transfer all recorded data from each Android device to the PC. The Android application packages the session's files (e.g. video, thermal images, local sensor logs) and sends them to the PC over the network. The PC saves each incoming file in the session folder and updates the session metadata with that file's entry (including file type and size). This automation ensures the researcher can retrieve all data without manually offloading devices. If any file fails to transfer, the system logs an error and (if possible) retries, so that data is not lost.
\end{itemize}

The above functional requirements are realised in various components of the code. For instance, connecting multiple devices and recording them simultaneously is handled by the \texttt{ShimmerPCApplication.start\_session} logic, and session metadata management is implemented in the \texttt{SessionManager} class. Detailed implementation examples are provided in Appendix F.

\section{Non-Functional Requirements}
In addition to core functionality, the system must meet several non-functional requirements to ensure it is suitable for a research setting. These include:

\begin{itemize}
  \item NFR1: Performance (Real-Time Data Handling) -- The system must handle data in real time, with minimal latency and sufficient throughput. It should support at least 128 Hz sensor sampling and 30 FPS video recording concurrently without data loss or buffering issues. The design uses multi-threading and asynchronous processing to achieve this. Video is recorded at ~5 Mbps and audio at 128 kbps, which the system writes to storage in real time. Even with multiple devices (e.g. 3+ cameras and a GSR sensor), the system should not drop frames or samples due to performance bottlenecks.

  \item NFR2: Temporal Accuracy -- Clock synchronisation accuracy between devices should be on the order of milliseconds or better. The system's built-in NTP time server and sync protocol aim to keep timestamp differences very low (e.g. $<5$ ms offset and jitter as logged after synchronisation). This is critical for valid sensor fusion; therefore, the system continuously synchronises all device clocks during a session. Timestamp precision is maintained in all logs (to the millisecond), and all devices use the PC's clock as the reference.

  \item NFR3: Reliability and Fault Tolerance -- The system must be robust to interruptions. If a sensor or network link fails, the rest of the system continues recording unaffected (as per FR8). Data already recorded should be safely preserved even if a session ends unexpectedly (e.g. the PC application crashes). The session design ensures that files are written incrementally and closed properly on stop, to avoid corruption. A recovery mechanism is in place to handle device reconnections (queuing messages while a device is offline). In addition, the Shimmer device interface includes an auto-reconnect feature to attempt re-establishing Bluetooth connections automatically.

  \item NFR4: Data Integrity and Validation -- All recorded data should be accurate and free from corruption. The system has a data validation mode for sensor data that checks incoming values are within expected ranges (for example, verifying GSR readings stay between 0.0 and 100.0 $\mu$S). Each file transfer from devices is verified for completeness (expected file sizes are known and logged in metadata). Session metadata acts as a manifest so that missing or inconsistent files can be detected easily. Also, the system will not overwrite existing session data -- each session is stored in a unique timestamped folder to avoid conflicts.

  \item NFR5: Security -- The system must safeguard the security and privacy of the recorded data. All network communication between the PC and the Android devices is encrypted (TLS is enabled in the configuration). The system requires authentication tokens for device connections (with a configurable minimum length of 32 characters) to prevent unauthorised devices from joining the session. Security checks at startup will warn if encryption or authentication is not properly configured. All recorded data files are stored locally on the PC; if cloud or external transfer is needed, it is performed only by the researcher (there is no inadvertent data upload). Additionally, the system checks file permissions and the runtime environment on startup to avoid insecure defaults.

  \item NFR6: Usability -- The system should be easy to use for researchers who are not software experts. The PC's graphical interface is designed to be intuitive, with clear controls to start/stop sessions and indicators for system status. For example, the UI shows a recording indicator when a session is active and displays device statuses (connected/disconnected, recording, battery level) in real time. Sensible default settings (e.g. a default dark theme and window size) ensure a good user experience out of the box. The Android app requires minimal user interaction after initial setup --- typically the researcher just needs to mount the devices and tap ``Connect'', with the PC orchestrating the rest. User manuals or on-screen guidance are provided for tasks like calibration.

  \item NFR7: Scalability -- The system's architecture should scale to accommodate multiple devices and long recording durations. It has been tested with up to \emph{8} Android devices streaming or recording concurrently (the configuration allows up to 10 connections). Similarly, the system supports sessions up to at least \emph{120 minutes} in duration by default. To manage large video files, recordings can be chunked into ~1 GB segments automatically so that individual file sizes remain manageable. This ensures that even high-resolution, extended sessions do not overwhelm the file system or hinder post-processing.

  \item NFR8: Maintainability and Modularity -- The system is built in a modular way to simplify maintenance. Components are separated (e.g., Calibration Manager, Session Manager, Shimmer Manager, Network Server) and communicate via clear interfaces. This modular design (evident in the repository structure) makes it easier to update one part (such as swapping out the thermal camera SDK) without affecting others. Configuration is externalised (through \texttt{config.json} and other settings), so that changes in requirements (e.g., adding new sensor types or changing sampling rates) can be accommodated by editing configuration rather than modifying code. Finally, the project includes test scripts and extensive logging to aid in debugging, which contributes to maintainability.
\end{itemize}

The non-functional requirements were validated through system testing and by reviewing configuration parameters. For example, the presence of TLS settings and runtime security checks shows that the security requirements were met, and the multi-threaded design and resource limits set in the config file indicate that performance requirements were satisfied.

\section{Use Case Scenarios}
To illustrate how the system is intended to be used, this section describes several \textbf{use case scenarios}. Each scenario outlines the typical interaction between the \textbf{user (researcher)} and the system, along with how the system's components work together to fulfil the requirements.

\subsection{Use Case 1: Conducting a Multi-Modal Recording Session}
\textbf{Description:} A researcher initiates and completes a recording session capturing GSR data alongside video and thermal streams from multiple devices. This is the primary use of the system, corresponding to a live experiment with a participant.

\textbf{Primary Actor:} Researcher (system operator).

\textbf{Secondary Actors:} Participant (subject being recorded), though they do not directly interact with the system UI.

\textbf{Preconditions:}
\begin{itemize}
  \item The Shimmer GSR sensor is charged and either connected to the PC (via Bluetooth dongle) or paired with an Android device.
  \item Android recording devices are powered on, running the recording app, and on the same network as the PC. The PC application is running and all devices have synchronised their clocks (either via initial NTP sync or prior calibration).
  \item The researcher has configured any necessary settings (e.g. chosen a session name, verified camera focus, etc.).
\end{itemize}

\textbf{Main Flow:}
\begin{itemize}
  \item The Researcher opens the PC control interface and \textbf{creates a new session} (providing a session name or accepting a default). The system validates the name and creates a session folder and metadata file on disk. The session is now ``active'' but not recording yet.
  \item The Researcher selects the devices to use. For example, they ensure the Shimmer sensor appears in the device list and one or more Android devices show as ``connected'' in the UI. If the Shimmer is not yet connected, the Researcher clicks ``Scan for Devices.'' The system performs a scan: it finds the Shimmer sensor either directly via Bluetooth or through an Android's paired devices. The Researcher then clicks ``Connect'' for the Shimmer. The system establishes a connection (or uses a simulated device if the real sensor is unavailable) and updates the UI status to ``Connected.''
  \item The Researcher checks that video previews from each Android (if available) are showing in the UI (small preview panels) and that the GSR signal is streaming (e.g. a live plot or at least a sample counter incrementing). Internally, the PC has started a background thread receiving data from the Shimmer sensor continuously. The system also maintains a heartbeat to each Android (pinging every few seconds) to ensure connectivity.
  \item The Researcher initiates recording by clicking ``Start Recording.'' The PC sends a \textbf{start command} to all connected Android devices (with a session ID). Each Android begins recording its camera (and thermal sensor, if present) and optionally starts streaming its own sensor data (if any) back to PC. Simultaneously, the PC instructs the Shimmer Manager to start logging data to file. This is done nearly simultaneously for all devices. The PC's session manager marks the session status as ``recording'' and timestamps the start time.
  \item During the recording, the Researcher can observe real-time status. For example, the UI might display the \textbf{elapsed time}, the number of data samples received so far, and the count of connected devices. Every 30 seconds, the system logs a status summary (e.g., ``Status: 1 Android, 1 Shimmer, 3000 samples''). If the Researcher has a specific event to mark, they can trigger a sync signal: for instance, pressing a ``Flash Sync'' button. When pressed, the system calls \texttt{send\_sync\_signal} to all Androids to flash their screen LEDs (creating a visible marker in the videos) and logs the event in the GSR data stream.
  \item If any device disconnects mid-session (e.g. an Android phone's WiFi drops out), the system warns the Researcher via the UI (perhaps highlighting that device in red). The recording on that device might continue offline (the Android app will still save its local video). The PC's Session Synchroniser marks the device as offline and queues any commands for it. The Researcher can continue the session if the other streams are still running. When the disconnected device comes back online (e.g. WiFi reconnects), the system automatically detects it, re-synchronises the session state, and executes any missed commands (all in the background). This recovery happens without user intervention, ensuring the session can proceed.
  \item The Researcher decides to end the recording after, say, 15 minutes. They click ``Stop Recording'' on the PC interface. The PC sends stop commands to all Androids, which then cease recording their cameras. The Shimmer Manager stops logging GSR data at the same time. Each component flushes and closes its output files. The session manager marks the session as completed, calculates the duration, and updates the session metadata (end time, duration, and status). A log message confirms the session has ended along with its duration and sample count stats.
  \item After stopping, the system \textbf{automatically initiates data transfer} from the Android devices. The File Transfer Manager on each Android packages the recorded files (e.g., \texttt{video\_20250807\_...mp4}, thermal data, etc.) and begins sending them to the PC, one file at a time. The PC receives each file (via its network server) and saves it into the session folder, simultaneously calling \texttt{SessionManager.add\_file\_to\_session()} to record the file's name and size in the metadata. A progress indicator may be shown to the Researcher.
  \item Once all files are transferred, the system notifies the Researcher that the session data collection is complete (e.g., ``Session 15min\_stress\_test completed -- 5 files saved''). The Researcher can then optionally review summary statistics (the UI might show, for example, average GSR level, or simply confirm the number of files and total data size). The session is now closed and all resources are cleaned up.
\end{itemize}

\textbf{Postconditions:} All recorded data (GSR CSV, video files, etc.) are safely stored in the PC's session directory. The session metadata JSON lists all devices that participated and all files collected. The system remains running, and the researcher could start a new session if needed. The participant's involvement is done, and the data are ready for analysis (outside the scope of the recording system). If any device failed to transfer data, the researcher is made aware so they can retrieve it manually if possible.

\textbf{Alternate Flows:}
\begin{itemize}
  \item \emph{No Shimmer available:} If the Shimmer sensor is not connected or malfunctions, the Researcher can still run a session with just video/thermal. The system will log that no GSR device is present, and it can operate in a video-only mode (possibly using a \textbf{simulated GSR signal} for demonstration).
  \item \emph{Calibration needed:} If this is the first session or devices have been re-arranged, the Researcher might perform a \textbf{calibration routine} before starting. In that case, they would use the Calibration Utility (see Use Case 2) to calibrate cameras. Once calibration is done and saved, the recording session proceeds as normal.
  \item \emph{Device battery low:} During recording, if an Android's battery is critically low, the system could alert the Researcher (since the device status includes battery level). The Researcher might decide to stop the session early or replace the device. The system will include the battery status in metadata for transparency.
  \item \emph{Network loss at end:} If the network connection to a device is lost exactly when ``Stop'' is pressed, the PC might not immediately receive confirmation from that device. In this case, the PC will mark the device as offline and proceed to finalise the session with whatever data it has. Later, when the device reconnects, the Session Synchroniser can still trigger the file transfer for that device's data so it eventually gets saved on the PC.
\end{itemize}

\subsection{Use Case 2: Camera Calibration for Thermal Alignment}
\textbf{Description:} Before conducting recordings that involve a thermal camera, the researcher performs a calibration procedure to align the thermal camera's view with the RGB camera view. This ensures that data from these two modalities can be compared pixel-to-pixel in analysis.

\textbf{Primary Actor:} Researcher.

\textbf{Preconditions:} At least one Android device with both an RGB and a thermal camera (or an external thermal camera attached) is available. A calibration pattern (e.g. a black-and-white checkerboard) is printed and ready. The system's calibration settings (pattern size, etc.) are configured if needed.

\textbf{Main Flow:}
\begin{itemize}
  \item The Researcher opens the \textbf{Calibration Tool} in the PC application (or on the Android app, depending on implementation --- assume PC-side coordination). They select the device(s) to calibrate (e.g., ``Device A -- RGB + Thermal'').
  \item The system instructs the device to enter calibration mode. Typically, the Android app might open a special calibration capture activity (with perhaps an overlay or just using both cameras). The Researcher holds the checkerboard pattern in front of the cameras and ensures it is visible to both the RGB and thermal cameras.
  \item The Researcher initiates capture (maybe pressing a ``Capture Image'' button). The device (or PC via the device) captures a pair of images -- one from the RGB camera and one from the thermal camera -- at the same moment. It may need multiple images from different angles; the configuration might specify capturing, say, 10 images. The system gives feedback after each capture (e.g., ``Image 1/10 captured'').
  \item After the required number of calibration images are collected, the Researcher clicks ``Compute Calibration.'' The system runs a calibration algorithm (likely implementing Zhang's method for camera calibration) on the collected image pairs. This computes parameters like camera intrinsics for each camera and the extrinsic transform aligning thermal to RGB.
  \item The system stores the resulting calibration parameters (e.g. in a calibration result file or in config). It also might display an estimate of calibration error (so the Researcher can judge quality). For instance, if the reprojection error exceeds the threshold (say threshold = 1.0 pixel), the system might warn that the calibration quality is low.
  \item The Researcher is satisfied with the calibration (error is acceptable). They save the calibration profile. Now the system will use this calibration data in future sessions to correct or align thermal images to the RGB frame if needed (this might be done in post-processing rather than during recording). The Researcher exits the calibration mode.
\end{itemize}

\textbf{Alternate Flows:}
\begin{itemize}
  \item \emph{Calibration failure:} If the system cannot detect the calibration pattern in the images (e.g., poor contrast in thermal image), it notifies the Researcher. The Researcher can then recapture images (maybe adjust the pattern distance or lighting) until the system successfully computes a calibration.
  \item \emph{Partial calibration:} The Researcher may choose to only calibrate intrinsics of each camera separately (for example, if thermal-RGB alignment is less important than ensuring each camera's lens distortion is corrected). In this case, the flow would be adjusted to capturing images of a known grid for each camera independently.
  \item \emph{Using stored calibration:} If calibration was done previously, the Researcher might skip this use case entirely and rely on the stored calibration parameters. The system allows loading a saved calibration file, which then becomes active for subsequent recordings.
\end{itemize}

\subsection{Use Case 3 (Secondary): Reviewing and Managing Session Data}
(Optional) -- This use case would describe how a researcher can use the system to review past session metadata and possibly replay or export data. (For brevity, this is not expanded here, but the system does include features like session listing and possibly data export tools, given that a web UI template for sessions exists.)

\section{System Analysis (Architecture \& Data Flow)}
\textbf{System Architecture:} The system adopts a \textbf{distributed architecture} with a central \textbf{PC Controller} and multiple \textbf{Mobile Recording Units}. The PC (a Python desktop application) acts as the master, coordinating all devices, while each Android device runs a recording application that functions as a client node. This architecture is essentially a \textbf{hub-and-spoke topology}, where the PC hub maintains control and timing, and the spokes (sensors/cameras) carry out data collection.

On the PC side, the software is organised into modular managers, each responsible for a subset of functionality: -- The \textbf{Session Manager} handles the overall session lifecycle (creation, metadata logging, and closure). -- The \textbf{Network Server} component (within the \texttt{AndroidDeviceManager} and \texttt{PCServer} classes) manages communication with Android devices over TCP/IP (listening on a specified port, e.g. 9000). It uses a custom JSON-based protocol for commands and status messages. -- The \textbf{Shimmer Manager} deals with the Shimmer GSR sensors, including Bluetooth connectivity (via the PyShimmer library if available) and data streaming to the PC. It also multiplexes data from multiple sensors and writes sensor data to CSV files in real-time. -- The \textbf{Time synchronisation Service} (Master Clock) runs on the PC to keep device clocks aligned. As seen in the code, an \texttt{NTPTimeServer} thread on the PC listens on a port (e.g. 8889) and services time-sync requests from clients. It periodically syncs with external NTP sources for accuracy and provides time offset information to the Android devices, which adjust their local clocks accordingly. -- The \textbf{GUI Module} (built with PyQt5 or a similar framework) provides the desktop interface. It includes panels for device status, session control, and live previews. This GUI updates based on callbacks and status data from the managers (for instance, when a new device connects, the Shimmer Manager invokes a callback that the GUI listens to, so it can display the device).

On the Android side, each device's application is composed of several components: -- A \textbf{Recording Controller} that receives start/stop commands from the PC and controls the local recording (camera and sensor capture). -- Separate \textbf{Recorder modules} for each modality: e.g., \texttt{CameraRecorder} for RGB video, \texttt{ThermalRecorder} for thermal imaging, and \texttt{ShimmerRecorder} if the Android is paired to a Shimmer sensor (see Appendix F.2 for implementation details). These recorders interface with hardware (camera APIs, etc.) and save data to local storage. A Network Client (or Device Connection Manager) that maintains the socket connection to the PC's server. It listens for commands (e.g., start/stop, sync signal) and sends back status updates or data as needed. A FileTransferManager on Android handles sending the recorded files to the PC upon request after recording. Utility components like a Security Manager (ensuring encryption if TLS is used), a Storage Manager (to check available space and organise files), etc., are also part of the design (many of these are referenced in the architecture documentation).

\textbf{Communication and Data Flow:} All communication between the PC and Android devices uses a client-server model. The PC runs the server (listening on a specified host/port, with a maximum number of connections defined), and each Android client connects to it when ready. Messages are encoded in JSON and sent over a persistent TCP socket [21]. Important message types include: device registration/hello, start session command, stop session command, sync signal command, status update from device, file transfer requests, etc.

During a session, the data flow is as follows: \textbf{Shimmer GSR Data:} If a Shimmer sensor is directly connected to the PC, it streams data via Bluetooth to the PC's Shimmer Manager, which then immediately enqueues the data for writing to a CSV and also triggers any real-time displays. If the Shimmer is connected to an Android (i.e., Android-mediated), the sensor data first goes to the Android (via Bluetooth), and the Android then forwards each GSR sample (or batch of samples) over the network to the PC. This is handled by the \texttt{AndroidDeviceManager.\_on\_android\_shimmer\_data} callback on the PC side, which receives \texttt{ShimmerDataSample} objects from the device and processes them similarly (see Appendix F.3 for implementation details). In both cases, each GSR sample is timestamped (using the synchronised clock) and logged. The PC might accumulate these in memory (e.g., in \texttt{data\_queues}) briefly for processing but ultimately writes them out via a background file-writing thread.

\textbf{Video and Thermal Data:} The Android devices record video and thermal streams locally to their flash storage (to avoid saturating the network by streaming raw video). The PC may receive low-frequency updates or thumbnails for monitoring, but the bulk video data stays on the device until session end. The temporal synchronisation of video with GSR is ensured by all devices starting recording upon the same start command and using synchronised clocks. Additionally, the PC's sync signal (flash) provides a reference point that can be seen in the video and is logged in the GSR timeline, tying the streams together. After the recording, when the PC issues the file transfer, the video files are sent to the PC. This transfer uses the network (possibly chunking files if large). The FileTransferHandler on PC receives each chunk or file and saves it. Because the PC knows the session start time and each video frame's device timestamp (the Android might embed timestamp metadata in video or provide a separate timestamp log), alignment can be done in post-processing. There is also a possibility that the Android app sends periodic timestamps during recording to the PC (as part of SessionSynchroniser updates) so the PC is aware of recording progress.

\textbf{Time Sync and Heartbeats:} Throughout a session, the PC might send periodic time sync packets to the Androids (or the Androids request them). The \texttt{SessionSynchronizer} on PC also keeps a heartbeat: it tracks if it hasn't heard from a device's state in a while, marking it offline after a threshold. Android devices likely send a small status message every few seconds (``I'm alive, recording, file X size = ...''). This data flow ensures the PC has up-to-date knowledge of each device (e.g., how many frames recorded, or storage used).

\textbf{Data Aggregation:} Once all data reaches the PC, the system has a session aggregation step (which can be considered post-session). For instance, the Session Manager might invoke a function to perform any post-processing including post-session hand segmentation processing on the recorded video. In practice, after all files are in place, the PC could combine or index them (for example, generating an index of timestamps). This ensures that all data from the distributed sources is now centralised in one place (the PC's file system) and organised.

\begin{figure}[ht]
  \centering
  % TODO: Render Mermaid diagram at docs\thesis_report\final\fig_3_1_architecture.mmd to PDF/PNG and include here via \includegraphics.
  \fbox{TODO: Insert Figure 3.1 rendered from Mermaid}
  \caption{System Architecture overview (see mermaid source: fig\_3\_1\_architecture.mmd)}
  \label{fig:figure_3_1_architecture}
\end{figure}

\textbf{Key Design Considerations:} The architecture ensures scalability by decoupling data producers (devices) from the central coordinator. Each Android operates largely independently during recording (writing to local disk), which avoids overloading the network. The PC focuses on low-bandwidth critical data (GSR streams, commands, and occasional thumbnails or status). By using local storage on devices and transferring after, the system mitigates the risk of network bandwidth issues affecting the recording quality. The use of threads and asynchronous I/O on the PC side (for writing files and handling multiple sockets) ensures that adding more devices will linearly increase resource usage but not deadlock the system.

The architecture also provides fault isolation: if one device crashes, it does not bring down the whole system -- the PC will continue managing others. The SessionSynchronizer component acts like a watchdog and queue, so even if connectivity returns after a lapse, the overall session can still be coherent.

\section{Risk Management and Mitigation Strategies}
The development and deployment of this multi-sensor recording system presents several technical and operational risks that require careful management:

\subsection{Technical Risks}
\textbf{Risk: Device Discovery Failures}\\
\emph{Impact}: Android devices may not be discoverable over the network, preventing session initiation\\
\emph{Likelihood}: Medium (network configuration dependencies)\\
\emph{Mitigation}: Implemented robust Zeroconf/mDNS service with manual IP fallback; retry mechanisms with exponential backoff; clear diagnostic messages for network troubleshooting

\textbf{Risk: Synchronisation Drift}\\
\emph{Impact}: Temporal misalignment between devices could compromise data quality for multi-modal analysis\\
\emph{Likelihood}: Medium (clock drift over long sessions)\\
\emph{Mitigation}: Continuous NTP-style clock synchronisation during active sessions; monotonic clock sources; post-processing alignment verification; configurable tolerance thresholds

\textbf{Risk: UI Responsiveness Issues}\\
\emph{Impact}: Interface freezes during high data throughput could interrupt user workflow\\
\emph{Likelihood}: Low (addressed through design)\\
\emph{Mitigation}: Asynchronous background threading for all I/O operations; C++ performance-critical components; progress indicators and user feedback; graceful degradation under load

\textbf{Risk: Data Transfer Integrity}\\
\emph{Impact}: Corrupted or incomplete file transfers could result in data loss\\
\emph{Likelihood}: Low (robust protocol design)\\
\emph{Mitigation}: SHA-256 checksums for all file transfers; retry mechanisms with timeout handling; atomic file operations; backup storage verification

\subsection{Operational Risks}
\textbf{Risk: Sensor Hardware Limitations}\\
\emph{Impact}: Thermal camera or GSR sensor malfunctions could affect specific data modalities\\
\emph{Likelihood}: Medium (hardware dependency)\\
\emph{Mitigation}: Modular architecture allows graceful degradation; alternative sensor support; comprehensive error reporting; device health monitoring

\textbf{Risk: Network Bandwidth Constraints}\\
\emph{Impact}: Insufficient network capacity could cause dropped frames or failed transfers\\
\emph{Likelihood}: Medium (environment dependent)\\
\emph{Mitigation}: Adaptive quality settings; compression algorithms; bandwidth monitoring; local storage with deferred transfer options

\subsection{Project Management Risks}
\textbf{Risk: Complexity Management}\\
\emph{Impact}: System complexity could lead to maintenance difficulties and technical debt\\
\emph{Likelihood}: Medium (inherent multi-platform complexity)\\
\emph{Mitigation}: Implemented ADR documentation for design decisions; modular architecture with clear interfaces; comprehensive testing strategy; code quality monitoring

\textbf{Risk: Integration Challenges}\\
\emph{Impact}: Cross-platform compatibility issues could limit system deployment\\
\emph{Likelihood}: Low (extensive testing)\\
\emph{Mitigation}: Multi-platform testing (Windows, Linux, macOS); standardised protocols; minimal external dependencies; comprehensive documentation

\bigskip
\noindent\rule{\linewidth}{0.4pt}

\section*{References}
See \emph{centralized references} (references.md) for all citations used throughout this thesis.
