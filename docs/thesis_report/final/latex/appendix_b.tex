\section{Getting Started}

This user manual provides step-by-step instructions for researchers using the Multi-Sensor Recording System to collect synchronized physiological and imaging data. The system is designed for researchers with basic technical knowledge who need to conduct multi-modal data collection experiments.

\subsection{System Overview for Users}

The Multi-Sensor Recording System enables simultaneous recording of:
\begin{itemize}
\item Galvanic Skin Response (GSR) data from Shimmer sensors
\item Thermal video from attached thermal cameras
\item RGB video from smartphone cameras
\item Synchronized timestamps across all modalities
\end{itemize}

All data is collected through a central desktop application that coordinates multiple Android devices and sensors.

\subsection{Prerequisites}

Before using the system, ensure you have:
\begin{itemize}
\item Valid ethics approval for human data collection (if applicable)
\item Participant consent forms completed
\item All hardware components available and charged
\item Network access configured by system administrator
\item Training completion certificate for equipment use
\end{itemize}

\section{Equipment Setup}

\subsection{Hardware Preparation}

\subsubsection{Desktop Controller Setup}

1. Power on the desktop computer
2. Ensure network connectivity (WiFi or Ethernet)
3. Verify all required software is installed
4. Check available storage space (minimum 10GB per hour of recording)

\subsubsection{Android Device Preparation}

For each Android device:
1. Charge battery to at least 80\%
2. Connect to the dedicated research WiFi network
3. Install or update the sensor application
4. Attach thermal camera via USB-C (if required)
5. Clean camera lenses and thermal sensor window

\subsubsection{Shimmer GSR Sensor Setup}

1. Check battery status (LED indicator should be green)
2. Prepare electrodes and conductive gel
3. Power on sensor and verify Bluetooth connectivity
4. Perform quick functionality test

\subsection{Participant Preparation}

\subsubsection{GSR Electrode Placement}

1. Clean participant's fingers with alcohol wipe
2. Allow to dry completely (30 seconds)
3. Apply small amount of conductive gel to electrodes
4. Attach electrodes to index and middle finger of non-dominant hand
5. Secure cables to prevent movement artifacts
6. Allow 5-minute adaptation period

\subsubsection{Positioning for Thermal/RGB Recording}

1. Position participant 0.8-1.2 meters from cameras
2. Ensure face is clearly visible and well-lit
3. Check thermal camera field of view includes nose and forehead
4. Minimize background thermal interference
5. Ensure participant comfort and natural posture

\section{Recording Procedures}

\subsection{Session Preparation}

\subsubsection{System Startup}

1. Launch desktop controller application
2. Wait for device discovery to complete (30-60 seconds)
3. Verify all expected devices appear in device list
4. Check synchronization status (should show green indicators)
5. Review system health dashboard

\subsubsection{Session Configuration}

1. Click ``New Session'' button
2. Enter session information:
   \begin{itemize}
   \item Participant ID (anonymized)
   \item Experiment protocol
   \item Session duration
   \item Special notes
   \end{itemize}
3. Select active devices and sensors
4. Configure recording parameters if needed
5. Verify storage location and available space

\subsection{Pre-Recording Checks}

\subsubsection{Signal Quality Verification}

1. Check GSR signal preview:
   \begin{itemize}
   \item Baseline should be stable
   \item No excessive noise or artifacts
   \item Electrode contact indicator green
   \end{itemize}

2. Verify video previews:
   \begin{itemize}
   \item Face clearly visible in RGB cameras
   \item Thermal image shows face temperature properly
   \item No camera obstruction or focus issues
   \end{itemize}

3. Test synchronization:
   \begin{itemize}
   \item All devices show synchronized timestamps
   \item Synchronization error <5ms
   \item No red warning indicators
   \end{itemize}

\subsubsection{Baseline Recording}

1. Instruct participant to sit quietly and relax
2. Start 2-minute baseline recording
3. Monitor for stable signals across all modalities
4. Note any issues or artifacts in session log

\subsection{Data Recording}

\subsubsection{Starting Recording}

1. Ensure participant is ready and comfortable
2. Click ``Start Recording'' button
3. Verify recording indicators appear on all devices
4. Begin experimental protocol
5. Monitor data quality throughout session

\subsubsection{During Recording}

\textbf{Monitoring Guidelines:}
\begin{itemize}
\item Watch GSR signal for expected responses
\item Check video feeds remain clear and focused
\item Monitor synchronization status continuously
\item Note any technical issues or participant comments
\item Ensure participant remains comfortable
\end{itemize}

\textbf{Quality Indicators:}
\begin{itemize}
\item Green: Normal operation, good signal quality
\item Yellow: Minor issues, continue with caution
\item Red: Significant problems, consider stopping
\end{itemize}

\subsubsection{Event Marking}

Use the event marking system to timestamp important events:
1. Click ``Mark Event'' button or press spacebar
2. Enter brief description of event
3. Events will be synchronized across all data streams
4. Use for stimulus presentation, behavioral observations, etc.

\subsection{Ending Recording}

\subsubsection{Session Termination}

1. Complete experimental protocol
2. Click ``Stop Recording'' button
3. Wait for all devices to confirm recording stopped
4. Allow data transfer to complete (may take 2-5 minutes)
5. Verify all files transferred successfully

\subsubsection{Post-Session Procedures}

1. Remove GSR electrodes carefully
2. Clean electrodes and store properly
3. Disconnect and store thermal cameras
4. Complete session notes and metadata
5. Backup data if required

\section{Data Management}

\subsection{Session Review}

\subsubsection{Immediate Quality Check}

1. Open session summary report
2. Review data completeness statistics
3. Check synchronization quality metrics
4. Identify any missing or corrupted files
5. Note issues for follow-up

\subsubsection{Data Validation}

1. Spot-check GSR signal quality
2. Verify thermal data temperature ranges
3. Check video file integrity and duration
4. Confirm timestamp alignment across modalities
5. Generate quality assurance report

\subsection{Data Export}

\subsubsection{Export Procedures}

1. Select completed session from list
2. Choose export format (CSV, HDF5, or ZIP archive)
3. Select data components to include
4. Specify anonymization options
5. Generate export package

\subsubsection{File Organization}

Exported data includes:
\begin{itemize}
\item \texttt{gsr\_data.csv}: GSR measurements with timestamps
\item \texttt{thermal\_data.csv}: Frame-by-frame temperature data
\item \texttt{rgb\_video.mp4}: High-resolution video files
\item \texttt{events.csv}: Event markers and metadata
\item \texttt{session\_info.json}: Configuration and quality metrics
\end{itemize}

\section{Troubleshooting}

\subsection{Common Issues}

\subsubsection{Device Connection Problems}

\textbf{Issue:} Android device not detected

\textbf{Solutions:}
\begin{itemize}
\item Verify device is connected to correct WiFi network
\item Restart sensor application on Android device
\item Check if device firewall is blocking connections
\item Try refreshing device discovery
\end{itemize}

\textbf{Issue:} Shimmer sensor not connecting

\textbf{Solutions:}
\begin{itemize}
\item Check battery level (replace if low)
\item Verify Bluetooth is enabled and paired
\item Try power cycling the sensor
\item Check for interference from other Bluetooth devices
\end{itemize}

\subsubsection{Signal Quality Issues}

\textbf{Issue:} Noisy GSR signal

\textbf{Solutions:}
\begin{itemize}
\item Check electrode contact and gel application
\item Minimize participant movement
\item Verify cables are not damaged
\item Allow longer adaptation period
\end{itemize}

\textbf{Issue:} Poor thermal image quality

\textbf{Solutions:}
\begin{itemize}
\item Clean thermal camera lens
\item Adjust participant distance and position
\item Check for thermal interference sources
\item Verify camera is not overheating
\end{itemize}

\subsubsection{Synchronization Problems}

\textbf{Issue:} High synchronization error

\textbf{Solutions:}
\begin{itemize}
\item Check network latency and stability
\item Restart synchronization service
\item Verify all devices are on same network
\item Reduce network traffic during recording
\end{itemize}

\subsection{Error Recovery}

\subsubsection{Recording Interruption}

If recording is interrupted:
1. Note the interruption time and cause
2. Check which data streams were affected
3. Restart recording if possible
4. Document incident in session notes
5. Assess whether session should be repeated

\subsubsection{Data Loss Prevention}

1. Save session frequently during long recordings
2. Monitor storage space continuously
3. Keep backup power sources available
4. Maintain redundant network connections if possible

\section{Best Practices}

\subsection{Experimental Design}

\begin{itemize}
\item Plan adequate time for setup and calibration
\item Include baseline recording periods
\item Design stimuli appropriate for GSR response timing
\item Consider individual differences in response patterns
\item Document all protocol deviations
\end{itemize}

\subsection{Data Quality}

\begin{itemize}
\item Perform pre-session equipment checks
\item Monitor signals continuously during recording
\item Use consistent environmental conditions
\item Maintain participant comfort and engagement
\item Document any issues immediately
\end{itemize}

\subsection{Participant Interaction}

\begin{itemize}
\item Explain the procedure clearly beforehand
\item Ensure informed consent is properly obtained
\item Monitor participant comfort throughout
\item Provide breaks as needed for long sessions
\item Respect participant's right to withdraw
\end{itemize}

\section{Safety Guidelines}

\subsection{Electrical Safety}

\begin{itemize}
\item All equipment should be properly grounded
\item Check cables for damage before each use
\item Do not use equipment with damaged insulation
\item Report any electrical issues immediately
\item Follow institutional safety protocols
\end{itemize}

\subsection{Participant Safety}

\begin{itemize}
\item Screen participants for skin allergies to electrode gel
\item Monitor for signs of discomfort or distress
\item Maintain appropriate privacy and dignity
\item Have emergency procedures readily available
\item Ensure easy exit from experimental setup
\end{itemize}

\subsection{Data Privacy}

\begin{itemize}
\item Use only anonymized participant identifiers
\item Secure data storage and transmission
\item Limit access to authorized personnel only
\item Follow institutional data protection policies
\item Obtain appropriate consent for data use
\end{itemize}

This user manual provides comprehensive guidance for researchers to effectively and safely use the Multi-Sensor Recording System for their data collection needs.
