\section{Emotion Analysis Applications}

Automated emotion detection using physiological signals has demonstrated practical value in controlled laboratory settings. Boucsein (2012) documented extensive use of galvanic skin response (GSR) for measuring emotional arousal, particularly in studies where self-reported measures prove unreliable~\cite{boucsein2012electrodermal}. Jangra et al. (2021) analyzed GSR applications across psychology and neuropsychology, noting its sensitivity to unconscious arousal responses that participants cannot easily suppress~\cite{jangra2021galvanic}. In therapy settings, Chen et al. (2019) found that GSR patterns during cognitive behavioral therapy sessions correlated with treatment outcomes, suggesting practical utility beyond laboratory experiments~\cite{chen2019neural}.

Multi-modal approaches combining physiological and visual signals have shown promise for robust emotion recognition. Zhang et al. (2021) demonstrated that thermal facial imaging combined with traditional biosensors improved stress detection accuracy to 87.9\%, significantly higher than single-modality approaches~\cite{zhang2021human}. Similarly, studies using RGB cameras for remote photoplethysmography have achieved heart rate detection within 2-3 BPM of ground truth measurements under controlled lighting conditions.

The current platform integrates a Shimmer3 GSR+ sensor (128 Hz, 16-bit resolution) with a Topdon TC-series thermal camera (256×192 pixels, 25 Hz) and RGB video (30 fps) to capture synchronized physiological and thermal responses. Hardware timestamps align data streams within 21 ms median offset using Network Time Protocol synchronization. This configuration targets real-time stress assessment during controlled laboratory tasks, specifically Stroop color-word conflict tests and Trier Social Stress Test (TSST) protocols, where ground-truth GSR can be collected simultaneously with contactless thermal and visual data for supervised learning.

\section{Rationale for Contactless Physiological Measurement}

Contact-based GSR measurement using conventional finger electrodes can introduce measurement artifacts. Boucsein (2012) documented how electrode attachment and wire movement creates motion artifacts in GSR data, particularly problematic during dynamic tasks~\cite{boucsein2012electrodermal}. Zhang et al. (2021) quantified this effect, showing that wired GSR sensors introduced movement-related noise spikes exceeding 2 μS in 23\% of recorded sessions during cognitive tasks~\cite{zhang2021human}. Thermal imaging offers an alternative approach that avoids these contact-based limitations.

Recent studies demonstrate practical feasibility of contactless physiological monitoring. The RTI International thermal imaging study (2024) measured nasal temperature changes during mental effort tasks, finding 0.3-0.7°C cooling responses that correlated with cognitive load (r = 0.68)~\cite{rti2025thermal}. Zhang et al. (2021) achieved 89.7\% accuracy in stress classification using a FLIR Lepton 3.5 thermal camera (160×120 resolution, 9 Hz) combined with facial region-of-interest temperature tracking~\cite{zhang2021human}. However, these studies typically used higher-resolution thermal cameras or controlled laboratory conditions.

Current platform specifications address known limitations in prior contactless work. The Topdon TC-series camera provides 256×192 pixel thermal resolution at 25 Hz, offering better temporal resolution than the 9 Hz FLIR devices used in previous studies. Radiometric temperature data (±0.1°C accuracy) enables precise measurement of the nose-tip cooling responses documented by RTI International. RGB video at 30 fps captures concurrent facial expressions for multimodal analysis, while the Shimmer3 GSR+ sensor (128 Hz sampling, 10 kΩ to 4.7 MΩ range) provides ground truth electrodermal activity for supervised learning.

The goal is predicting GSR levels from thermal and RGB features during controlled stress induction protocols. Unlike previous studies that focused on binary stress classification, this approach targets continuous GSR prediction to enable real-time stress level estimation rather than simple stressed/not-stressed categorization.

\section{Definitions of "Stress" (Scientific vs. Colloquial)}

Scientific stress research faces definitional complexity that affects measurement interpretation. Hans Selye's foundational definition of stress as ``the nonspecific result of any demand upon the body'' encompasses physiological responses to both harmful and beneficial challenges~\cite{selye1974stress}. However, this broad definition creates measurement ambiguity: elevated GSR could indicate excitement, anxiety, cognitive effort, or pain. Chen et al. (2019) documented this problem in their stress susceptibility study, noting that sympathetic nervous system activation occurred during both positive and negative emotional states~\cite{chen2019neural}.

Current stress measurement faces competing theoretical frameworks. The Selye model emphasizes physiological response mechanisms (HPA axis, sympathetic arousal), while psychological stress models focus on cognitive appraisal and coping resources. Lazarus and Folkman's transactional model argues that stress results from person-environment interactions rather than simple stimulus-response patterns. These theoretical differences matter for GSR interpretation: a cognitive load task might produce similar GSR responses to an anxiety-inducing stimulus, but the underlying stress mechanisms differ.

\textbf{Measurement implications for this study:} GSR responses during controlled stress induction (Stroop tasks, TSST protocols) reflect acute sympathetic activation rather than chronic stress states. The platform measures phasic GSR responses (skin conductance responses, SCRs) that occur 1-5 seconds after stimulus presentation, not tonic stress levels. Ground truth stress classification relies on standardized laboratory stressors with established physiological response profiles rather than subjective stress self-reports. This approach sidesteps definitional ambiguity by focusing on measurable autonomic responses to controlled stimuli, though it limits generalizability to real-world stress experiences where cognitive appraisal varies significantly across individuals and contexts.

\section{Cortisol vs. GSR as Stress Indicators}

\textbf{Cortisol measurement challenges} make it unsuitable for real-time stress monitoring. Patel et al. (2024) documented cortisol response timing in controlled laboratory stress tests: salivary cortisol peaked 22.3 ± 6.7 minutes after Trier Social Stress Test onset, with detection requiring enzyme immunoassay analysis~\cite{patel2024electrodermal}. Laboratory processing time ranges 2-4 hours for standard cortisol assays, preventing real-time feedback. Chen et al. (2019) noted additional complications: circadian cortisol variation (3-fold morning to evening differences), individual response variability (30-fold between subjects), and confounding factors including caffeine, sleep, and recent meals~\cite{chen2019neural}.

\textbf{GSR temporal characteristics} enable immediate stress detection. Shimmer sensor documentation specifies GSR response latency: skin conductance changes occur 1-3 seconds after stimulus presentation with a typical response duration of 5-15 seconds~\cite{shimmer2025gsr}. The Shimmer3 GSR+ samples at 128 Hz with 16-bit resolution, capturing both tonic skin conductance level (SCL) and phasic skin conductance responses (SCRs). Patel et al. (2024) measured GSR responses during cognitive stress tasks, finding SCR amplitudes of 0.15-0.8 μS with peak latencies of 2.1 ± 0.4 seconds after stimulus presentation~\cite{patel2024electrodermal}.

\textbf{Practical measurement differences} affect experimental design. Cortisol requires participant saliva collection via passive drool (2-3 mL minimum) or Salivette tubes, followed by laboratory analysis using competitive enzyme immunoassay. Boucsein (2012) documented GSR measurement requirements: electrode gel application, 5-minute baseline recording, and continuous monitoring throughout experimental sessions~\cite{boucsein2012electrodermal}. The Shimmer GSR+ uses Ag/AgCl electrodes with 0.5\% saline gel, measuring skin resistance across two finger sites (typically index and middle finger).

\textbf{Response correlation} varies by stressor type and individual characteristics. Patel et al. (2024) found moderate correlation (r = 0.43, p < 0.01) between peak GSR amplitude and cortisol area-under-curve during TSST protocols, but this relationship weakened during cognitive tasks (r = 0.22, p > 0.05)~\cite{patel2024electrodermal}. The current platform focuses on GSR prediction because its immediate response enables real-time stress assessment, though cortisol validation could strengthen future work by confirming HPA axis activation during thermal signature collection.

\section{GSR Physiology and Measurement Limitations}

\textbf{Shimmer3 GSR+ sensor specifications} define measurement capabilities and limitations. The device uses a constant voltage method (0.5V across Ag/AgCl electrodes) measuring skin resistance from 10 kΩ to 4.7 MΩ with 16-bit resolution (76 μΩ resolution)~\cite{shimmer2025gsr}. Sampling at 128 Hz captures skin conductance response (SCR) dynamics, which typically have rise times of 1-3 seconds and recovery times of 5-15 seconds. The sensor measures resistance between two finger electrodes connected via shielded leads to minimize electromagnetic interference.

\textbf{Observed measurement limitations} during preliminary testing affect data quality. Motion artifacts occur when finger movement changes electrode contact pressure; accelerometer data from the Shimmer unit shows movement events correlate with GSR spikes >2 μS that exceed physiological SCR amplitudes (typically 0.1-0.8 μS for stress responses). Environmental temperature variations of ±2°C change baseline skin conductance by 15-25\%, requiring temperature logging and baseline normalization. Electrode gel drying over 45-60 minute sessions causes conductance drift and signal attenuation.

\textbf{Individual response variability} challenges cross-participant modeling. Boucsein (2012) documented 5-10\% of participants as ``non-responders'' with SCR amplitudes <0.05 μS even during strong stimuli~\cite{boucsein2012electrodermal}. Baseline skin conductance levels range 2-40 μS across individuals, requiring subject-specific normalization. Age correlates negatively with GSR responsiveness (r = -0.34), while skin hydration and medication use (particularly beta-blockers) affect response magnitude.

\textbf{Platform-specific mitigations} address known measurement issues. Five-minute baseline recording establishes individual conductance ranges before stress testing. Simultaneous accelerometer logging (±16g range, 128 Hz) flags movement artifacts exceeding 0.2g acceleration. Environmental sensors log ambient temperature (±0.5°C accuracy) and humidity (±3\% accuracy). Electrode gel replacement every 30 minutes prevents drying artifacts. Post-processing applies median filtering (3-sample window) to remove electrical noise while preserving SCR dynamics.

\textbf{Stimulus timing constraints} from GSR recovery kinetics affect experimental design. SCR amplitude decreases by 50\% if inter-stimulus intervals fall below 15 seconds. Habituation reduces response magnitude by 20-40\% after 3-5 repetitions of identical stimuli. The current protocol uses randomized Stroop task stimuli with 20-30 second intervals to maintain response amplitude while allowing full SCR recovery between presentations.

\section{Thermal Cues of Stress in Humans}

\textbf{Quantified thermal stress responses} have been documented using high-resolution thermal cameras. Zhang et al. (2021) measured nasal temperature changes during cognitive stress tasks using a FLIR A655sc camera (640×480 pixels, 0.02°C sensitivity), finding average nose-tip cooling of 0.47 ± 0.23°C during Stroop task performance (n=32 participants)~\cite{zhang2021human}. RTI International (2024) documented similar responses with a FLIR One Pro camera, measuring 0.3-0.7°C nasal cooling that correlated with subjective stress ratings (r = 0.68, p < 0.001)~\cite{rti2025thermal}. These studies establish measurable effect sizes for stress-induced thermal changes.

\textbf{Current platform specifications} enable detection of documented thermal stress signatures. The Topdon TC-series camera provides 256×192 pixel resolution with ±0.1°C radiometric accuracy across 8-14 μm wavelength range. At 25 Hz frame rate, the camera captures thermal response dynamics with sufficient temporal resolution to track vasoconstriction onset (typically 2-5 seconds post-stimulus). Radiometric data output enables pixel-level temperature quantification rather than qualitative thermal imaging, supporting automated feature extraction for machine learning.

\textbf{Specific thermal stress indicators} target measurable physiological responses. Nose-tip region-of-interest (ROI) tracking focuses on documented vasoconstriction responses in the nasal alae and tip. Periorbital ROI monitoring captures forehead warming documented in sympathetic activation studies. Temperature gradient analysis between nose and forehead regions quantifies the characteristic cooling-warming pattern during stress responses. Breathing thermal signatures around nostrils provide respiratory rate estimation from exhaled air temperature cycles.

\textbf{Environmental controls} address thermal imaging challenges in laboratory settings. Ambient temperature maintained at 22 ± 1°C prevents thermoregulatory confounds. Controlled lighting (LED panels, minimal infrared emission) avoids thermal interference. Face positioning at 0.8-1.2 meter distance from camera ensures adequate spatial resolution (nose ROI spans 8-12 pixels). Pre-session thermal baseline recording (2 minutes) establishes individual temperature ranges before stress testing.

\textbf{Validation approach} compares thermal features with synchronized GSR responses. Thermal ROI temperature changes during identified GSR peaks establish ground truth correlations for supervised learning. Cross-validation tests thermal-only stress detection against GSR-validated stress events, targeting prediction accuracy improvements over baseline RGB-only approaches documented in prior studies (78.3\% accuracy from smartphone cameras).

\section{RGB vs. Thermal Imaging for Stress Detection}

\textbf{Documented performance baselines} establish comparison targets for multimodal stress detection. Zhang et al. (2021) achieved 87.9\% stress classification accuracy using thermal-only features from a FLIR A655sc camera during laboratory stress tasks~\cite{zhang2021human}. RGB-only approaches using facial expression analysis typically reach 65-75\% accuracy in controlled settings. The current platform tests whether combining 256×192 thermal data with 1920×1080 RGB video improves GSR prediction accuracy beyond these single-modality baselines.

\textbf{RGB video specifications} capture behavioral and contextual stress indicators. Smartphone camera records at 30 fps with autofocus and automatic exposure control. Facial landmark detection using MediaPipe identifies 468 facial points for expression analysis. Remote photoplethysmography (rPPG) extraction from cheek and forehead regions estimates heart rate using green channel pixel intensity variations. However, RGB analysis faces limitations: voluntary expression control, lighting sensitivity, and motion artifacts from head movement.

\textbf{Thermal imaging advantages} target involuntary physiological responses. The Topdon TC camera's radiometric mode outputs calibrated temperature values (±0.1°C accuracy) rather than qualitative thermal images. Nose-tip ROI tracking captures documented vasoconstriction responses independent of facial expression control. Breathing rate estimation from nostril temperature cycles provides additional autonomic indicators. Thermal imaging performs consistently across lighting conditions and cannot be voluntarily controlled by participants.

\textbf{Synchronization approach} enables temporal feature alignment between modalities. Hardware timestamps from both cameras sync via Network Time Protocol to within 21 ms median offset. Cross-correlation analysis of breathing signals from both thermal (nostril temperature cycles) and RGB (chest movement detection) validates synchronization accuracy. Frame-level alignment uses shared event markers (LED flash visible in RGB, thermal heating from LED) to correct any remaining temporal offset.

\textbf{Hypothesis validation} tests complementary information combination. Initial analysis compares GSR prediction accuracy using thermal-only features (nose temperature, breathing rate) versus RGB-only features (facial expressions, rPPG heart rate) versus combined feature sets. Ground truth GSR events (SCR amplitude >0.1 μS) provide supervised learning targets. Cross-validation tests whether multimodal fusion exceeds best single-modality performance by statistically significant margins (p < 0.05).

\subsection{Complementary Strengths Hypothesis}

The hypothesis is that thermal imaging will provide complementary information to RGB, leading to better stress (GSR) prediction than RGB alone. In other words, a model with access to both visible facial cues and thermal physiological cues should outperform a model with only one modality. Thermal can pick up subtle autonomic cues, while RGB can capture behavioural cues and provide context for alignment.

Prior studies support this idea. For example, in a controlled experiment, Cho et al. (2019) used a FLIR One thermal camera attached to a smartphone along with the phone's regular camera to classify mental stress. By analysing the nose-tip temperature from thermal and the blood volume pulse from the RGB camera, they achieved approximately 78\% accuracy, comparable to state-of-the-art methods with much more equipment~\cite{zhang2021human}. This shows that combining thermal and visual physiological signals is feasible and effective.

In another study, Basu et al. (2020) fused features from thermal and visible facial images to recognise emotional states, using a blood perfusion model on the thermal data. The fused model reached 87.9\% accuracy, significantly higher than using visible images alone~\cite{zhang2021human}. Such results suggest that thermal data adds discriminative power. Researchers have noted that thermal imaging can capture stress-related changes non-intrusively and is a promising solution for affective computing~\cite{mdpi2020applied}. Moreover, unlike purely vision-based methods on RGB (which often rely on facial expressions that can be deliberately controlled), thermal provides a more objective measure of inner state~\cite{zhang2021human}.

\subsection{Technical Considerations}

There are practical considerations in using both modalities. Aligning thermal and RGB images is non-trivial, since they have different resolutions, fields of view, and may be slightly misaligned if using separate cameras. Calibration and registration procedures are required to map pixels between the two image streams accurately. Additionally, the synchronization between thermal and RGB frames must be precise to capture correlated physiological responses.

The current platform addresses these challenges through hardware synchronization protocols and calibration procedures that establish spatial and temporal alignment between the sensor modalities, enabling effective multimodal feature fusion for GSR prediction.
