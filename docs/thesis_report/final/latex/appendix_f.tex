\section{Core System Components}

This appendix provides key code listings that illustrate the implementation of critical system components. Complete source code is available in the project repository.

\subsection{Android Thermal Camera Integration}

\subsubsection{Topdon Thermal Camera Handler}

\begin{lstlisting}[language=Java, caption=Thermal Camera Integration Implementation]
public class TopdonThermalCamera implements ThermalCamera {
    private UVCCamera uvcCamera;
    private USBMonitor usbMonitor;
    private Surface previewSurface;
    private boolean isRecording = false;
    private CSVWriter thermalDataWriter;

    @Override
    public void initialize(Context context) {
        usbMonitor = new USBMonitor(context, onDeviceConnectListener);
        usbMonitor.register();
    }

    private final USBMonitor.OnDeviceConnectListener onDeviceConnectListener =
        new USBMonitor.OnDeviceConnectListener() {
            @Override
            public void onAttach(UsbDevice device) {
                if (isTopdonCamera(device)) {
                    usbMonitor.requestPermission(device);
                }
            }

            @Override
            public void onConnect(UsbDevice device, USBMonitor.UsbControlBlock ctrlBlock,
                                boolean createNew) {
                uvcCamera = new UVCCamera();
                uvcCamera.open(ctrlBlock);
                configureCamera();
                startPreview();
            }
        };

    private void configureCamera() {
        // Set thermal camera to radiometric mode
        uvcCamera.setPreviewSize(256, 192);
        uvcCamera.setPreviewCallback(thermalFrameCallback);
    }

    private final IFrameCallback thermalFrameCallback = new IFrameCallback() {
        @Override
        public void onFrame(ByteBuffer frame) {
            if (isRecording) {
                long timestamp = System.nanoTime();
                float[] temperatures = extractTemperatureData(frame);
                logThermalFrame(timestamp, temperatures);
            }
        }
    };

    private float[] extractTemperatureData(ByteBuffer frame) {
        float[] temperatures = new float[256 * 192];
        frame.rewind();

        for (int i = 0; i < temperatures.length; i++) {
            // Convert raw thermal data to temperature values
            short rawValue = frame.getShort();
            temperatures[i] = convertToTemperature(rawValue);
        }

        return temperatures;
    }

    private float convertToTemperature(short rawValue) {
        // Apply calibration coefficients
        return (rawValue * TEMP_SCALE_FACTOR) + TEMP_OFFSET;
    }

    private void logThermalFrame(long timestamp, float[] temperatures) {
        String[] csvRow = new String[temperatures.length + 2];
        csvRow[0] = String.valueOf(timestamp);
        csvRow[1] = String.valueOf(frameCounter++);

        for (int i = 0; i < temperatures.length; i++) {
            csvRow[i + 2] = String.format("%.2f", temperatures[i]);
        }

        thermalDataWriter.writeNext(csvRow);
    }

    @Override
    public void startRecording(String sessionPath) {
        try {
            File thermalFile = new File(sessionPath, "thermal_data.csv");
            thermalDataWriter = new CSVWriter(new FileWriter(thermalFile));

            // Write CSV header
            String[] header = createCSVHeader();
            thermalDataWriter.writeNext(header);

            isRecording = true;
            frameCounter = 0;
        } catch (IOException e) {
            Log.e(TAG, "Failed to start thermal recording", e);
        }
    }

    @Override
    public void stopRecording() {
        isRecording = false;
        if (thermalDataWriter != null) {
            try {
                thermalDataWriter.close();
            } catch (IOException e) {
                Log.e(TAG, "Error closing thermal data file", e);
            }
        }
    }
}
\end{lstlisting}

\subsection{Shimmer GSR Integration}

\subsubsection{GSR Sensor Communication Handler}

\begin{lstlisting}[language=Java, caption=Shimmer GSR Sensor Integration]
public class ShimmerGsrSensor implements PhysiologicalSensor {
    private Shimmer shimmer;
    private boolean isStreaming = false;
    private CSVWriter gsrDataWriter;
    private final Object streamingLock = new Object();

    public ShimmerGsrSensor(String bluetoothAddress) {
        shimmer = new Shimmer(bluetoothAddress);
        shimmer.setShimmerUserAssignedName("GSR_Sensor");
        setupSensorConfiguration();
    }

    private void setupSensorConfiguration() {
        // Configure GSR sensor settings
        shimmer.enableSensor(Shimmer.SENSOR_GSR);
        shimmer.enableSensor(Shimmer.SENSOR_ACCEL);
        shimmer.enableSensor(Shimmer.SENSOR_GYRO);

        // Set sampling rate to 128Hz
        shimmer.setSamplingRateShimmer(128.0);

        // Configure GSR range and resolution
        shimmer.setGSRRange(Shimmer.GSR_RANGE_AUTO);
        shimmer.setAccelRange(Shimmer.ACCEL_RANGE_2G);

        // Set up data callback
        shimmer.setDataProcessing(dataCallback);
    }

    private final DataProcessing dataCallback = new DataProcessing() {
        @Override
        public void onNewDataAvailable(ObjectCluster objectCluster) {
            if (isStreaming) {
                processIncomingData(objectCluster);
            }
        }
    };

    private void processIncomingData(ObjectCluster data) {
        synchronized (streamingLock) {
            try {
                long timestamp = System.nanoTime();

                // Extract GSR data
                double gsrResistance = getCalData(data, Shimmer.CHANNEL_TYPE.CAL.toString(),
                                                 "GSR Resistance");
                double gsrConductance = 1000.0 / gsrResistance; // Convert to μS

                // Extract accelerometer data for motion detection
                double accelX = getCalData(data, Shimmer.CHANNEL_TYPE.CAL.toString(),
                                          "Accelerometer X");
                double accelY = getCalData(data, Shimmer.CHANNEL_TYPE.CAL.toString(),
                                          "Accelerometer Y");
                double accelZ = getCalData(data, Shimmer.CHANNEL_TYPE.CAL.toString(),
                                          "Accelerometer Z");

                // Extract temperature for drift compensation
                double temperature = getCalData(data, Shimmer.CHANNEL_TYPE.CAL.toString(),
                                               "Temperature");

                // Log data to CSV
                logGsrData(timestamp, gsrConductance, temperature, accelX, accelY, accelZ);

            } catch (Exception e) {
                Log.e(TAG, "Error processing GSR data", e);
            }
        }
    }

    private double getCalData(ObjectCluster data, String dataType, String sensorName) {
        FormatCluster formatCluster = data.getFormatCluster(dataType, sensorName);
        return formatCluster != null ? formatCluster.mData : Double.NaN;
    }

    private void logGsrData(long timestamp, double gsr, double temp,
                           double accelX, double accelY, double accelZ) {
        String[] csvRow = {
            String.valueOf(timestamp),
            String.format("%.6f", gsr),
            String.format("%.2f", temp),
            String.format("%.4f", accelX),
            String.format("%.4f", accelY),
            String.format("%.4f", accelZ)
        };

        gsrDataWriter.writeNext(csvRow);
    }

    @Override
    public void startStreaming(String sessionPath) {
        try {
            File gsrFile = new File(sessionPath, "gsr_data.csv");
            gsrDataWriter = new CSVWriter(new FileWriter(gsrFile));

            // Write CSV header
            String[] header = {"timestamp_ns", "gsr_microsiemens", "temperature_c",
                              "accel_x", "accel_y", "accel_z"};
            gsrDataWriter.writeNext(header);

            synchronized (streamingLock) {
                isStreaming = true;
            }

            // Start Shimmer streaming
            shimmer.startStreaming();

        } catch (IOException e) {
            Log.e(TAG, "Failed to start GSR streaming", e);
        }
    }

    @Override
    public void stopStreaming() {
        synchronized (streamingLock) {
            isStreaming = false;
        }

        shimmer.stopStreaming();

        if (gsrDataWriter != null) {
            try {
                gsrDataWriter.close();
            } catch (IOException e) {
                Log.e(TAG, "Error closing GSR data file", e);
            }
        }
    }

    @Override
    public void connect() {
        shimmer.connect();
    }

    @Override
    public void disconnect() {
        shimmer.disconnect();
    }

    @Override
    public boolean isConnected() {
        return shimmer.getShimmerState() == Shimmer.STATE_CONNECTED;
    }
}
\end{lstlisting}

\section{Desktop Controller Implementation}

\subsection{Session Management Core}

\subsubsection{Session Controller}

\begin{lstlisting}[language=Python, caption=Session Management Implementation]
class SessionController:
    def __init__(self, config_manager, device_manager):
        self.config = config_manager
        self.device_manager = device_manager
        self.current_session = None
        self.session_data = {}
        self.recording_state = RecordingState.IDLE

    def create_session(self, session_info):
        """Create a new recording session with metadata."""
        session_id = self._generate_session_id()
        session_path = self._create_session_directory(session_id)

        self.current_session = {
            'id': session_id,
            'path': session_path,
            'info': session_info,
            'start_time': None,
            'end_time': None,
            'devices': [],
            'metadata': {}
        }

        # Initialize session metadata file
        metadata_file = os.path.join(session_path, 'session_metadata.json')
        with open(metadata_file, 'w') as f:
            json.dump(self.current_session, f, indent=2, default=str)

        return session_id

    def start_recording(self):
        """Start synchronized recording across all devices."""
        if self.recording_state != RecordingState.IDLE:
            raise ValueError("Recording already in progress")

        if not self.current_session:
            raise ValueError("No active session")

        try:
            # Synchronize clocks before recording
            self._synchronize_device_clocks()

            # Prepare all devices for recording
            self._prepare_devices_for_recording()

            # Calculate synchronized start time
            start_time = time.time_ns() + 2_000_000_000  # 2 seconds from now

            # Send start commands to all devices
            start_commands = []
            for device in self.device_manager.get_connected_devices():
                cmd = {
                    'type': 'SESSION_START',
                    'session_id': self.current_session['id'],
                    'start_time': start_time,
                    'session_path': f"session_{self.current_session['id']}"
                }
                start_commands.append((device, cmd))

            # Send all commands simultaneously
            self._send_synchronized_commands(start_commands)

            # Update session state
            self.current_session['start_time'] = start_time
            self.recording_state = RecordingState.RECORDING

            # Start local data logging if applicable
            self._start_local_logging()

            logging.info(f"Recording started for session {self.current_session['id']}")

        except Exception as e:
            logging.error(f"Failed to start recording: {e}")
            self.recording_state = RecordingState.ERROR
            raise

    def stop_recording(self):
        """Stop synchronized recording across all devices."""
        if self.recording_state != RecordingState.RECORDING:
            raise ValueError("No recording in progress")

        try:
            stop_time = time.time_ns()

            # Send stop commands to all devices
            stop_commands = []
            for device in self.device_manager.get_connected_devices():
                cmd = {
                    'type': 'SESSION_STOP',
                    'session_id': self.current_session['id'],
                    'stop_time': stop_time
                }
                stop_commands.append((device, cmd))

            self._send_synchronized_commands(stop_commands)

            # Stop local logging
            self._stop_local_logging()

            # Update session metadata
            self.current_session['end_time'] = stop_time
            self.current_session['duration'] = stop_time - self.current_session['start_time']

            # Wait for data transfer completion
            self._wait_for_data_transfer()

            # Finalize session
            self._finalize_session()

            self.recording_state = RecordingState.COMPLETED
            logging.info(f"Recording completed for session {self.current_session['id']}")

        except Exception as e:
            logging.error(f"Failed to stop recording: {e}")
            self.recording_state = RecordingState.ERROR
            raise

    def _synchronize_device_clocks(self):
        """Synchronize clocks across all connected devices."""
        sync_requests = []

        for device in self.device_manager.get_connected_devices():
            # Send time synchronization request
            sync_cmd = {
                'type': 'TIME_SYNC',
                'server_time': time.time_ns(),
                'sequence': self._get_next_sequence()
            }
            sync_requests.append((device, sync_cmd))

        # Send sync requests and collect responses
        responses = self._send_sync_requests(sync_requests)

        # Calculate clock offsets
        for device_id, response in responses.items():
            offset = self._calculate_clock_offset(response)
            device = self.device_manager.get_device(device_id)
            device.clock_offset = offset

            logging.debug(f"Device {device_id} clock offset: {offset}ms")

    def _calculate_clock_offset(self, sync_response):
        """Calculate clock offset using NTP-like algorithm."""
        t1 = sync_response['t1']  # Client request time
        t2 = sync_response['t2']  # Server receive time
        t3 = sync_response['t3']  # Server response time
        t4 = time.time_ns()       # Client response receive time

        # Calculate round-trip delay and offset
        delay = (t4 - t1) - (t3 - t2)
        offset = ((t2 - t1) + (t3 - t4)) / 2

        return {
            'offset_ns': offset,
            'delay_ns': delay,
            'accuracy_ns': delay / 2
        }

    def _finalize_session(self):
        """Finalize session with quality assessment and metadata."""
        # Generate quality report
        quality_report = self._generate_quality_report()

        # Update session metadata
        metadata_updates = {
            'quality_report': quality_report,
            'file_manifest': self._generate_file_manifest(),
            'synchronization_report': self._generate_sync_report(),
            'completion_status': 'SUCCESS'
        }

        self.current_session.update(metadata_updates)

        # Save final metadata
        metadata_file = os.path.join(self.current_session['path'], 'session_metadata.json')
        with open(metadata_file, 'w') as f:
            json.dump(self.current_session, f, indent=2, default=str)

        # Archive session if configured
        if self.config.get('auto_archive', False):
            self._archive_session()

    def _generate_quality_report(self):
        """Generate comprehensive quality assessment report."""
        return {
            'data_completeness': self._assess_data_completeness(),
            'synchronization_quality': self._assess_sync_quality(),
            'signal_quality': self._assess_signal_quality(),
            'error_summary': self._summarize_errors(),
            'recommendations': self._generate_recommendations()
        }
\end{lstlisting}

\subsection{Network Communication Protocol}

\subsubsection{Device Communication Handler}

\begin{lstlisting}[language=Python, caption=Network Communication Implementation]
class DeviceCommunicationManager:
    def __init__(self, port=8080):
        self.port = port
        self.server = None
        self.connected_devices = {}
        self.message_handlers = {}
        self.sequence_counter = 0

    async def start_server(self):
        """Start WebSocket server for device communication."""
        self.server = await websockets.serve(
            self.handle_device_connection,
            "0.0.0.0",
            self.port,
            ping_interval=10,
            ping_timeout=5
        )
        logging.info(f"Device communication server started on port {self.port}")

    async def handle_device_connection(self, websocket, path):
        """Handle incoming device connections."""
        device_id = None
        try:
            # Device registration handshake
            registration_msg = await websocket.recv()
            registration = json.loads(registration_msg)

            if registration['type'] != 'DEVICE_REGISTER':
                await websocket.close(code=4001, reason="Invalid registration")
                return

            device_id = registration['device_id']
            capabilities = registration['capabilities']

            # Store device connection
            self.connected_devices[device_id] = {
                'websocket': websocket,
                'capabilities': capabilities,
                'status': 'connected',
                'last_seen': time.time(),
                'clock_offset': 0
            }

            logging.info(f"Device {device_id} registered with capabilities: {capabilities}")

            # Send registration acknowledgment
            ack_msg = {
                'type': 'REGISTRATION_ACK',
                'status': 'success',
                'server_time': time.time_ns(),
                'device_id': device_id
            }
            await websocket.send(json.dumps(ack_msg))

            # Handle ongoing communication
            async for message in websocket:
                await self.process_device_message(device_id, message)

        except websockets.exceptions.ConnectionClosed:
            logging.info(f"Device {device_id} disconnected")
        except Exception as e:
            logging.error(f"Error handling device {device_id}: {e}")
        finally:
            if device_id and device_id in self.connected_devices:
                del self.connected_devices[device_id]

    async def process_device_message(self, device_id, message):
        """Process incoming messages from devices."""
        try:
            msg_data = json.loads(message)
            msg_type = msg_data['type']

            # Update device last seen time
            self.connected_devices[device_id]['last_seen'] = time.time()

            # Route message to appropriate handler
            if msg_type in self.message_handlers:
                await self.message_handlers[msg_type](device_id, msg_data)
            else:
                logging.warning(f"No handler for message type: {msg_type}")

        except json.JSONDecodeError:
            logging.error(f"Invalid JSON from device {device_id}: {message}")
        except Exception as e:
            logging.error(f"Error processing message from {device_id}: {e}")

    async def send_to_device(self, device_id, message):
        """Send message to specific device."""
        if device_id not in self.connected_devices:
            raise ValueError(f"Device {device_id} not connected")

        device = self.connected_devices[device_id]
        websocket = device['websocket']

        try:
            # Add message metadata
            message.update({
                'timestamp': time.time_ns(),
                'sequence': self._get_next_sequence(),
                'source': 'desktop_controller'
            })

            await websocket.send(json.dumps(message))

        except websockets.exceptions.ConnectionClosed:
            logging.warning(f"Device {device_id} connection closed during send")
            del self.connected_devices[device_id]
            raise

    async def broadcast_to_devices(self, message, device_filter=None):
        """Broadcast message to all or filtered devices."""
        target_devices = self.connected_devices.items()

        if device_filter:
            target_devices = [(id, dev) for id, dev in target_devices
                            if device_filter(dev)]

        tasks = []
        for device_id, device in target_devices:
            task = self.send_to_device(device_id, message.copy())
            tasks.append(task)

        # Send to all devices concurrently
        results = await asyncio.gather(*tasks, return_exceptions=True)

        # Check for failures
        failed_devices = []
        for i, result in enumerate(results):
            if isinstance(result, Exception):
                device_id = list(dict(target_devices).keys())[i]
                failed_devices.append((device_id, result))
                logging.error(f"Failed to send to {device_id}: {result}")

        return failed_devices

    def register_message_handler(self, message_type, handler):
        """Register handler for specific message types."""
        self.message_handlers[message_type] = handler

    def get_connected_devices(self):
        """Get list of currently connected devices."""
        return list(self.connected_devices.keys())

    def get_device_capabilities(self, device_id):
        """Get capabilities of specific device."""
        if device_id in self.connected_devices:
            return self.connected_devices[device_id]['capabilities']
        return None

    def _get_next_sequence(self):
        """Get next sequence number for message ordering."""
        self.sequence_counter += 1
        return self.sequence_counter
\end{lstlisting}

This appendix provides essential code listings demonstrating the core implementation approaches used in the Multi-Sensor Recording System. The complete source code with additional components, error handling, and documentation is available in the project repository.
