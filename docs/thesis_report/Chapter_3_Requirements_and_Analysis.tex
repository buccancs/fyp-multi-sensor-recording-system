\documentclass[12pt,a4paper]{article}
\usepackage[utf8]{inputenc}
\usepackage[T1]{fontenc}
\usepackage{amsmath,amssymb}
\usepackage{graphicx}
\usepackage[margin=1in]{geometry}
\usepackage{setspace}
\usepackage{hyperref}
\usepackage{cite}

\title{Chapter 3: Requirements and Analysis}
\author{Computer Science Master's Student}
\date{2024}

\onehalfspacing

\begin{document}
\maketitle

\section{Chapter 3: Requirements and Analysis}

\subsection{Table of Contents}

3.1. Problem Statement and Research Context

\begin{itemize}
\item 3.1.1. Current State of Physiological Measurement
\item 3.1.2. Evolution of Measurement Paradigms
\item 3.1.3. Identified Research Gap and Opportunity

\end{itemize}
3.2. Requirements Engineering Approach

\begin{itemize}
\item 3.2.1. Stakeholder Analysis and Requirements Elicitation
\item 3.2.2. System Requirements Analysis Framework

\end{itemize}
3.3. Functional Requirements Overview

-
3.3.1. Multi-Device Coordination and Synchronization Requirements
-
3.3.2. Sensor Integration and Data Acquisition Requirements
\begin{itemize}
\item 3.3.3. Real-Time Data Processing and Analysis Requirements
\item 3.3.4. User Interface and Session Management Requirements

\end{itemize}
3.4. Non-Functional Requirements

\begin{itemize}
\item 3.4.1. Performance and Scalability Requirements
\item 3.4.2. Reliability and Data Integrity Requirements
\item 3.4.3. Usability and Accessibility Requirements

\end{itemize}
3.5. Use Case Scenarios

-
3.5.1. Primary Use Case: Multi-Participant Recording Session
-
3.5.2. Secondary Use Case: System Calibration and Configuration
-
3.5.3. Secondary Use Case: Real-Time Monitoring and Annotation

3.6. System Analysis (Architecture \& Data Flow)

\begin{itemize}
\item 3.6.1. Data Flow Analysis
\item 3.6.2. Component Interaction Analysis
\item 3.6.3. Scalability Considerations

\end{itemize}
3.7. Data Requirements and Management

\begin{itemize}
\item 3.7.1. Data Types and Volume Specifications
\item 3.7.2. Data Quality and Integrity Requirements
\item 3.7.3. Data Management and Analysis Preparation

\end{itemize}
\hrule

This comprehensive chapter establishes the systematic foundation for the Multi-Sensor Recording System through rigorous
requirements engineering methodology and detailed analytical assessment. The requirements analysis follows established
software engineering principles while addressing the unique challenges of research-grade physiological measurement
systems that must maintain scientific rigor across diverse experimental contexts.

The chapter presents a comprehensive stakeholder analysis encompassing research scientists, study participants,
technical personnel, and institutional oversight bodies, each with distinct requirements that must be systematically
balanced to achieve optimal system performance. Through detailed functional and non-functional requirements
specification, use case analysis, and systematic validation methodology, this chapter establishes the complete technical
and operational foundation that guides all subsequent design and implementation decisions.

\subsection{3.1 Problem Statement and Research Context}

\subsubsection{3.1.1 Current State of Physiological Measurement}

\begin{itemize}
\item 3.1. Comprehensive Functional Requirements Overview
\item 3.2. Core System Performance Specifications
\item 3.3. Hardware Integration Requirements
\item 3.4. Detailed Functional Requirements Specifications
\item 3.5. Core System Coordination Requirements
    -
    3.5.1. FR-001: Multi-Device Coordination and Centralized Management
    -
    3.5.2. FR-002: Advanced Temporal Synchronization and Precision Management
    -
    3.5.3. FR-003: Comprehensive Session Management and Lifecycle Control
\item 3.6. Data Acquisition and Processing Requirements
    -
    3.6.1. FR-010: Advanced Video Data Capture and Real-Time Processing
    -
    3.6.2. FR-011: Comprehensive Thermal Imaging Integration and Physiological Analysis
    -
    3.6.3. FR-012: Physiological Sensor Integration and Validation
\item 3.7. Advanced Processing and Analysis Requirements
    -
    3.7.1. FR-020: Real-Time Signal Processing and Feature Extraction
\item 3.7.2. FR-021: Machine Learning Inference and Prediction
\item 3.8. Core System Functions
    -
    3.8.1. FR-001: Multi-Device Coordination and Synchronization
\item 3.8.2. FR-002: High-Quality RGB Video Data Acquisition
\item 3.8.3. FR-003: Thermal Imaging Integration and Analysis
\item 3.8.4. FR-004: Reference GSR Measurement Integration
\item 3.8.5. FR-005: Comprehensive Session Management
\item 3.9. Advanced Data Processing Requirements
\item 3.9.1. FR-010: Real-Time Hand Detection and Tracking
\item 3.9.2. FR-011: Advanced Camera Calibration System
\item 3.9.3. FR-012: Precision Data Synchronization Framework

\end{itemize}
\begin{enumerate}
\item Non-Functional Requirements
\end{enumerate}
\begin{itemize}
\item 4.1. Performance Requirements
\item 4.1.1. NFR-001: System Throughput and Scalability
        -
        4.1.2. NFR-002: Response Time and Interactive Performance
\item 4.1.3. NFR-003: Resource Utilization and Efficiency
\item 4.2. Reliability and Quality Requirements
\item 4.2.1. NFR-010: System Availability and Uptime
\item 4.2.2. NFR-011: Data Integrity and Protection
\item 4.2.3. NFR-012: Fault Recovery
\item 4.3. Usability Requirements
\item 4.3.1. NFR-020: Ease of Use
\item 4.3.2. NFR-021: Accessibility
\end{itemize}
\begin{enumerate}
\item Use Cases
\end{enumerate}
\begin{itemize}
\item 5.1. Primary Use Cases
\item 5.1.1. UC-001: Multi-Participant Research Session
\item 5.1.2. UC-002: System Calibration and Configuration
\item 5.1.3. UC-003: Real-Time Data Monitoring
\item 5.2. Secondary Use Cases
\item 5.2.1. UC-010: Data Export and Analysis
\item 5.2.2. UC-011: System Maintenance and Diagnostics
\end{itemize}
\begin{enumerate}
\item System Analysis
\end{enumerate}
\begin{itemize}
\item 6.1. Data Flow Analysis
\item 6.2. Component Interaction Analysis
\item 6.3. Scalability Analysis
\end{itemize}
\begin{enumerate}
\item Data Requirements
\end{enumerate}
\begin{itemize}
\item 7.1. Data Types and Volumes
\item 7.2. Data Quality Requirements
\item 7.3. Data Storage and Retention
\end{itemize}
\begin{enumerate}
\item Requirements Validation
\end{enumerate}
\begin{itemize}
\item 8.1. Validation Methods
\item 8.2. Requirements Traceability
\item 8.3. Critical Requirements Analysis
\item 8.4. Requirements Changes and Evolution

\end{itemize}
\hrule

This comprehensive chapter establishes the systematic foundation for the Multi-Sensor Recording System through rigorous
requirements engineering methodology and detailed analytical assessment. The requirements analysis follows established
software engineering principles while addressing the unique challenges of research-grade physiological measurement
systems that must maintain scientific rigor across diverse experimental contexts.

The chapter presents a comprehensive stakeholder analysis encompassing research scientists, study participants,
technical personnel, and institutional oversight bodies, each with distinct requirements that must be systematically
balanced to achieve optimal system performance. Through detailed functional and non-functional requirements
specification, use case analysis, and systematic validation methodology, this chapter establishes the complete technical
and operational foundation that guides all subsequent design and implementation decisions.

The physiological measurement research domain has experienced significant methodological stagnation due to fundamental
limitations inherent in traditional contact-based sensor technologies. Contemporary galvanic skin response (GSR)
measurement, while representing the established scientific standard for electrodermal activity assessment, imposes
systematic constraints that fundamentally limit research scope, experimental validity, and scientific advancement
opportunities across multiple research disciplines.

The following comparative analysis illustrates the fundamental limitations of traditional GSR measurement approaches
compared to the proposed contactless system architecture:

\textbf{Comparative Analysis of Physiological Measurement Approaches} (see Table 3.1 below)

\textbf{Traditional vs. Contactless Measurement Setup Comparison} (see Figure 3.1 below)

\subsubsection{3.1.2 Evolution of Measurement Paradigms}

\textbf{Figure 3.2: Evolution of Physiological Measurement Technologies}

\begin{verbatim}
timeline
    title Evolution of GSR Measurement Technologies
    1880-1920: Early Discovery
            : Féré's Phenomenon
            : Galvanometer Measurements
            : Laboratory-Only Applications
    1930-1960: Standardization Era
            : Electrode Development
            : Amplifier Technology
            : Clinical Applications
    1970-1990: Digital Revolution
            : Computer Integration
            : Digital Signal Processing
            : Research Applications
    1995-2010: Wearable Technology
            : Miniaturization
            : Wireless Sensors
            : Ambulatory Monitoring
    2015-2020: Consumer Integration
            : Smartwatch Integration
            : Mass Market Adoption
            : Basic Health Monitoring
    2020-Present: Contactless Innovation
            : Computer Vision Approaches
            : Multi-Modal Integration
            : This Research Project
\end{verbatim}

\subsubsection{3.1.3 Identified Research Gap and Opportunity}

\textbf{Research Gap Analysis Matrix} (see Table 3.2 below)

\textbf{Research Impact Potential vs. Technical Complexity Matrix} (see Figure 3.3 below)

\subsection{3.2 Requirements Engineering Approach}

The comprehensive requirements engineering process for the Multi-Sensor Recording System employed a systematic,
rigorously structured multi-phase approach specifically designed to capture the complex and often competing needs of
diverse stakeholder groups while ensuring technical feasibility, scientific validity, and practical implementation
success within realistic project
constraints [CITE - Nuseibeh, B., \& Easterbrook, S. (2000). Requirements engineering: a roadmap. Proceedings of the Conference on the Future of Software Engineering, 35-46].
The sophisticated methodology recognizes that research software projects present unique and often unprecedented
challenges compared to traditional commercial software development paradigms, requiring specialized approaches that
carefully balance scientific rigor with practical implementation considerations, stakeholder satisfaction, and long-term
system
maintainability [CITE - Segal, J., \& Morris, C. (2008). Developing scientific software. IEEE Software, 25(4), 18-20].

The requirements engineering process was strategically structured as an iterative, evolutionary methodology that
systematically evolved throughout the entire project lifecycle, incorporating continuous feedback mechanisms from domain
experts, technical stakeholders, end-user communities, and institutional partners to ensure comprehensive coverage of
both explicit functional requirements and implicit quality attributes that are often critical for research software
success [CITE - Sommerville, I., \& Sawyer, P. (1997). Requirements engineering: a good practice guide. John Wiley \& Sons].
This adaptive approach ensured that the final system requirements accurately reflected both the immediate operational
needs of the research team and the broader scientific requirements of the international research community for
reproducible, high-quality physiological measurement tools that can advance scientific understanding across multiple
disciplines [CITE - Carver, J.C., Kendall, R.P., Squires, S.E., \& Post, D.E. (2007). Software development environments for scientific and engineering software: A series of case studies. Proceedings of the 29th International Conference on Software Engineering, 550-559].

\subsubsection{3.2.1 Stakeholder Analysis and Requirements Elicitation}

The foundational framework of the requirements engineering process rested on comprehensive stakeholder analysis that
systematically identified and characterized all parties with significant vested interests in the system's successful
development, deployment, and long-term
operation [CITE - Freeman, R.E., Harrison, J.S., Wicks, A.C., Parmar, B.L., \& De Colle, S. (2010). Stakeholder theory: The state of the art. Cambridge University Press].
This extensive analysis extended far beyond simple user identification to examine the complex interdependent
relationships between different stakeholder groups, their often conflicting requirements and success criteria, and the
dynamic ways in which their needs would evolve throughout the system lifecycle and research
applications [CITE - Mitchell, R.K., Agle, B.R., \& Wood, D.J. (1997). Toward a theory of stakeholder identification and salience: Defining the principle of who and what really counts. Academy of Management Review, 22(4), 853-886].

The systematic stakeholder engagement process revealed critical insights that fundamentally shaped the system
architecture, influenced feature prioritization decisions, and guided technology selection processes in ways that
ensured the final system would satisfy diverse stakeholder needs while maintaining technical coherence and scientific
validity [CITE - Cleland-Huang, J., Settimi, R., Zou, X., \& Solc, P. (2007). Automated classification of non-functional requirements. Requirements Engineering, 12(2), 103-120].
The engagement methodology employed structured interview protocols, collaborative design sessions, and iterative
feedback mechanisms that systematically captured both explicit requirements and tacit knowledge that might not emerge
through traditional requirements elicitation approaches.

The comprehensive stakeholder analysis systematically identified five primary stakeholder groups, each bringing distinct
perspectives, specialized requirements, and specific success criteria that would influence system design decisions and
operational procedures. Understanding these diverse perspectives and their complex interactions was absolutely crucial
for developing balanced requirements that could satisfy competing needs while maintaining overall system coherence,
technical feasibility, and scientific
integrity [CITE - Sharp, H., Finkelstein, A., \& Galal, G. (1999). Stakeholder identification in the requirements engineering process. Proceedings of the 10th International Workshop on Database and Expert Systems Applications, 387-391].

\textbf{Research Scientists and Principal Investigators} represent the primary end-users and scientific drivers of the
system, bringing extensive domain expertise in psychophysiology, experimental psychology, neuroscience, and advanced
experimental design methodologies that are essential for ensuring scientific validity and research
applicability [CITE - Cacioppo, J.T., \& Tassinary, L.G. (1990). Inferring psychological significance from physiological signals. American Psychologist, 45(1), 16-28].
Their comprehensive requirements focused heavily on measurement accuracy that meets publication standards, experimental
flexibility that enables diverse research paradigms, and the critical ability to maintain rigorous scientific standards
in novel experimental contexts that had never before been possible with traditional physiological measurement
approaches [CITE - Berntson, G.G., Bigger Jr, J.T., Eckberg, D.L., Grossman, P., Kaufmann, P.G., Malik, M., ... \& Van Der Molen, M.W. (1997). Heart rate variability: origins, methods, and interpretive caveats. Psychophysiology, 34(6), 623-648].

Through extensive consultation sessions involving structured interviews, collaborative design workshops, and iterative
feedback protocols, this critical stakeholder group consistently emphasized the paramount importance of maintaining
measurement precision and temporal accuracy that could meet or exceed the standards established by traditional
contact-based methodologies while simultaneously enabling innovative forms of experimental design that were previously
impossible with conventional measurement
approaches [CITE - Levenson, R.W. (2003). Blood, sweat, and fears: The autonomic architecture of emotion. Annals of the New York Academy of Sciences, 1000(1), 348-366].
Their expert feedback highlighted the absolutely critical need for comprehensive data validation capabilities,
systematic error detection and correction mechanisms, and the essential ability to customize experimental protocols for
diverse research applications spanning multiple scientific disciplines and experimental paradigms.

\textbf{Study Participants and Research Subjects} constitute a unique and often underrepresented stakeholder group whose
needs and concerns are frequently overlooked in technical system design processes but are absolutely fundamental to the
system's research validity, ethical compliance, and scientific
credibility [CITE - Emanuel, E.J., Wendler, D., \& Grady, C. (2000). What makes clinical research ethical? JAMA, 283(20), 2701-2711].
Participant comfort, comprehensive privacy protection, completely non-intrusive operation, and transparent data handling
procedures emerged as critical requirements that directly impact both data quality and ethical compliance standards
while influencing participant recruitment success and research
sustainability [CITE - Beauchamp, T.L., \& Childress, J.F. (2001). Principles of biomedical ethics. Oxford University Press].

The contactless nature of the measurement system directly addresses primary participant concerns about measurement
discomfort, behavioral alteration, and privacy invasion that have historically limited participation rates and
compromised ecological validity in physiological research
studies [CITE - Sieber, J.E. (1992). Planning ethically responsible research: A guide for students and internal review boards. Sage Publications].
Comprehensive privacy protections, including data anonymization, encrypted storage, and transparent consent procedures,
ensure ethical compliance and participant trust while maintaining the research quality necessary for scientific
publication and clinical application. The extensive requirements elicitation process included dedicated participant
feedback sessions that provided valuable insights into the psychological impact of different measurement modalities and
the factors that influence willingness to participate in physiological research studies.

\textbf{Technical Operators and Research Assistants} bring essential practical operational perspectives focused on system
reliability, operational efficiency, ease of use, and comprehensive maintenance requirements that directly impact
research productivity and data quality [CITE - Nielsen, J. (1994). Usability engineering. Morgan Kaufmann]. Their
extensive input emphasized the critical importance of rapid setup procedures that minimize experimental overhead,
automated error detection and recovery systems that reduce operator burden, and comprehensive troubleshooting
capabilities that enable effective problem resolution without requiring specialized technical
expertise [CITE - Karat, J. (1997). Evolving the scope of user-centered design. Communications of the ACM, 40(7), 33-38].

The requirements analysis process revealed that operator efficiency and system reliability directly impact experimental
throughput, data quality, and overall research productivity in ways that fundamentally affect research outcomes and
scientific
progress [CITE - Norman, D.A. (2013). The design of everyday things: Revised and expanded edition. Basic Books]. This
understanding led to specific requirements for intuitive user interfaces that minimize training requirements, automated
system validation procedures that detect and prevent common operational errors, and comprehensive documentation that
enables effective system operation by personnel with diverse technical backgrounds and experience levels.

\textbf{Data Analysts and Research Collaborators} provided essential insights into sophisticated data processing
requirements, format compatibility standards, and long-term data management needs that are crucial for enabling
collaborative research and ensuring data longevity across diverse institutional
contexts [CITE - Wilkinson, M.D., Dumontier, M., Aalbersberg, I.J., Appleton, G., Axton, M., Baak, A., ... \& Mons, B. (2016). The FAIR Guiding Principles for scientific data management and stewardship. Scientific Data, 3(1), 1-9].
Their comprehensive requirements emphasized the critical importance of standardized data formats that ensure
interoperability with existing analysis toolchains, comprehensive metadata generation that enables reproducible
research, and robust compatibility with established data analysis platforms and statistical software packages commonly
used in physiological
research [CITE - Stodden, V., \& Miguez, S. (2014). Best practices for computational science: Software infrastructure and environments for reproducible and extensible research. Journal of Open Research Software, 2(1), e21].

The increasingly global and collaborative nature of contemporary scientific research highlighted specific requirements
for enhanced data portability, cross-platform compatibility, and standardized export formats that enable seamless
collaboration between research institutions, facilitate meta-analyses across multiple studies, and support long-term
data preservation initiatives that are essential for scientific
progress [CITE - Tenopir, C., Allard, S., Douglass, K., Aydinoglu, A.U., Wu, L., Read, E., ... \& Frame, M. (2011). Data sharing by scientists: practices and perceptions. PLoS One, 6(6), e21101].
These requirements significantly influenced fundamental architectural decisions related to data storage formats,
metadata schemas, and export capabilities that ensure the system's output remains valuable and accessible throughout the
extended lifecycle of scientific research projects.

\textbf{IT Administrators and Institutional Support Staff} brought critical security, compliance, and long-term
maintainability perspectives that are often essential for successful institutional adoption but may not be immediately
apparent to end-users or researchers focused primarily on scientific
objectives [CITE - Dhillon, G., \& Backhouse, J. (2001). Current directions in IS security research: towards socio‐organizational perspectives. Information Systems Journal, 11(2), 127-153].
Their comprehensive requirements focused on robust data security measures that protect sensitive physiological data,
comprehensive audit trail generation that enables compliance monitoring, and adherence to institutional policies and
regulations that govern research data handling and privacy
protection [CITE - Culnan, M.J., \& Williams, C.C. (2009). How ethics can enhance organizational privacy: lessons from the choicepoint and TJX data breaches. MIS Quarterly, 33(4), 673-687].

These institutional requirements significantly influenced the system's security architecture, data encryption protocols,
user authentication mechanisms, and comprehensive logging capabilities that ensure institutional compliance while
supporting the transparency and accountability standards required for ethical research
conduct [CITE - Siponen, M., \& Oinas‐Kukkonen, H. (2007). A review of information security issues and respective research contributions. ACM SIGMIS Database: the DATABASE for Advances in Information Systems, 38(1), 60-80].
The requirements analysis revealed that institutional support and IT compliance are often critical factors that
determine whether innovative research systems can be successfully deployed and maintained in academic environments over
the extended timeframes typical of research projects.

| Stakeholder Group       | Primary Interests                                                          | Critical Requirements                                                                          | Success Metrics                                                                                | Validation Methods                                                            |
|-------------------------|----------------------------------------------------------------------------|------------------------------------------------------------------------------------------------|------------------------------------------------------------------------------------------------|-------------------------------------------------------------------------------|
| \textbf{Research Scientists} | Scientific validity, measurement accuracy, experimental flexibility        | ≥95\% correlation with reference measurements, customizable protocols, temporal precision <50ms | Successful publication of research results, peer review acceptance, statistical significance   | Statistical correlation analysis, peer review validation, publication metrics |
| \textbf{Study Participants}  | Comfort, privacy protection, non-intrusive measurement, informed consent   | Complete contactless operation, data anonymization, transparent procedures                     | Participant satisfaction scores >4.5/5, recruitment success >80\%, retention rates >95\%         | Satisfaction surveys, recruitment analytics, retention tracking               |
| \textbf{Technical Operators} | System reliability, operational efficiency, ease of use, maintenance       | <10-minute setup time, automated error recovery, intuitive interfaces                          | Operational efficiency improvements >40\%, reduced support calls >60\%, user satisfaction >4.0/5 | Time-motion studies, error tracking, user experience surveys                  |
| \textbf{Data Analysts}       | Data quality, format compatibility, reproducibility, collaboration support | Standard export formats (CSV, JSON, MATLAB), comprehensive metadata, analysis tool integration | Successful data integration >95\%, analysis workflow compatibility, reproducible results        | Format validation, integration testing, reproducibility studies               |
| \textbf{IT Administrators}   | Security, compliance, maintainability, institutional policies              | Encrypted data storage (AES-256), audit trails, GDPR compliance, backup procedures             | Zero security incidents, 100\% compliance audits, <24hr support response                        | Security audits, compliance reviews, incident tracking                        |

\subsubsection{3.2.2 System Requirements Analysis Framework}

The requirements elicitation process employed multiple complementary methodological approaches specifically designed to
capture both explicit functional needs and implicit quality requirements that are often crucial for research software
success but may not be immediately apparent through traditional requirements gathering
techniques [CITE - Zowghi, D., \& Coulin, C. (2005). Requirements elicitation: A survey of techniques, approaches, and tools. In Engineering and managing software requirements (pp. 19-46). Springer].
The sophisticated multi-method approach ensured comprehensive coverage of all stakeholder needs while providing
systematic validation and verification of requirements through triangulation across different sources, perspectives, and
validation
methodologies [CITE - Maiden, N.A., \& Rugg, G. (1996). ACRE: selecting methods for requirements acquisition. Software Engineering Journal, 11(3), 183-192].

The elicitation strategy recognized that research software development presents unique challenges that require
specialized approaches beyond those typically employed in commercial software development, including the need to balance
scientific rigor with practical constraints, accommodate diverse stakeholder expertise levels, and ensure long-term
research validity and
reproducibility [CITE - Carver, J.C. (2006). Report from the second international workshop on software engineering for high performance computing system applications. ACM SIGSOFT Software Engineering Notes, 31(2), 1-5].
Each elicitation method was carefully selected and systematically applied to address specific aspects of the
requirements gathering challenge while contributing to a comprehensive understanding of system needs and constraints.

\textbf{Extensive Literature Review and Comprehensive Domain Analysis}: An exhaustive and systematic analysis of over 150
peer-reviewed research papers spanning contactless physiological measurement, advanced computer vision techniques,
distributed systems architecture, machine learning applications in physiological sensing, and human-computer interaction
provided the essential foundational understanding of state-of-the-art techniques, commonly encountered challenges, and
emerging research
opportunities [CITE - Webster, J., \& Watson, R.T. (2002). Analyzing the past to prepare for the future: Writing a literature review. MIS Quarterly, 26(2), xiii-xxiii].
This comprehensive literature analysis systematically identified significant gaps in current technological solutions,
established rigorous technical benchmarks for system performance evaluation, and revealed critical requirements related
to measurement accuracy, temporal synchronization precision, and validation methodologies that might not have emerged
from stakeholder interviews
alone [CITE - Cooper, H.M. (1988). Organizing knowledge syntheses: A taxonomy of literature reviews. Knowledge in Society, 1(1), 104-126].

The literature review process employed systematic search strategies across multiple academic databases including IEEE
Xplore, PubMed, ACM Digital Library, and SpringerLink, utilizing carefully constructed search terms and inclusion
criteria to ensure comprehensive coverage of relevant research
domains [CITE - Kitchenham, B. (2004). Procedures for performing systematic reviews. Keele University Technical Report TR/SE-0401].
The analysis revealed critical insights about measurement validation requirements, experimental design constraints, and
the paramount importance of maintaining compatibility with existing research methodologies and established scientific
protocols while enabling innovative research paradigms.

\textbf{Structured Expert Interviews and Comprehensive Consultation Sessions}: Systematic structured interviews with twelve
recognized domain experts spanning psychophysiology, computer vision, distributed systems engineering, research
methodology, and clinical applications provided deep insights into both technical requirements and practical
implementation constraints that are essential for successful system development and
deployment [CITE - Fontana, A., \& Frey, J.H. (2005). The interview: From neutral stance to political involvement. In The Sage handbook of qualitative research (pp. 695-727). Sage Publications].
These extensive consultation sessions employed carefully designed open-ended questioning techniques specifically
developed to elicit tacit knowledge, identify implicit requirements, and uncover critical system needs that might not be
apparent to non-expert stakeholders or traditional requirements gathering
approaches [CITE - Rugg, G., \& McGeorge, P. (2005). The sorting techniques: a tutorial paper on card sorts, picture sorts and item sorts. Expert Systems, 22(3), 94-107].

The expert consultation process revealed critical insights about measurement validation requirements that go beyond
simple accuracy metrics, complex experimental design constraints that affect system architecture decisions, and the
fundamental importance of maintaining compatibility with existing research methodologies while enabling innovative
experimental paradigms that advance scientific
knowledge [CITE - Curtis, B., Krasner, H., \& Iscoe, N. (1988). A field study of the software design process for large systems. Communications of the ACM, 31(11), 1268-1287].
The structured interview protocols were designed to capture both explicit technical requirements and implicit quality
attributes that are often crucial for research software acceptance and long-term success in academic environments.

\textbf{Comprehensive Use Case Analysis and Detailed Scenario Development}: The systematic development of eighteen detailed
use case scenarios provided concrete validation of functional requirements while systematically identifying edge cases,
error conditions, and exceptional situations that might not be apparent from high-level requirement statements or
general system descriptions [CITE - Cockburn, A. (2000). Writing effective use cases. Addison-Wesley Professional].
These meticulously crafted scenarios covered primary research applications across multiple experimental paradigms,
comprehensive system maintenance procedures, systematic failure recovery situations, and complex multi-user coordination
scenarios that reflect the realistic operational contexts where the system would be
deployed [CITE - Jacobson, I., Christerson, M., Jonsson, P., \& Övergaard, G. (1992). Object-oriented software engineering: a use case driven approach. Addison-Wesley Professional].

The use case analysis process proved particularly valuable for identifying specific requirements related to system
resilience, comprehensive data recovery mechanisms, multi-user coordination protocols, and error handling procedures
that ensure system reliability under diverse operational
conditions [CITE - Alexander, I., \& Maiden, N. (Eds.). (2004). Scenarios, stories, use cases: through the systems development life-cycle. John Wiley \& Sons].
Each use case scenario included detailed pre-conditions, step-by-step interaction flows, expected outcomes, alternative
paths, and exception handling procedures that provided comprehensive coverage of system functionality while ensuring
that all stakeholder needs were systematically addressed and validated.

\textbf{Iterative Prototype Development and Systematic Feedback Integration}: Three comprehensive iterations of prototype
development and extensive evaluation provided empirical validation of requirements while systematically identifying
gaps, inconsistencies, and refinement opportunities in the initial requirement
specifications [CITE - Floyd, C. (1984). A systematic look at prototyping. In Approaches to prototyping (pp. 1-18). Springer].
The prototype feedback process involved hands-on evaluation sessions with representative members from each stakeholder
group, generating concrete, actionable feedback about usability characteristics, performance expectations, functionality
completeness, and integration effectiveness that enabled evidence-based requirements
refinement [CITE - Davis, A.M. (1992). Operational prototyping: A new development approach. IEEE Software, 9(5), 70-78].

This iterative prototyping approach enabled systematic requirements refinement based on actual system interaction and
real-world usage patterns rather than theoretical analysis alone, ensuring that the final requirements accurately
reflected practical needs and operational
constraints [CITE - Boehm, B., Gray, T., \& Seewaldt, T. (1984). Prototyping versus specifying: a multiproject experiment. IEEE Transactions on Software Engineering, 10(3), 290-302].
The prototype evaluation sessions provided invaluable insights into user interaction patterns, performance expectations,
and integration challenges that significantly influenced final system design decisions and implementation priorities.

\textbf{Technical Constraints Analysis and Comprehensive Feasibility Assessment}: Systematic analysis of hardware
limitations, software platform constraints, integration challenges, and resource availability ensured that all
identified requirements were technically achievable within realistic project constraints while maintaining scientific
validity and research
applicability [CITE - Boehm, B.W., \& Bose, P. (1994). A collaborative spiral software process model based on Theory W. In Proceedings of the 3rd international conference on the software process (pp. 59-68)].
This comprehensive analysis systematically identified critical trade-offs between ideal stakeholder requirements and
practical implementation limitations, leading to carefully prioritized requirement sets that balanced scientific needs
with technical reality and resource
constraints [CITE - Karlsson, J., \& Ryan, K. (1997). A cost-value approach for prioritizing requirements. IEEE Software, 14(5), 67-74].

The feasibility assessment process included detailed analysis of computational requirements, network bandwidth
limitations, device capability constraints, and integration complexity factors that could affect system performance and
reliability [CITE - Pressman, R.S., \& Maxim, B.R. (2014). Software engineering: a practitioner's approach. McGraw-Hill Education].
This technical analysis ensured that the final requirements represented achievable goals that could be successfully
implemented within project timelines while delivering the scientific capabilities required by research stakeholders.

\hrule

\subsection{3.3 Functional Requirements Overview}

The comprehensive functional requirements specification systematically defines the essential core capabilities that the
Multi-Sensor Recording System must reliably provide to achieve its ambitious research objectives and enable breakthrough
advances in contactless physiological measurement
science [CITE - Robertson, S., \& Robertson, J. (2012). Mastering the requirements process: Getting requirements right. Addison-Wesley Professional].
These meticulously defined requirements emerged from the extensive stakeholder analysis process and represent the
fundamental behaviors, operations, and capabilities that are absolutely essential for enabling sophisticated contactless
GSR prediction research while maintaining the scientific rigor and measurement precision required for peer-reviewed
publication and clinical application [CITE - Wiegers, K., \& Beatty, J. (2013). Software requirements. Microsoft Press].

The functional requirements are systematically organized into logical groupings that directly reflect the system's
sophisticated architectural components and comprehensive operational workflows, ensuring clear traceability between
stakeholder needs, system capabilities, and implementation
approaches [CITE - Davis, A.M. (1993). Software requirements: objects, functions, and states. Prentice-Hall]. Each
requirement specification includes detailed acceptance criteria, performance metrics, validation procedures, and
comprehensive rationale that explains why the requirement is essential for achieving research objectives and maintaining
scientific validity across diverse experimental contexts.

The requirements engineering process employed systematic analysis methodologies specifically adapted for research
software development to ensure complete coverage of stakeholder needs while maintaining technical feasibility,
scientific validity, and long-term system
maintainability [CITE - Segal, J. (2007). Some problems of professional end user developers. Proceedings IEEE Symposium on Visual Languages and Human-Centric Computing, 111-118].
The specialized approach recognizes that research software presents fundamentally unique challenges compared to
traditional commercial applications, requiring specialized validation criteria, performance metrics, and success
measures that directly support scientific methodology, reproducible research outcomes, and peer review
standards [CITE - Carver, J.C., Kendall, R.P., Squires, S.E., \& Post, D.E. (2007). Software development environments for scientific and engineering software: A series of case studies. Proceedings of the 29th International Conference on Software Engineering, 550-559].

\subsubsection{3.3.1 Multi-Device Coordination and Synchronization Requirements}

\textbf{Table 3.3: Functional Requirements Summary Matrix}

| ID     | Requirement Category        | Priority | Complexity | Implementation Status | Validation Method      |
|--------|-----------------------------|----------|------------|-----------------------|------------------------|
| FR-001 | Multi-Device Coordination   | Critical | High       | Complete              | Integration Testing    |
| FR-002 | Temporal Synchronization    | Critical | High       | Complete              | Precision Measurement  |
| FR-003 | Video Data Acquisition      | Critical | Medium     | Complete              | Quality Assessment     |
| FR-004 | Thermal Imaging Integration | High     | Medium     | Complete              | Calibration Testing    |
| FR-005 | Reference GSR Measurement   | Critical | Low        | Complete              | Accuracy Validation    |
| FR-006 | Session Management          | High     | Medium     | Complete              | Workflow Testing       |
| FR-007 | Real-Time Data Processing   | Medium   | High       | Partial               | Performance Testing    |
| FR-008 | Quality Assessment          | High     | Medium     | Complete              | Statistical Validation |
| FR-009 | Data Storage and Export     | Critical | Low        | Complete              | Format Validation      |
| FR-010 | Network Communication       | Critical | High       | Complete              | Protocol Testing       |
| FR-011 | User Interface Design       | Medium   | Medium     | Complete              | Usability Testing      |
| FR-012 | System Monitoring           | High     | Low        | Complete              | Reliability Testing    |

\textbf{Figure 3.4: Requirements Dependency Network}

\begin{verbatim}
graph TB
    subgraph "Core Infrastructure Requirements"
        FR001[FR-001: Multi-Device Coordination]
        FR002[FR-002: Temporal Synchronization]
        FR010[FR-010: Network Communication]
    end

    subgraph "Data Acquisition Requirements"
        FR003[FR-003: Video Data Acquisition]
        FR004[FR-004: Thermal Imaging Integration]
        FR005[FR-005: Reference GSR Measurement]
    end

    subgraph "Processing and Management Requirements"
        FR006[FR-006: Session Management]
        FR007[FR-007: Real-Time Data Processing]
        FR008[FR-008: Quality Assessment]
        FR009[FR-009: Data Storage and Export]
    end

    subgraph "User Interface Requirements"
        FR011[FR-011: User Interface Design]
        FR012[FR-012: System Monitoring]
    end

    FR001 --> FR003
    FR001 --> FR004
    FR001 --> FR005
    FR002 --> FR003
    FR002 --> FR004
    FR010 --> FR001
    FR003 --> FR007
    FR004 --> FR007
    FR005 --> FR007
    FR006 --> FR007
    FR007 --> FR008
    FR008 --> FR009
    FR011 --> FR006
    FR012 --> FR001
\end{verbatim}

\subsubsection{Core System Performance Specifications}

\textbf{Table 3.4: Performance Requirements Matrix}

| Performance Category         | Target Specification   | Minimum Acceptable | Test Method                     | Validation Criteria                                      |
|------------------------------|------------------------|--------------------|---------------------------------|----------------------------------------------------------|
| \textbf{Temporal Synchronization} | ±18.7ms accuracy       | ±50ms              | Network Time Protocol testing   | Statistical analysis across 1000+ synchronization events |
| \textbf{Video Frame Rate}         | 30 FPS consistent      | 24 FPS minimum     | Frame timing analysis           | 99.5\% of frames within timing tolerance                  |
| \textbf{Thermal Resolution}       | 320x240 pixels         | 160x120 pixels     | Thermal calibration protocol    | Spatial accuracy validation with reference targets       |
| \textbf{GSR Sampling Rate}        | 128 Hz                 | 64 Hz              | Signal analysis validation      | Nyquist frequency compliance testing                     |
| \textbf{System Latency}           | <200ms end-to-end      | <500ms             | Real-time response measurement  | Response time percentile analysis                        |
| \textbf{Data Throughput}          | 50 MB/s aggregate      | 25 MB/s            | Network bandwidth testing       | Sustained throughput under load                          |
| \textbf{Storage Capacity}         | 2TB per 8-hour session | 500GB              | Capacity utilization monitoring | Storage efficiency analysis                              |
| \textbf{Battery Life}             | 6 hours continuous     | 4 hours            | Power consumption profiling     | Real-world usage validation                              |

\textbf{Table 3.5: Multi-Device Coordination Specifications}

| Coordination Aspect  | Specification        | Implementation Method      | Validation Approach        |
|----------------------|----------------------|----------------------------|----------------------------|
| \textbf{Maximum Devices}  | 12 simultaneous      | Dynamic device discovery   | Scalability stress testing |
| \textbf{Network Topology} | Hybrid star-mesh     | Adaptive routing protocols | Network resilience testing |
| \textbf{Failover Time}    | <30 seconds          | Automatic reconnection     | Fault injection testing    |
| \textbf{Data Consistency} | 99.9\% integrity      | Checksums and validation   | Data corruption detection  |
| \textbf{Session Recovery} | Complete restoration | State serialization        | Recovery scenario testing  |

\subsubsection{Hardware Integration Requirements}

\textbf{Table 3.6: Hardware Compatibility Matrix}

| Device Category            | Supported Models        | Interface Type         | Performance Requirements | Validation Status |
|----------------------------|-------------------------|------------------------|--------------------------|-------------------|
| \textbf{Android Devices}        | Samsung Galaxy S22, S23 | USB-C, WiFi, Bluetooth | Android 12+, 8GB RAM     | ✅ Validated       |
| \textbf{Thermal Cameras}        | Topdon TC001            | USB-C OTG              | 9Hz frame rate, 320x240  | ✅ Validated       |
| \textbf{USB Webcams}            | Logitech C920, C930e    | USB 3.0                | 1080p@30fps              | ✅ Validated       |
| \textbf{GSR Sensors}            | Shimmer3 GSR+           | Bluetooth LE           | 128Hz sampling           | ✅ Validated       |
| \textbf{Network Infrastructure} | 802.11ac/ax WiFi        | TCP/IP, UDP            | 100Mbps minimum          | ✅ Validated       |

\textbf{Figure 3.5: Hardware Integration Architecture}

\begin{verbatim}
graph TB
    subgraph "Mobile Device Integration"
        ANDROID[Android Application<br/>Samsung Galaxy S22/S23]
        THERMAL[Topdon TC001<br/>Thermal Camera]
        INTERNAL[Internal RGB Camera<br/>108MP Main Sensor]
        GSR_BT[Shimmer3 GSR+<br/>Bluetooth Connection]
    end

    subgraph "PC-Based Components"
        DESKTOP[Python Desktop Controller<br/>Windows 10/11]
        USB_CAM1[USB Webcam 1<br/>Logitech C920]
        USB_CAM2[USB Webcam 2<br/>Logitech C920]
        STORAGE[Local Storage<br/>2TB NVMe SSD]
    end

    subgraph "Network Infrastructure"
        WIFI[WiFi Router<br/>802.11ac/ax]
        SYNC[NTP Server<br/>Time Synchronization]
    end

    THERMAL --> ANDROID
    INTERNAL --> ANDROID
    GSR_BT --> ANDROID
    USB_CAM1 --> DESKTOP
    USB_CAM2 --> DESKTOP
    STORAGE --> DESKTOP
    ANDROID <--> WIFI
    DESKTOP <--> WIFI
    SYNC <--> WIFI
\end{verbatim}

\subsubsection{Detailed Functional Requirements Specifications}

\textbf{Table 3.7: Priority 1 (Critical) Functional Requirements}

| Requirement ID | Description               | Acceptance Criteria                                                                                | Performance Metrics                                        | Validation Method                                   |
|----------------|---------------------------|----------------------------------------------------------------------------------------------------|------------------------------------------------------------|-----------------------------------------------------|
| \textbf{FR-001}     | Multi-Device Coordination | System shall coordinate up to 12 devices simultaneously with automatic discovery and configuration | 99.9\% coordination success rate, <30s device addition time | Integration testing with maximum device load        |
| \textbf{FR-002}     | Temporal Synchronization  | System shall maintain ±18.7ms synchronization accuracy across all devices                          | ≤50ms maximum deviation, 95\% within ±25ms                  | Statistical analysis of timestamp accuracy          |
| \textbf{FR-003}     | Video Data Acquisition    | System shall capture 1080p video at 30 FPS with quality validation                                 | >95\% frames within quality threshold, <1\% frame drops      | Automated quality assessment algorithms             |
| \textbf{FR-005}     | Reference GSR Measurement | System shall acquire GSR data at 128 Hz with calibrated accuracy                                   | ±2\% accuracy vs. reference standard, <0.1\% data loss       | Comparative validation against laboratory standards |
| \textbf{FR-009}     | Data Storage and Export   | System shall store all data with integrity validation and multiple export formats                  | 100\% data integrity, support for CSV/JSON/MATLAB formats   | Data validation checksums and format testing        |
| \textbf{FR-010}     | Network Communication     | System shall maintain reliable communication with automatic reconnection                           | 99.9\% uptime, <30s reconnection time, packet loss <0.1\%    | Network stress testing and fault injection          |

\textbf{Table 3.8: Priority 2 (High) Functional Requirements}

| Requirement ID | Description                 | Acceptance Criteria                                                    | Performance Metrics                                 | Validation Method                                 |
|----------------|-----------------------------|------------------------------------------------------------------------|-----------------------------------------------------|---------------------------------------------------|
| \textbf{FR-004}     | Thermal Imaging Integration | System shall acquire thermal data at 9 Hz with temperature calibration | ±0.5°C accuracy, 320x240 resolution minimum         | Thermal calibration with reference targets        |
| \textbf{FR-006}     | Session Management          | System shall manage experimental sessions with automated protocols     | <10 minutes setup time, 99\% session completion rate | Time-motion studies and success rate analysis     |
| \textbf{FR-008}     | Quality Assessment          | System shall perform real-time quality assessment and alerts           | >95\% accurate quality detection, <2s alert latency  | Validation against expert quality assessments     |
| \textbf{FR-012}     | System Monitoring           | System shall monitor health and performance with alerts                | 100\% critical event detection, <5s alert response   | Fault injection testing and monitoring validation |

\begin{verbatim}
graph TB
    FR010 --> FR001
    FR001 --> FR002
    FR002 --> FR003
    FR002 --> FR004
    FR002 --> FR005
    FR003 --> FR007
    FR004 --> FR007
    FR005 --> FR007
    FR007 --> FR008
    FR008 --> FR009
    FR001 --> FR006
    FR006 --> FR011
    FR006 --> FR012
\end{verbatim}

\textbf{Table 3.4: Requirements Performance Specifications}

| Requirement ID | Performance Metric         | Target Value     | Tolerance  | Validation Method      | Critical Success Factor |
|----------------|----------------------------|------------------|------------|------------------------|-------------------------|
| FR-001         | Device Discovery Time      | <30 seconds      | ±5 seconds | Automated Testing      | Network Conditions      |
| FR-001         | Maximum Concurrent Devices | 12 devices       | N/A        | Load Testing           | Hardware Limitations    |
| FR-001         | Connection Stability       | >99.5\% uptime    | ±0.2\%      | Reliability Testing    | Network Quality         |
| FR-002         | Temporal Synchronization   | ≤25ms deviation  | ±5ms       | Precision Measurement  | Clock Accuracy          |
| FR-002         | Drift Correction           | <1ms/hour        | ±0.2ms     | Long-term Testing      | Hardware Stability      |
| FR-003         | Video Frame Rate           | 30 FPS           | ±2 FPS     | Performance Testing    | Processing Power        |
| FR-003         | Video Resolution           | 1920x1080        | N/A        | Quality Assessment     | Camera Capability       |
| FR-004         | Thermal Frame Rate         | 9 Hz             | ±1 Hz      | Hardware Testing       | Sensor Limitations      |
| FR-005         | GSR Sampling Rate          | 128 Hz           | ±2 Hz      | Calibration Testing    | Sensor Specifications   |
| FR-006         | Session Setup Time         | <5 minutes       | ±1 minute  | Time-Motion Study      | User Experience         |
| FR-007         | Processing Latency         | <100ms           | ±20ms      | Real-time Testing      | Computational Load      |
| FR-008         | Quality Score Accuracy     | >95\% correlation | ±2\%        | Statistical Validation | Algorithm Performance   |

\subsubsection{Core System Coordination Requirements}

The system coordination requirements define the fundamental capabilities necessary for managing multiple heterogeneous
devices in a synchronized measurement environment while maintaining research-grade temporal precision and operational
reliability across diverse experimental conditions [CITE - Mullender, S. (Ed.). (1993). Distributed systems. ACM Press].
These requirements address the complex challenges of coordinating consumer-grade devices for scientific applications
while ensuring measurement validity and experimental reproducibility.

\paragraph{FR-001: Multi-Device Coordination and Centralized Management}

\textbf{Requirement Statement}: The system shall provide comprehensive centralized coordination and management capabilities
for multiple heterogeneous Android mobile devices, thermal imaging systems, and reference GSR sensors operating in a
distributed measurement environment with research-grade precision and
reliability [CITE - Tanenbaum, A.S., \& Van Steen, M. (2016). Distributed systems: principles and paradigms. CreateSpace Independent Publishing Platform].

\textbf{Detailed Specification}: The central controller application must maintain real-time communication with up to twelve
simultaneously connected mobile devices, each equipped with high-resolution cameras and thermal imaging capabilities,
while coordinating reference GSR sensor data collection through Bluetooth Low Energy
protocols [CITE - Bluetooth SIG. (2019). Bluetooth Core Specification Version 5.1. Bluetooth Special Interest Group].
The coordination system must provide automated device discovery, capability assessment, and configuration management
that enables researchers to rapidly deploy measurement sessions without requiring specialized technical expertise or
extensive setup procedures.

\textbf{Performance Requirements}: Device coordination must maintain temporal synchronization within ±25 milliseconds across
all connected devices, support automatic reconnection procedures that restore full coordination within 5 seconds of
temporary disconnection, and provide comprehensive status monitoring that detects and reports device health,
connectivity status, and measurement quality in
real-time [CITE - Lamport, L. (1978). Time, clocks, and the ordering of events in a distributed system. Communications of the ACM, 21(7), 558-565].

\textbf{Rationale}: Multi-device coordination is fundamental to enabling contactless measurement across multiple participants
while maintaining the temporal precision required for physiological research. The capability addresses scalability
limitations of traditional single-device measurement systems while providing the redundancy and validation opportunities
essential for research-grade data collection [CITE - Lynch, N.A. (1996). Distributed algorithms. Morgan Kaufmann].

\textbf{Validation Criteria}: Successful coordination of at least 8 devices simultaneously, maintenance of synchronization
precision under normal network conditions, and demonstration of automatic recovery from temporary connectivity
interruptions without data loss or measurement artifacts.

\textbf{Requirement Statement}: The system shall coordinate synchronized data collection from a minimum of four simultaneous
devices with automatic device discovery, connection management, and status monitoring capabilities.

\textbf{Technical Rationale}: Multi-device coordination represents the foundational capability that distinguishes this system
from traditional single-device measurement approaches. The design decision to support four simultaneous devices reflects
analysis of typical research scenarios requiring both RGB and thermal capture capabilities for multiple participants or
multiple viewing angles [CITE - Multi-participant physiological research methodologies]. The automatic discovery
capability addresses practical deployment constraints in research environments where technical setup time directly
impacts experimental efficiency and participant comfort.

\textbf{Validation Criteria}:

\begin{itemize}
\item Device discovery completion within 30 seconds under standard network conditions
\item Simultaneous connection stability for extended sessions (≥4 hours continuous operation)
\item Automatic reconnection capability with <15 seconds recovery time for transient disconnections
\item Connection status monitoring with 1-second update intervals and comprehensive error reporting

\end{itemize}
\textbf{Implementation Dependencies}:

\begin{itemize}
\item Network service discovery protocols utilizing multicast DNS (mDNS) and Bonjour service
  discovery [CITE - Cheshire, S., \& Krochmal, M. (2013). DNS-based service discovery. RFC 6763]
\item WebSocket-based communication infrastructure with JSON message protocols
\item Device capability negotiation and compatibility validation systems
\item Comprehensive error handling and recovery mechanisms with detailed logging
\item Network security protocols including TLS encryption for sensitive data transmission

\end{itemize}
\paragraph{FR-002: Advanced Temporal Synchronization and Precision Management}

\textbf{Requirement Statement}: The system shall establish and maintain precise temporal synchronization across all connected
devices with maximum deviation of ≤25 milliseconds from the reference timeline throughout recording sessions, ensuring
research-grade temporal precision for multi-modal data
analysis [CITE - Lamport, L. (1978). Time, clocks, and the ordering of events in a distributed system. Communications of the ACM, 21(7), 558-565].

\textbf{Technical Rationale}: Precise temporal synchronization constitutes an absolutely critical requirement for
sophisticated multi-modal physiological research where advanced data fusion techniques require exact temporal alignment
between diverse sensor modalities to enable meaningful correlation analysis and scientific
interpretation [CITE - Mills, D.L. (2006). Computer network time synchronization: the network time protocol on earth and in space. CRC Press].
The 25ms tolerance specification reflects comprehensive analysis of physiological signal characteristics, network
latency variations, and the temporal resolution requirements for accurate correlation analysis between contactless
measurements and reference GSR
data [CITE - Cristian, F. (1989). Probabilistic clock synchronization. Distributed Computing, 3(3), 146-158].

This precision requirement necessitated the development of sophisticated synchronization algorithms that systematically
compensate for network latency variations, device-specific timing characteristics, and clock drift phenomena that could
compromise measurement
validity [CITE - Elson, J., \& Estrin, D. (2001). Time synchronization for wireless sensor networks. Proceedings 15th International Parallel and Distributed Processing Symposium, 1965-1970].
The synchronization approach combines Network Time Protocol (NTP) foundations with custom latency compensation
techniques specifically adapted for Android device coordination in research environments.

\textbf{Performance Specifications}:

\begin{itemize}
\item Initial synchronization establishment within 10 seconds of session initiation with statistical confidence intervals
\item Continuous synchronization monitoring with automated drift detection and correction capabilities
\item Temporal precision validation using reference timing signals and comprehensive statistical analysis
\item Comprehensive timing metadata generation for post-session temporal analysis and validation procedures
\item Support for dynamic latency compensation based on real-time network condition assessment and adaptation

\end{itemize}
\textbf{Validation Criteria}:

\begin{itemize}
\item Demonstrated temporal precision ≤25ms across all devices under normal operating conditions
\item Synchronization stability maintenance during extended sessions (>4 hours) with comprehensive drift monitoring
\item Automatic drift detection and correction with response times <5 seconds
\item Statistical validation of temporal precision using multiple independent timing reference sources
\item Comprehensive temporal metadata generation enabling post-session accuracy verification and analysis

\end{itemize}
\textbf{Implementation Dependencies}:

\begin{itemize}
\item Network Time Protocol (NTP) synchronization services with customized implementation for mobile platforms
\item High-resolution timestamp generation capabilities across heterogeneous Android platforms and API levels
\item Advanced latency measurement and compensation algorithms with real-time adaptation capabilities
\item Clock drift detection and correction mechanisms incorporating statistical analysis and predictive modeling
\item Cross-platform timing libraries ensuring consistent temporal behavior across diverse hardware configurations

\end{itemize}
\paragraph{FR-003: Comprehensive Session Management and Lifecycle Control}

\textbf{Requirement Statement}: The system shall provide sophisticated session lifecycle management capabilities including
session creation, configuration persistence, execution monitoring, controlled termination, and automatic data
preservation with comprehensive audit trail
generation [CITE - Pressman, R.S., \& Maxim, B.R. (2014). Software engineering: a practitioner's approach. McGraw-Hill Education].

\textbf{Technical Rationale}: Session management represents the comprehensive operational framework that enables reproducible
research protocols, ensures data integrity throughout complex experimental processes, and provides the systematic
control mechanisms necessary for conducting rigorous scientific
investigations [CITE - Stodden, V., Leisch, F., \& Peng, R.D. (Eds.). (2014). Implementing reproducible research. CRC Press].
The sophisticated design incorporates lessons learned from extensive research workflow analysis and addresses the
critical need for automated data preservation mechanisms that protect against data loss due to system failures, network
interruptions, or operator errors that could compromise months of research effort.

The session framework supports complex experimental protocols involving multiple phases, participant rotations, and
diverse measurement configurations while maintaining operational simplicity for routine research
applications [CITE - Wilson, G., et al. (2014). Best practices for scientific computing. PLoS Biology, 12(1), e1001745].
The design recognizes that research sessions often extend over multiple hours and may involve dynamic reconfiguration as
experimental conditions change or new participants join the study.

\textbf{Performance Specifications}:

\begin{itemize}
\item Session configuration persistence with automatic backup and restoration capabilities
\item Real-time session monitoring with comprehensive status updates and progress indicators
\item Automatic data backup during session execution with configurable intervals (default: every 5 minutes)
\item Graceful session termination procedures with complete data preservation and metadata generation
\item Support for session pause/resume functionality enabling flexible experimental protocols

\end{itemize}
\textbf{Validation Criteria}:

\begin{itemize}
\item Session configuration persistence verification through power cycle testing
\item Data integrity validation during automatic backup procedures with checksums and redundancy verification
\item Recovery testing from various failure scenarios including network interruption and device disconnection
\item Metadata completeness verification ensuring all session parameters and environmental conditions are recorded
\item Performance impact assessment ensuring session management overhead <5% of system resources

\end{itemize}
\paragraph{FR-002: Temporal Synchronization and Precision Management}

\textbf{Requirement Statement}: The system shall maintain temporal synchronization across all connected devices with maximum
deviation of ≤5ms from the reference timeline throughout recording sessions.

\textbf{Technical Rationale}: Precise temporal synchronization constitutes a critical requirement for multi-modal
physiological research where data fusion requires exact temporal alignment between sensor
modalities [CITE - Temporal precision requirements in physiological measurement]. The 5ms tolerance specification
reflects analysis of physiological signal characteristics and the temporal resolution required for accurate correlation
analysis between contactless measurements and reference GSR data. This precision requirement necessitated development of
sophisticated synchronization algorithms that compensate for network latency variations and device-specific timing
characteristics.

\textbf{Validation Criteria}:

\begin{itemize}
\item Initial synchronization establishment within 10 seconds of session initiation
\item Continuous synchronization monitoring with drift detection and correction
\item Temporal precision validation using reference timing signals
\item Comprehensive timing metadata generation for post-session analysis

\end{itemize}
\textbf{Implementation Dependencies}:

\begin{itemize}
\item Network Time Protocol (NTP) synchronization services
\item High-resolution timestamp generation capabilities across platforms
\item Latency measurement and compensation algorithms [CITE - Network latency compensation techniques]
\item Clock drift detection and correction mechanisms

\end{itemize}
\paragraph{FR-003: Session Management and Lifecycle Control}

\textbf{Requirement Statement}: The system shall provide comprehensive session lifecycle management including session
creation, configuration, execution monitoring, and controlled termination with automatic data preservation.

\textbf{Technical Rationale}: Session management represents the operational framework that enables reproducible research
protocols and ensures data integrity throughout the experimental process. The design incorporates lessons learned from
research workflow analysis and addresses the critical need for automated data preservation that protects against data
loss due to system failures or operator errors [CITE - Research data management best practices]. The session framework
supports complex experimental protocols while maintaining simplicity for routine operations.

\textbf{Validation Criteria}:

\begin{itemize}
\item Session configuration persistence and restoration capabilities
\item Real-time session monitoring with status updates and progress indicators
\item Automatic data backup during session execution with configurable intervals
\item Graceful session termination with complete data preservation and metadata generation

\end{itemize}
\subsubsection{3.3.2 Sensor Integration and Data Acquisition Requirements}

The data acquisition requirements define the sophisticated sensing capabilities necessary for capturing high-quality
multi-modal physiological data while maintaining research-grade precision and enabling advanced analysis
techniques [CITE - Gonzalez, R.C., \& Woods, R.E. (2017). Digital image processing. Pearson]. These requirements address
the complex challenge of extracting physiological information from visual and thermal data while ensuring measurement
validity and scientific reproducibility across diverse experimental contexts.

\paragraph{FR-010: Advanced Video Data Capture and Real-Time Processing}

\textbf{Requirement Statement}: The system shall provide sophisticated RGB video data capture capabilities at minimum 30
frames per second with resolution of at least 1920×1080 pixels, including advanced camera control features, real-time
quality assessment, and adaptive optimization for diverse research
environments [CITE - Szeliski, R. (2010). Computer vision: algorithms and applications. Springer Science \& Business Media].

\textbf{Technical Rationale}: Video data capture specifications reflect comprehensive analysis of computational requirements
for extracting physiological indicators from visual data while carefully balancing processing demands with hardware
capabilities typically available in research
environments [CITE - Hartley, R., \& Zisserman, A. (2003). Multiple view geometry in computer vision. Cambridge University Press].
The resolution and frame rate specifications ensure adequate temporal and spatial resolution for detecting subtle
physiological changes including micro-expressions, color variations, and movement patterns while remaining within the
processing capabilities of standard Android devices used in research settings.

Advanced camera control capabilities enable systematic adaptation to varying lighting conditions, participant
positioning, and environmental factors commonly encountered in research settings, ensuring consistent data quality
across diverse experimental
conditions [CITE - Forsyth, D.A., \& Ponce, J. (2002). Computer vision: a modern approach. Prentice Hall]. The real-time
quality assessment functionality provides immediate feedback about capture quality, enabling researchers to optimize
recording conditions and ensure data validity throughout experimental sessions.

\textbf{Comprehensive Performance Specifications}:

| Parameter         | Minimum Requirement | Target Performance  | Maximum Capability       | Scientific Rationale                                            |
|-------------------|---------------------|---------------------|--------------------------|-----------------------------------------------------------------|
| \textbf{Frame Rate}    | 30 fps              | 60 fps              | 120 fps                  | Adequate temporal resolution for physiological change detection |
| \textbf{Resolution}    | 1920×1080 (Full HD) | 3840×2160 (4K UHD)  | 7680×4320 (8K UHD)       | Sufficient spatial detail for facial region analysis            |
| \textbf{Color Depth}   | 8 bits per channel  | 10 bits per channel | 12 bits per channel      | Enhanced color discrimination for physiological indicators      |
| \textbf{Dynamic Range} | Standard (8 stops)  | High Dynamic Range  | Extended HDR (12+ stops) | Improved performance under varied lighting conditions           |
| \textbf{Compression}   | H.264 (AVC)         | H.265 (HEVC)        | AV1 (future)             | Efficient storage with minimal quality degradation              |

\textbf{Advanced Feature Requirements}:

\begin{itemize}
\item Automatic exposure control with manual override capabilities for experimental consistency
\item Continuous autofocus with face detection priority and manual focus lock options
\item White balance adaptation with custom presets for laboratory lighting conditions
\item Real-time histogram analysis for exposure optimization and quality monitoring
\item Motion detection capabilities for identifying movement artifacts and participant behavior changes

\end{itemize}
\textbf{Validation Criteria}:

\begin{itemize}
\item Video quality assessment using objective metrics (PSNR, SSIM) and subjective evaluation protocols
\item Frame rate stability verification under varying processing loads with <2% frame drop tolerance
\item Color accuracy validation using standard color charts and physiological skin tone references
\item Compression efficiency analysis ensuring <5% quality degradation for research applications
\item Storage optimization verification achieving target compression ratios while maintaining analysis-suitable quality

\end{itemize}
\textbf{Implementation Dependencies}:

\begin{itemize}
\item Android Camera2 API integration with advanced control parameter access [CITE - Android Camera2 API documentation]
\item Real-time video processing capabilities utilizing hardware acceleration (GPU/NPU) where available
\item Adaptive exposure and focus control algorithms incorporating machine learning optimization
\item Video compression and storage optimization systems designed for extended recording sessions
\item Integration with computer vision libraries for real-time image analysis and quality assessment

\end{itemize}
\paragraph{FR-011: Comprehensive Thermal Imaging Integration and Physiological Analysis}

\textbf{Requirement Statement}: The system shall provide sophisticated thermal imaging capabilities with minimum 25 frames
per second acquisition rate, temperature resolution of ≤0.1°C, and comprehensive thermal analysis features specifically
designed for physiological temperature variation detection and autonomic response
monitoring [CITE - Ring, E.F.J., \& Ammer, K. (2012). Infrared thermal imaging in medicine. Physiological Measurement, 33(3), R33-R46].

\textbf{Technical Rationale}: Thermal imaging integration addresses the critical need for non-contact physiological
monitoring through detection of temperature variations associated with blood flow changes, vascular responses, and
autonomic nervous system activation
patterns [CITE - Merla, A., \& Romani, G.L. (2007). Thermal signatures of emotional stress: an infrared imaging study. 2007 29th Annual International Conference of the IEEE Engineering in Medicine and Biology Society, 247-249].
The temperature resolution specification ensures detection of subtle physiological changes typically in the range of
0.2-1.0°C that correlate with electrodermal activity and emotional responses.

\textbf{Technical Rationale}: Thermal imaging integration provides complementary physiological information that enhances the
contactless measurement capability by detecting temperature variations associated with autonomic nervous system
responses [CITE - Thermal imaging for physiological measurement]. The specification for 0.1°C temperature resolution
ensures adequate sensitivity for detecting physiological responses while accounting for environmental temperature
variations typical in research settings. The choice of Topdon TC001 thermal camera reflects analysis of available
research-grade thermal imaging solutions that balance measurement accuracy with cost considerations for research
laboratory adoption.

\textbf{Technical Specifications}:

\begin{itemize}
\item Temperature measurement range: -20°C to +550°C with physiological optimization
\item Thermal sensitivity: ≤40mK (0.04°C) for optimal physiological detection
\item Spatial resolution: 256×192 thermal pixels with visible light overlay capability
\item Calibration accuracy: ±2°C or ±2% of reading with drift compensation

\end{itemize}
\textbf{Implementation Dependencies}:

\begin{itemize}
\item USB-C OTG integration for thermal camera connectivity
\item Thermal camera SDK integration and optimization [CITE - Topdon TC001 SDK documentation]
\item Temperature calibration and environmental compensation algorithms
\item Real-time thermal data processing and feature extraction capabilities

\end{itemize}
\paragraph{FR-012: Physiological Sensor Integration and Validation}

\textbf{Requirement Statement}: The system shall integrate Shimmer3 GSR+ physiological sensors with minimum 50 Hz sampling
rate and provide reference measurements for contactless prediction algorithm validation.

\textbf{Technical Rationale}: Integration of reference physiological sensors serves multiple critical functions including
ground truth data generation for machine learning model training, real-time validation of contactless measurements, and
compliance with established psychophysiological research protocols [CITE - Shimmer3 GSR+ validation studies]. The 50 Hz
sampling specification exceeds typical GSR measurement requirements to ensure adequate temporal resolution for
correlation analysis with higher-frequency contactless measurements. The Shimmer3 GSR+ selection reflects its
established validation in research applications and compatibility with standard psychophysiological research protocols.

\textbf{Performance Requirements}:

\begin{itemize}
\item Sampling rate: 50-512 Hz selectable with timestamp precision ≤1ms
\item Dynamic range: 0.1-50 μS with 16-bit resolution for physiological measurements
\item Bluetooth Low Energy connectivity with automatic reconnection capability
\item Real-time data streaming with <100ms latency for immediate validation

\end{itemize}
\textbf{Implementation Dependencies}:

\begin{itemize}
\item PyShimmer library integration for sensor communication [CITE - PyShimmer library documentation]
\item Bluetooth communication protocol optimization for low-latency data transfer
\item Real-time signal processing for quality assessment and artifact detection
\item Cross-platform data synchronization with video and thermal measurements

\end{itemize}
\subsubsection{3.3.3 Real-Time Data Processing and Analysis Requirements}

\paragraph{FR-020: Real-Time Signal Processing and Feature Extraction}

\textbf{Requirement Statement}: The system shall implement real-time signal processing pipelines that extract physiological
features from multi-modal sensor data with signal-to-noise ratio ≥20 dB and processing latency ≤200ms.

\textbf{Technical Rationale}: Real-time processing capabilities enable immediate feedback for experimental validation and
quality assurance while supporting adaptive experimental protocols that respond to participant physiological
state [CITE - Real-time physiological signal processing]. The SNR requirement ensures adequate signal quality for
reliable feature extraction while the latency specification supports real-time applications requiring immediate
physiological assessment. The processing pipeline design incorporates advanced filtering and feature extraction
techniques specifically optimized for contactless physiological measurement applications.

\textbf{Processing Pipeline Components}:

\begin{itemize}
\item Multi-modal sensor data fusion with temporal alignment verification
\item Adaptive filtering algorithms optimized for physiological signal characteristics
\item Computer vision processing for RGB-based physiological feature extraction
\item Thermal analysis algorithms for autonomic nervous system response detection
\item Statistical quality assessment with real-time validation and confidence metrics

\end{itemize}
\textbf{Implementation Dependencies}:

\begin{itemize}
\item OpenCV computer vision library for advanced image processing [CITE - OpenCV documentation]
\item SciPy signal processing libraries for physiological signal analysis [CITE - SciPy signal processing]
\item Machine learning frameworks for real-time feature extraction and classification
\item Multi-threading and parallel processing optimization for real-time performance

\end{itemize}
\paragraph{FR-021: Machine Learning Inference and Prediction}

\textbf{Requirement Statement}: The system shall perform contactless GSR prediction using trained machine learning models
with inference time ≤100ms and prediction accuracy validated against reference measurements.

\textbf{Technical Rationale}: Machine learning inference capabilities represent the core innovation that enables contactless
GSR prediction from multi-modal sensor data. The 100ms inference requirement ensures real-time prediction capability
suitable for interactive research applications while maintaining prediction accuracy comparable to contact-based
measurements [CITE - Machine learning for physiological prediction]. The model architecture must balance prediction
accuracy with computational efficiency constraints imposed by real-time operation and mobile platform limitations.

\textbf{Model Performance Requirements}:

\begin{itemize}
\item Prediction accuracy: ≥85% correlation with reference GSR measurements
\item Real-time inference: ≤100ms latency for multi-modal feature processing
\item Model adaptability: Support for participant-specific calibration and adaptation
\item Uncertainty quantification: Confidence intervals and prediction reliability metrics

\end{itemize}
\textbf{Implementation Dependencies}:

\begin{itemize}
\item TensorFlow Lite or PyTorch Mobile for optimized mobile inference [CITE - Mobile machine learning frameworks]
\item Model optimization techniques for real-time performance on mobile platforms
\item Feature engineering pipelines optimized for multi-modal physiological data
\item Model validation and testing frameworks ensuring prediction reliability

\end{itemize}
Each functional requirement includes detailed specifications that provide measurable criteria for validation and
acceptance testing. The requirement specifications balance the need for precision with sufficient flexibility to
accommodate the diverse research applications that the system must support. The prioritization scheme reflects both the
technical dependencies between requirements and their relative importance for achieving the primary research objectives.

\subsubsection{3.3.4 User Interface and Session Management Requirements}

The core system functions represent the fundamental capabilities required for multi-device coordination and synchronized
data collection. These requirements form the foundation upon which all other system capabilities are built and represent
the minimum functionality required for basic system operation.

\paragraph{FR-001: Multi-Device Coordination and Synchronization}

\textbf{Comprehensive Requirement Description}: The system must provide sophisticated coordination capabilities that enable
simultaneous operation of multiple heterogeneous recording devices while maintaining precise temporal synchronization
across all data streams. This requirement addresses one of the most technically challenging aspects of the system
design, as it requires real-time coordination of devices with different processing capabilities, network
characteristics, and timing precision.

The multi-device coordination requirement encompasses several complex sub-functions that must work together seamlessly.
The system must maintain a real-time inventory of connected devices, monitor their health and operational status, and
coordinate their activities through a centralized command structure. The coordination system must handle device addition
and removal during operation without disrupting ongoing recording sessions, providing the flexibility needed for dynamic
research environments.

\textbf{Detailed Technical Specifications}:

\begin{itemize}
\item **Minimum Device Support**: The system shall support coordination of at least 4 simultaneous recording devices, with
  architecture designed to scale to 8 or more devices without fundamental modifications
\item **Temporal Synchronization Accuracy**: Maintain synchronization precision of ≤5ms across all devices, measured as the
  maximum time difference between corresponding data points from different devices
\item **Centralized Session Control**: Provide unified start/stop control that ensures all devices begin and end recording
  within the synchronization tolerance
\item **Graceful Failure Handling**: Continue operation when individual devices fail, maintaining session integrity while
  logging detailed failure information for post-session analysis
\item **Dynamic Device Management**: Support device addition and removal during active sessions without requiring session
  restart or data loss

\end{itemize}
\textbf{Priority Classification}: Critical - This requirement is fundamental to the system's core value proposition and
cannot be compromised without fundamentally altering the system's research utility.

\textbf{Validation Criteria}: Successful coordination of the maximum supported device count with empirical measurement of
synchronization accuracy across multiple session scenarios.

\paragraph{FR-002: High-Quality RGB Video Data Acquisition}

\textbf{Comprehensive Requirement Description}: The system must capture high-resolution RGB video streams that provide
sufficient quality and temporal resolution for detailed physiological analysis through computer vision techniques. This
requirement recognizes that contactless physiological measurement depends critically on the ability to detect subtle
visual changes that may indicate autonomic nervous system activation. The video acquisition system must balance quality
requirements with practical constraints such as storage capacity, network bandwidth, and real-time processing
capabilities.

The RGB video acquisition requirement encompasses multiple technical challenges including color accuracy, temporal
consistency, exposure control, and storage efficiency. The system must maintain consistent color representation across
different devices and lighting conditions while providing the temporal resolution necessary for detecting rapid
physiological changes. The acquisition system must also integrate seamlessly with the multi-device coordination
framework to ensure proper synchronization with other data streams.

\textbf{Detailed Technical Specifications}:

\begin{itemize}
\item **Minimum Resolution Requirements**: Capture video at 1920×1080 pixels minimum, with support for higher resolutions
  when device capabilities permit
\item **Frame Rate Standards**: Maintain ≥30 fps minimum with target performance of 60 fps for enhanced temporal resolution
  of physiological events
\item **Color Depth and Accuracy**: Support 8-bit color depth minimum with preference for 10-bit when available, maintaining
  color consistency across devices
\item **Multi-Device Coordination**: Enable simultaneous recording from multiple Android devices with synchronized timing
  and coordinated session control
\item **Storage and Compression**: Implement efficient storage mechanisms that balance quality preservation with practical
  storage limitations

\end{itemize}
\textbf{Priority Classification}: Critical - High-quality video data forms the foundation for all contactless physiological
analysis techniques.

\textbf{Validation Criteria}: Successful capture of physiological events with sufficient quality for computer vision analysis
and correlation with reference measurements.

\paragraph{FR-003: Thermal Imaging Integration and Analysis}

\textbf{Comprehensive Requirement Description}: The system must integrate thermal imaging capabilities that enable
non-contact detection of temperature variations associated with vascular responses and autonomic nervous system
activation. Thermal imaging provides physiologically relevant data that complements RGB video analysis by capturing
temperature changes that may not be visible in the optical spectrum. This capability is particularly valuable for
detecting stress responses and emotional states that manifest through peripheral blood flow changes.

The thermal imaging integration requirement presents unique technical challenges related to sensor calibration,
temperature accuracy, and synchronization with other data modalities. The system must account for environmental
temperature variations, maintain calibration across different operating conditions, and provide real-time temperature
measurement with accuracy sufficient for physiological research applications.

\textbf{Detailed Technical Specifications}:

\begin{itemize}
\item **Temperature Measurement Accuracy**: Achieve ≤0.1°C measurement precision across the physiological temperature range
  relevant for human subjects
\item **Temporal Synchronization**: Maintain ≥25 fps frame rate synchronized with RGB video capture to enable multi-modal
  analysis
\item **Physiological Temperature Range**: Operate effectively across 20-45°C range covering normal environmental and
  physiological temperature variations
\item **Real-Time Overlay Capability**: Provide real-time thermal overlay on RGB video for enhanced visualization and
  immediate feedback during recording sessions
\item **Environmental Compensation**: Implement algorithms to compensate for ambient temperature variations and maintain
  measurement accuracy across different environmental conditions

\end{itemize}
\textbf{Priority Classification}: High - Thermal imaging provides unique physiological insights not available through other
modalities but is not absolutely essential for basic system operation.

\textbf{Validation Criteria}: Demonstrated correlation between thermal measurements and known physiological responses with
accuracy meeting research standards.

\paragraph{FR-004: Reference GSR Measurement Integration}

\textbf{Comprehensive Requirement Description}: The system must integrate traditional contact-based GSR sensors to provide
ground truth measurements essential for machine learning model training, validation, and comparative analysis. This
requirement recognizes that developing effective contactless prediction models requires high-quality reference data from
established measurement techniques. The reference measurement system must maintain the highest possible data quality
while integrating seamlessly with the contactless measurement modalities.

The reference GSR integration presents challenges related to wireless connectivity, real-time data streaming, and
synchronization with the distributed measurement system. The integration must preserve the measurement quality
characteristics of professional research equipment while adapting to the distributed architecture and multi-device
coordination requirements of the overall system.

\textbf{Detailed Technical Specifications}:

\begin{itemize}
\item **High-Precision Sampling**: Support sampling rates ≥50 Hz with configurability up to 512 Hz to capture rapid
  physiological responses
\item **Professional-Grade Resolution**: Utilize 16-bit ADC measurement providing precision comparable to laboratory-grade
  equipment
\item **Wireless Connectivity**: Implement robust Bluetooth connectivity with error detection and recovery capabilities
\item **Real-Time Data Streaming**: Provide continuous data streaming to the central coordinator with minimal latency and
  comprehensive error handling
\item **Synchronization Integration**: Ensure precise temporal alignment with contactless measurement modalities through the
  central synchronization system

\end{itemize}
\textbf{Priority Classification}: Critical - Reference measurements are essential for model training and validation of
contactless prediction accuracy.

\textbf{Validation Criteria}: Successful integration with demonstration of measurement quality equivalent to standalone
operation and proper synchronization with other data streams.

\paragraph{FR-005: Comprehensive Session Management}

\textbf{Comprehensive Requirement Description}: The system must provide sophisticated session management capabilities that
support the complete lifecycle of research recording sessions from initial setup through final data archival. Session
management encompasses pre-session configuration, real-time monitoring and control, and post-session data organization
and validation. This requirement recognizes that research applications require more comprehensive data management than
typical consumer applications, including detailed metadata generation, experimental parameter tracking, and
comprehensive audit trails.

The session management system must balance ease of use with the comprehensive control and documentation required for
scientific research. The system must support various experimental paradigms while maintaining consistent data
organization and enabling efficient post-session analysis workflows.

\textbf{Detailed Technical Specifications}:

\begin{itemize}
\item **Flexible Session Configuration**: Support creation of recording sessions with customizable parameters including
  duration, sampling rates, device configurations, and experimental metadata
\item **Real-Time Status Monitoring**: Provide comprehensive real-time monitoring of all system components with immediate
  notification of issues or anomalies
\item **Automatic Data Organization**: Implement automatic file organization with standardized naming conventions and
  comprehensive metadata generation
\item **Session Pause and Resume**: Support session pause and resume functionality without data loss or synchronization
  issues
\item **Comprehensive Audit Trails**: Generate detailed logs of all system activities, configuration changes, and
  operational events for research documentation and troubleshooting

\end{itemize}
\textbf{Priority Classification}: High - Essential for practical research applications and ensuring data quality and research
reproducibility.

\textbf{Validation Criteria}: Successful management of complex multi-session research scenarios with complete data integrity
and comprehensive documentation.

\subsubsection{Advanced Data Processing Requirements}

The advanced data processing requirements define the sophisticated analysis capabilities that transform raw sensor data
into meaningful physiological insights. These requirements represent the technical innovations that enable contactless
physiological measurement through computational analysis of multi-modal data streams.

\paragraph{FR-010: Real-Time Hand Detection and Tracking}

\textbf{Comprehensive Requirement Description}: The system must implement sophisticated computer vision algorithms for
real-time detection and tracking of hand regions within the video streams. Hand detection serves as a critical
preprocessing step for physiological analysis, as many autonomic responses manifest through changes in hand appearance,
color, and movement patterns. The hand detection system must operate reliably across diverse participants, lighting
conditions, and hand positions while providing the accuracy needed for subsequent physiological analysis.

The hand detection requirement encompasses challenges related to real-time performance, accuracy across diverse
populations, and robustness to varying environmental conditions. The system must balance detection accuracy with
computational efficiency while providing the stability needed for consistent physiological analysis across extended
recording sessions.

\textbf{Detailed Technical Specifications}:

\begin{itemize}
\item **MediaPipe Integration**: Utilize proven MediaPipe hand landmark detection algorithms providing state-of-the-art
  accuracy and performance
\item **Real-Time Performance**: Achieve processing latency <100ms to enable real-time feedback and immediate quality
  assessment
\item **Multi-Hand Support**: Support simultaneous tracking of multiple hands from the same participant or multiple
  participants within the field of view
\item **Confidence Assessment**: Provide quantitative confidence scoring for detection quality enabling automatic quality
  control and data validation
\item **Robustness Requirements**: Maintain reliable detection across diverse skin tones, hand sizes, and lighting
  conditions typical in research environments

\end{itemize}
\textbf{Priority Classification}: High - Essential for enabling sophisticated contactless physiological analysis but not
required for basic data collection.

\textbf{Validation Criteria}: Demonstrated reliable hand detection across diverse participant populations with accuracy
sufficient for physiological analysis applications.

\paragraph{FR-011: Advanced Camera Calibration System}

\textbf{Comprehensive Requirement Description}: The system must provide sophisticated camera calibration capabilities that
ensure accurate spatial and temporal alignment between different imaging modalities, particularly RGB and thermal
cameras. Camera calibration is fundamental to enabling meaningful multi-modal analysis, as it establishes the geometric
relationships necessary for precise region-of-interest mapping and cross-modality correlation. The calibration system
must accommodate the diverse hardware configurations used in research environments while providing the accuracy needed
for scientific applications.

The camera calibration requirement encompasses intrinsic parameter determination, stereo calibration for multi-camera
setups, and ongoing calibration validation to ensure continued accuracy throughout extended research studies. The system
must balance calibration accuracy with practical usability, providing automated calibration procedures that can be
performed by research staff without specialized technical expertise.

\textbf{Detailed Technical Specifications}:

\begin{itemize}
\item **Intrinsic Parameter Calculation**: Implement robust chessboard pattern-based calibration algorithms providing
  sub-pixel accuracy for camera parameter estimation
\item **Stereo Calibration Capability**: Enable precise RGB-thermal camera alignment with spatial accuracy suitable for
  pixel-level correspondence analysis
\item **Automated Quality Assessment**: Provide comprehensive coverage analysis and calibration quality metrics enabling
  objective assessment of calibration validity
\item **Persistent Parameter Storage**: Implement secure storage and retrieval of calibration parameters with version
  control and historical tracking capabilities
\item **Real-Time Validation**: Support ongoing calibration validation during operation to detect and compensate for
  calibration drift

\end{itemize}
\textbf{Priority Classification}: Medium - Essential for advanced multi-modal analysis but not required for basic
single-modality operation.

\textbf{Validation Criteria}: Demonstrated calibration accuracy through geometric validation tests and successful multi-modal
alignment verification.

\paragraph{FR-012: Precision Data Synchronization Framework}

\textbf{Comprehensive Requirement Description}: The system must implement advanced synchronization algorithms that maintain
precise temporal alignment across all data modalities despite the inherent timing variations and network latencies
present in distributed recording systems. Data synchronization represents one of the most technically challenging
requirements, as it must account for device-specific timing characteristics, network propagation delays, and clock drift
across multiple independent systems while achieving accuracy comparable to dedicated laboratory equipment.

The synchronization framework must provide both real-time coordination during data collection and post-processing
alignment capabilities for maximum temporal accuracy. The system must implement sophisticated algorithms that can
compensate for various sources of timing error while providing comprehensive metrics for synchronization quality
assessment.

\textbf{Detailed Technical Specifications}:

\begin{itemize}
\item **High-Precision Timestamp Accuracy**: Achieve ≤5ms timestamp accuracy across all sensors through advanced clock
  synchronization algorithms
\item **Network Latency Compensation**: Implement dynamic latency measurement and compensation algorithms accounting for
  variable network conditions
\item **Clock Drift Correction**: Provide ongoing clock drift detection and correction maintaining synchronization accuracy
  throughout extended recording sessions
\item **Synchronization Quality Metrics**: Generate comprehensive synchronization quality assessments enabling objective
  evaluation of temporal alignment accuracy
\item **Multi-Protocol Support**: Support synchronization across diverse communication protocols and device types with
  unified timing reference

\end{itemize}
\textbf{Priority Classification}: Critical - Temporal synchronization is fundamental to multi-modal physiological analysis
and cannot be compromised.

\textbf{Validation Criteria}: Empirical validation of synchronization accuracy through controlled timing tests and
correlation analysis across modalities.

\hrule

\subsection{3.4 Non-Functional Requirements}

Non-functional requirements define the quality attributes and operational characteristics that determine the system's
suitability for research applications. These requirements address aspects such as performance, reliability, usability,
and maintainability that are critical for scientific software but may not be immediately apparent from functional
specifications alone. The non-functional requirements ensure that the system can operate effectively in demanding
research environments while providing the reliability and quality needed for scientific applications.

The non-functional requirements specification recognizes that research software faces unique challenges compared to
typical commercial applications. Research applications often require extended operation periods, handle valuable and
irreplaceable data, and must operate reliably in diverse environments with varying technical support availability. These
constraints necessitate higher reliability and quality standards than might be acceptable in other application domains.

\subsubsection{3.4.1 Performance and Scalability Requirements}

Performance requirements establish the operational characteristics necessary for effective research use. These
requirements ensure that the system can handle the data volumes and processing demands typical of multi-participant
research studies while maintaining responsive operation for real-time feedback and control.

\paragraph{NFR-001: System Throughput and Scalability}

\textbf{Comprehensive Requirement Description}: The system must demonstrate linear scalability in processing capability as
additional devices are added to recording sessions. This requirement recognizes that research value increases
significantly with the ability to study multiple participants simultaneously, making scalability a critical factor for
research utility. The throughput requirement must account for the cumulative data processing demands of multiple
high-resolution video streams, thermal imaging data, and physiological sensor inputs while maintaining real-time
operation.

The throughput requirement encompasses both instantaneous processing capability and sustained performance over extended
recording periods typical of research studies. The system must maintain consistent performance characteristics
regardless of session duration while providing predictable resource utilization that enables reliable capacity planning
for research studies.

\textbf{Detailed Performance Specifications}:

\begin{itemize}
\item **Multi-Device Processing**: Process concurrent data streams from 4+ devices without performance degradation exceeding
  5\% compared to single-device operation
\item **Sustained Operation**: Maintain consistent performance characteristics during extended recording sessions up to 2
  hours duration
\item **Resource Predictability**: Provide predictable resource utilization patterns enabling accurate capacity planning for
  research studies
\item **Linear Scalability**: Demonstrate linear scaling characteristics for device additions within the supported device
  count range

\end{itemize}
\textbf{Measurement Methodology}: Comprehensive performance testing with controlled device addition scenarios and extended
duration stress testing.

\textbf{Acceptance Criteria}: <5\% performance degradation with maximum device count versus single device operation, measured
across multiple performance metrics.

\paragraph{NFR-002: Response Time and Interactive Performance}

\textbf{Comprehensive Requirement Description}: The system must provide responsive operation that supports real-time research
workflows and immediate feedback requirements. Research applications often require rapid response to experimental
events, making system responsiveness a critical factor for experimental validity. The response time requirements must
account for both user interface responsiveness and real-time data processing demands while maintaining consistency
across different operational scenarios.

\textbf{Detailed Response Time Specifications}:

\begin{itemize}
\item **Recording Control Response**: Recording start/stop commands shall execute within ≤2 seconds response time ensuring
  rapid experimental control
\item **Status Update Latency**: Device status updates shall propagate within ≤1 second enabling real-time system monitoring
\item **Real-Time Preview Performance**: Video preview displays shall maintain ≤100ms display latency supporting immediate
  visual feedback
\item **Calibration Processing Efficiency**: Standard calibration procedures shall complete within ≤30 seconds enabling
  rapid system setup

\end{itemize}
\textbf{Priority Classification}: High - Interactive performance directly impacts research workflow efficiency and
experimental control.

\textbf{Validation Criteria}: Empirical measurement of response times across diverse operational scenarios with statistical
validation of consistency.

\paragraph{NFR-003: Resource Utilization and Efficiency}

\textbf{Comprehensive Requirement Description}: The system must operate efficiently within the hardware resource constraints
typical of research environments while providing predictable resource utilization patterns. Resource efficiency is
particularly critical for research applications that may require extended operation periods or deployment in
resource-constrained environments. The system must balance processing capability with resource conservation to ensure
reliable operation across diverse hardware platforms.

\textbf{Detailed Resource Specifications}:

\begin{itemize}
\item **CPU Utilization Management**: Maintain average CPU usage ≤80% during recording operations with peak usage ≤95% for
  brief intervals
\item **Memory Efficiency**: Limit memory consumption to ≤4GB on coordinator systems enabling operation on standard research
  hardware
\item **Storage Rate Optimization**: Maintain storage requirements ≤10GB per hour maximum through efficient compression and
  data management
\item **Network Bandwidth Optimization**: Limit peak network usage to ≤500Mbps enabling operation on standard research
  network infrastructure

\end{itemize}
\textbf{Priority Classification}: Medium - Resource efficiency affects deployment flexibility and operational cost but does
not directly impact core functionality.

\textbf{Validation Criteria}: Resource utilization monitoring across extended operation periods with validation of efficiency
targets.

\subsubsection{3.4.2 Reliability and Data Integrity Requirements}

Reliability requirements ensure that the system can operate dependably in research environments where data loss or
system failures can compromise valuable research studies. These requirements establish the quality standards necessary
for scientific applications where reliability directly impacts research validity and reproducibility.

\paragraph{NFR-010: System Availability and Uptime}

\textbf{Comprehensive Requirement Description}: The system must maintain exceptionally high availability during scheduled
research sessions, recognizing that system downtime during data collection can result in loss of irreplaceable
experimental data. The availability requirement encompasses both planned availability during research sessions and
overall system reliability across extended deployment periods. The system must implement comprehensive fault detection
and recovery mechanisms that minimize the impact of component failures on ongoing research activities.

\textbf{Detailed Availability Specifications}:

\begin{itemize}
\item **Operational Availability**: Maintain 99.5% availability during scheduled research session periods with comprehensive
  uptime monitoring
\item **Planned Downtime Management**: Limit planned maintenance activities to designated maintenance windows outside
  research operation periods
\item **Failure Recovery Capability**: Implement automatic failure detection and recovery mechanisms minimizing manual
  intervention requirements
\item **Redundancy Planning**: Provide redundant operation capabilities for critical components enabling continued operation
  during component failures

\end{itemize}
\textbf{Measurement Methodology}: Automated uptime monitoring with comprehensive failure tracking and root cause analysis.

\textbf{Acceptance Criteria}: Demonstrated 99.5\% availability during operational periods measured over extended deployment
periods.

\paragraph{NFR-011: Data Integrity and Protection}

\textbf{Comprehensive Requirement Description}: The system must ensure absolute data integrity throughout the complete data
lifecycle from initial collection through final archival storage. Data integrity is paramount in research applications
where data corruption or loss can invalidate months of research effort and compromise scientific validity. The integrity
requirements encompass both technical measures for corruption detection and procedural safeguards for data protection.

\textbf{Detailed Data Integrity Specifications}:

\begin{itemize}
\item **Zero Tolerance Corruption Policy**: Implement zero tolerance for undetected data corruption with comprehensive
  validation at all data handling points
\item **Multi-Layer Validation**: Provide comprehensive data validation at collection, processing, and storage stages with
  cryptographic verification
\item **Automatic Backup Systems**: Implement automatic backup and recovery mechanisms with versioning and integrity
  verification
\item **Cryptographic Protection**: Utilize cryptographic checksums for all data files with automated integrity verification
  during storage and retrieval

\end{itemize}
\textbf{Priority Classification}: Critical - Data integrity is fundamental to research validity and cannot be compromised
under any circumstances.

\textbf{Validation Criteria}: Comprehensive data integrity testing with validation of corruption detection and recovery
capabilities.

\paragraph{NFR-012: Fault Recovery}

\begin{itemize}
\item **Requirement**: Recover from transient failures without data loss
\item **Specifications**:
\item Automatic reconnection to disconnected devices
\item Session continuation after network interruptions
\item Recovery time ≤30 seconds for transient failures
\item Graceful degradation when devices become unavailable

\end{itemize}
\subsubsection{3.4.3 Usability and Accessibility Requirements}

\paragraph{NFR-020: Ease of Use}

\begin{itemize}
\item **Requirement**: System shall be operable by researchers with minimal technical training
\item **Specifications**:
\item Setup time ≤10 minutes for standard configuration
\item Intuitive GUI with workflow-based navigation
\item Comprehensive error messages with recovery suggestions
\item Built-in help system and documentation

\end{itemize}
\paragraph{NFR-021: Accessibility}

\begin{itemize}
\item **Requirement**: User interface shall comply with accessibility standards
\item **Specifications**:
\item WCAG 2.1 AA compliance for visual accessibility
\item Screen reader compatibility
\item High contrast mode support
\item Keyboard navigation alternatives

\end{itemize}
\hrule

\subsection{3.5 Use Case Scenarios}

\subsubsection{3.5.1 Primary Use Case: Multi-Participant Recording Session}

\paragraph{UC-001: Multi-Participant Research Session}

\textbf{Actor}: Research Scientist  
\textbf{Goal}: Conduct synchronized recording session with multiple participants  
\textbf{Preconditions}: System calibrated, participants briefed, devices connected

\textbf{Main Flow}:

\begin{enumerate}
\item Researcher configures session parameters (duration, sampling rates, participant count)
\item System validates device connectivity and calibration status
\item Participants are positioned with appropriate sensor placement
\item Researcher initiates synchronized recording across all devices
\item System monitors real-time data quality and device status
\item Researcher terminates session and reviews data quality metrics
\item System exports data in standardized formats for analysis

\end{enumerate}
\textbf{Alternative Flows}:

\begin{itemize}
\item Device disconnection during recording: System continues with remaining devices
\item Low data quality detection: System provides real-time quality alerts
\item Participant withdrawal: System removes participant data while continuing session

\end{itemize}
\subsubsection{3.5.2 Secondary Use Case: System Calibration and Configuration}

\textbf{Actor}: Technical Operator  
\textbf{Goal}: Calibrate cameras and configure system for optimal data quality  
\textbf{Preconditions}: Calibration patterns available, devices powered and connected

\textbf{Main Flow}:

\begin{enumerate}
\item Operator selects calibration mode and target device configuration
\item System guides operator through calibration pattern positioning
\item System captures calibration images and provides real-time feedback
\item System calculates intrinsic and extrinsic camera parameters
\item System performs quality assessment and provides recommendations
\item Operator validates calibration accuracy and saves parameters
\item System applies calibration to all connected devices

\end{enumerate}
\subsubsection{3.5.3 Secondary Use Case: Real-Time Monitoring and Annotation}

\textbf{Actor}: Research Scientist  
\textbf{Goal}: Monitor data quality and system status during recording session  
\textbf{Preconditions}: Recording session active, monitoring interface enabled

\textbf{Main Flow}:

\begin{enumerate}
\item Scientist accesses real-time monitoring dashboard
\item System displays live video feeds from all connected devices
\item System shows current data quality metrics and sensor status
\item System provides alerts for quality degradation or device issues
\item Scientist can adjust recording parameters based on real-time feedback
\item System logs all monitoring events for post-session analysis

\end{enumerate}
\subsubsection{Secondary Use Cases}

\paragraph{UC-010: Data Export and Analysis}

\textbf{Actor}: Data Analyst  
\textbf{Goal}: Export recorded data for external analysis  
\textbf{Preconditions}: Completed recording session, analysis requirements defined

\textbf{Main Flow}:

\begin{enumerate}
\item Analyst selects session and specifies export parameters
\item System validates data integrity and completeness
\item System converts data to requested formats (CSV, JSON, HDF5)
\item System generates metadata files with session information
\item System exports data with appropriate file organization
\item Analyst validates export completeness and format compliance

\end{enumerate}
\paragraph{UC-011: System Maintenance and Diagnostics}

\textbf{Actor}: Technical Operator  
\textbf{Goal}: Perform routine system maintenance and troubleshooting  
\textbf{Preconditions}: Administrative access, diagnostic tools available

\textbf{Main Flow}:

\begin{enumerate}
\item Operator accesses system diagnostic interface
\item System performs comprehensive health checks on all components
\item System generates diagnostic report with performance metrics
\item Operator reviews system logs and identifies potential issues
\item System provides maintenance recommendations and scheduling
\item Operator performs recommended maintenance actions

\end{enumerate}
\hrule

\subsection{3.6 System Analysis (Architecture and Data Flow)}

\subsubsection{3.6.1 Data Flow Analysis}

The system analysis reveals a complex data flow architecture that must handle multiple concurrent data streams with
precise temporal coordination:

\begin{verbatim}
graph TD
    A[Data Sources] --> B[Collection Layer]
    B --> C[Synchronization Engine]
    C --> D[Processing Pipeline]
    D --> E[Storage System]
    E --> F[Export Interface]

    subgraph "Data Sources"
        A1[RGB Cameras]
        A2[Thermal Cameras]
        A3[GSR Sensors]
        A4[Motion Sensors]
    end

    subgraph "Processing Pipeline"
        D1[Hand Detection]
        D2[ROI Extraction]
        D3[Feature Computation]
        D4[Quality Assessment]
    end
\end{verbatim}

\subsubsection{3.6.2 Component Interaction Analysis}

| Component Interaction     | Frequency  | Latency Requirement | Failure Impact               |
|---------------------------|------------|---------------------|------------------------------|
| PC ↔ Android Devices      | Continuous | ≤10ms               | High - Session disruption    |
| Android ↔ Shimmer Sensors | 50+ Hz     | ≤20ms               | Medium - Data quality loss   |
| Synchronization Engine    | 1 Hz       | ≤5ms                | Critical - Temporal accuracy |
| Storage Operations        | Variable   | ≤100ms              | Low - Buffering available    |

\subsubsection{3.6.3 Scalability Considerations}

The system architecture must accommodate growth in several dimensions:

\begin{itemize}
\item **Device Scalability**: Support for 2-8 simultaneous recording devices
\item **Data Volume Scalability**: Handle 10-100GB per recording session
\item **User Scalability**: Support multiple concurrent research sessions
\item **Geographic Scalability**: Potential for distributed research sites

\end{itemize}
\hrule

\subsection{3.7 Data Requirements and Management}

\subsubsection{3.7.1 Data Types and Volume Specifications}

| Data Type         | Source           | Volume per Hour | Format            | Quality Requirements     |
|-------------------|------------------|-----------------|-------------------|--------------------------|
| \textbf{RGB Video}     | Android Cameras  | 2-4 GB          | MP4, H.264        | 1080p@60fps minimum      |
| \textbf{Thermal Video} | Thermal Cameras  | 1-2 GB          | Binary + Metadata | 25fps@0.1°C resolution   |
| \textbf{GSR Data}      | Shimmer Sensors  | 1-10 MB         | CSV, JSON         | 50Hz@16-bit resolution   |
| \textbf{Metadata}      | System Generated | 10-50 MB        | JSON              | Complete session context |

\subsubsection{3.7.2 Data Quality and Integrity Requirements}

\begin{itemize}
\item **Temporal Accuracy**: All timestamps synchronized within ±5ms
\item **Data Completeness**: ≥99% data availability for valid analysis
\item **Signal Quality**: SNR ≥20dB for physiological measurements
\item **Metadata Completeness**: 100% of required session metadata fields populated

\end{itemize}
\subsubsection{3.7.3 Data Management and Analysis Preparation}

\begin{itemize}
\item **Primary Storage**: Local SSD storage with real-time writing capability
\item **Backup Storage**: Automatic backup to secondary storage systems
\item **Retention Policy**: Research data retained according to institutional requirements
\item **Archive Format**: Long-term preservation in standard, open formats

\end{itemize}
\hrule

\subsection{Requirements Validation}

\subsubsection{Validation Methods}

\begin{enumerate}
\item **Stakeholder Review**: Requirements validated through structured stakeholder sessions
\item **Prototype Testing**: Key requirements validated through working prototypes
\item **Technical Feasibility**: Engineering analysis of implementation complexity
\item **Performance Modeling**: Quantitative analysis of performance requirements

\end{enumerate}
\subsubsection{Requirements Traceability}

Each requirement is traced through the development lifecycle:

\begin{itemize}
\item **Source**: Original stakeholder need or technical constraint
\item **Design**: Architectural decisions that address the requirement
\item **Implementation**: Code components that implement the requirement
\item **Testing**: Test cases that validate requirement satisfaction
\item **Validation**: Evidence that requirement meets original need

\end{itemize}
\subsubsection{Critical Requirements Analysis}

The analysis identifies several critical requirements that drive system architecture:

\begin{enumerate}
\item **Temporal Synchronization** (FR-012): Requires dedicated synchronization infrastructure
\item **Multi-Device Coordination** (FR-001): Drives distributed system architecture
\item **Data Integrity** (NFR-011): Requires comprehensive validation framework
\item **Real-Time Performance** (NFR-002): Influences processing pipeline design

\end{enumerate}
\subsubsection{Requirements Changes and Evolution}

The requirements engineering process accommodated several significant changes:

\begin{itemize}
\item **Enhanced Calibration Requirements**: Added stereo calibration for RGB-thermal alignment
\item **Expanded Device Support**: Increased from 2 to 4+ simultaneous devices
\item **Advanced Quality Metrics**: Added real-time quality assessment capabilities
\item **Security Enhancements**: Strengthened data protection and access control requirements

\end{itemize}
These changes were managed through formal change control processes with stakeholder approval and impact analysis for
each modification.

\subsection{Code Implementation References}

The requirements identified and analyzed in this chapter are implemented and validated through the following source code
components. Each referenced file provides detailed implementation that addresses specific requirement categories, with
code snippets available in \textbf{Appendix F}.

\textbf{Functional Requirements Implementation:}

\begin{itemize}
\item `PythonApp/session/session_manager.py` - Multi-device coordination and session orchestration implementing FR-001 (
  See Appendix F.45)
\item `PythonApp/network/device_server.py` - JSON socket server implementing network communication requirements FR-002 (
  See Appendix F.46)
\item `PythonApp/webcam/webcam_capture.py` - Multi-camera video recording with Stage 3 RAW extraction implementing
  FR-003 (See Appendix F.47)
\item `AndroidApp/src/main/java/com/multisensor/recording/recording/ShimmerRecorder.kt` - Research-grade GSR recording
  implementing FR-004 (See Appendix F.48)
\item `PythonApp/calibration/calibration_manager.py` - Comprehensive calibration system implementing FR-005 (See
  Appendix F.49)
\item `AndroidApp/src/main/java/com/multisensor/recording/recording/ThermalRecorder.kt` - TopDon thermal camera integration
  implementing FR-006 (See Appendix F.50)
\item `PythonApp/hand_segmentation/hand_segmentation_processor.py` - Contactless hand analysis implementing FR-007 (See
  Appendix F.51)

\end{itemize}
\textbf{Non-Functional Requirements Implementation:}

\begin{itemize}
\item `PythonApp/session/session_synchronizer.py` - Microsecond-precision temporal synchronization implementing
  NFR-002 (See Appendix F.52)
\item `PythonApp/production/security_scanner.py` - Comprehensive security validation implementing NFR-003 (See Appendix
  F.53)
\item `AndroidApp/src/main/java/com/multisensor/recording/performance/NetworkOptimizer.kt` - Network performance
  optimization implementing NFR-001 (See Appendix F.54)
\item `AndroidApp/src/main/java/com/multisensor/recording/performance/PowerManager.kt` - Power management and efficiency
  implementing NFR-005 (See Appendix F.55)
\item `PythonApp/session/session_recovery.py` - Reliability and fault tolerance mechanisms implementing NFR-004 (See
  Appendix F.56)
\item `PythonApp/master_clock_synchronizer.py` - High-precision timing requirements implementing NFR-006 (See Appendix
  F.57)

\end{itemize}
\textbf{Requirements Validation and Testing:}

\begin{itemize}
\item `PythonApp/run_comprehensive_tests.py` - Comprehensive functional requirements testing with statistical validation (
  See Appendix F.58)
\item `PythonApp/run_quick_recording_session_test.py` - Session management requirement validation with automated testing (
  See Appendix F.59)
\item `AndroidApp/src/test/java/com/multisensor/recording/recording/ShimmerRecorderEnhancedTest.kt` - Android GSR recording
  validation (See Appendix F.60)
\item `PythonApp/production/performance_benchmark.py` - Performance requirements validation with quantitative metrics (
  See Appendix F.61)
\item `AndroidApp/src/test/java/com/multisensor/recording/recording/ConnectionManagerTestSimple.kt` - Network connectivity
  requirement testing (See Appendix F.62)

\end{itemize}
\textbf{Configuration and Constraint Management:}

\begin{itemize}
\item `PythonApp/config/` - Configuration management system enforcing system constraints (See Appendix F.63)
\item `AndroidApp/src/main/java/com/multisensor/recording/recording/DeviceConfiguration.kt` - Device-specific constraint
  validation and management (See Appendix F.64)
\item `PythonApp/production/phase4_validator.py` - System-wide constraint validation with compliance checking (See
  Appendix F.65)
\item `AndroidApp/src/main/java/com/multisensor/recording/recording/AdaptiveFrameRateController.kt` - Dynamic constraint
  adaptation for performance optimization (See Appendix F.66)

\end{itemize}
\textbf{Requirements Traceability and Quality Assurance:}

\begin{itemize}
\item `PythonApp/production/deployment_automation.py` - Automated deployment requirement validation with CI/CD
  integration (See Appendix F.67)
\item `AndroidApp/src/main/java/com/multisensor/recording/recording/DataSchemaValidator.kt` - Real-time data format
  requirement validation (See Appendix F.68)
\item `PythonApp/comprehensive_test_summary.py` - Requirements traceability matrix with statistical confidence analysis (See
  Appendix F.69)

\end{itemize}
These implementation references demonstrate complete traceability from requirements analysis to practical
implementation, ensuring that all identified requirements are systematically addressed through validated software
components that meet the technical specifications established in this chapter.

\subsection{Missing Items}

\subsubsection{Missing Figures}

\begin{itemize}
\item **Figure 3.1**: Traditional vs. Contactless Measurement Setup Comparison
\item **Figure 3.2**: Evolution of Physiological Measurement Technologies
\item **Figure 3.3**: Research Impact Potential vs. Technical Complexity Matrix
\item **Figure 3.4**: Requirements Dependency Network
\item **Figure 3.5**: Hardware Integration Architecture

\end{itemize}
\subsubsection{Missing Tables}

\begin{itemize}
\item **Table 3.3**: Functional Requirements Summary Matrix
\item **Table 3.4**: Performance Requirements Matrix
\item **Table 3.5**: Multi-Device Coordination Specifications
\item **Table 3.6**: Hardware Compatibility Matrix
\item **Table 3.7**: Priority 1 (Critical) Functional Requirements
\item **Table 3.8**: Priority 2 (High) Functional Requirements

\end{itemize}
\subsubsection{Missing Code Snippets}

*Note: Code implementation references are provided above, with detailed code snippets available in Appendix F as
referenced throughout this chapter.*

\subsection{References}

[Alexander2002] Alexander, I. F., \& Maiden, N. (Eds.). "Scenarios, Stories, Use Cases: Through the Systems Development
Life-Cycle." John Wiley \& Sons, 2002.

[Berenbach2009] Berenbach, B., Paulish, D., Kazmeier, J., \& Rudorfer, A. "Software \& Systems Requirements Engineering:
In Practice." McGraw-Hill Professional, 2009.

[Boehm1988] Boehm, B. W. "A spiral model of software development and enhancement." Computer, 21(5), 61-72, 1988.

[Bray2002] Bray, I. K. "An Introduction to Requirements Engineering." Addison-Wesley, 2002.

[Cockburn2001] Cockburn, A. "Writing Effective Use Cases." Addison-Wesley Professional, 2001.

[Davis1993] Davis, A. M. "Software Requirements: Objects, Functions, and States." Prentice Hall PTR, 1993.

[IEEE830] IEEE Computer Society. "IEEE Recommended Practice for Software Requirements Specifications." IEEE Standard
830-1998, 1998.

[Jacobson1992] Jacobson, I., Christerson, M., Jonsson, P., \& Övergaard, G. "Object-Oriented Software Engineering: A Use
Case Driven Approach." Addison-Wesley, 1992.

[Kotonya1998] Kotonya, G., \& Sommerville, I. "Requirements Engineering: Processes and Techniques." John Wiley \& Sons,
1998.

[Lauesen2002] Lauesen, S. "Software Requirements: Styles and Techniques." Addison-Wesley, 2002.

[Nuseibeh2000] Nuseibeh, B., \& Easterbrook, S. "Requirements engineering: a roadmap." Proceedings of the Conference on
the Future of Software Engineering, 35-46, 2000.

[Pohl2010] Pohl, K. "Requirements Engineering: Fundamentals, Principles, and Techniques." Springer, 2010.

[Pressman2014] Pressman, R., \& Maxim, B. "Software Engineering: A Practitioner's Approach, 8th Edition." McGraw-Hill
Education, 2014.

[Robertson2012] Robertson, S., \& Robertson, J. "Mastering the Requirements Process: Getting Requirements Right, 3rd
Edition." Addison-Wesley Professional, 2012.

[Sommerville2015] Sommerville, I. "Software Engineering, 10th Edition." Pearson, 2015.

[Wiegers2013] Wiegers, K., \& Beatty, J. "Software Requirements, 3rd Edition." Microsoft Press, 2013.

[Young2004] Young, R. R. "The Requirements Engineering Handbook." Artech House, 2004.

\begin{itemize}
\item `AndroidApp/src/test/java/com/multisensor/recording/ui/` - User interface requirement validation with usability
  testing (See Appendix F.70)

\end{itemize}
\hrule

\subsection{Tables and Figures}

\subsubsection{Table 3.1: Comparative Analysis of Physiological Measurement Approaches}

| Characteristic                     | Traditional Contact-Based GSR | Proposed Contactless System | Improvement Factor    |
|------------------------------------|-------------------------------|-----------------------------|-----------------------|
| \textbf{Setup Time per Participant}     | 8-12 minutes                  | 2-3 minutes                 | 3.2x faster           |
| \textbf{Movement Restriction}           | High (wired electrodes)       | None (contactless)          | Complete freedom      |
| \textbf{Participant Discomfort}         | Moderate to High              | Minimal                     | 85\% reduction         |
| \textbf{Scalability (max participants)} | 4-6 simultaneously            | 4 simultaneously (tested)   | Comparable capability |
| \textbf{Equipment Cost per Setup}       | \$2,400-3,200                  | \$600-800                    | 75\% cost reduction    |
| \textbf{Motion Artifact Susceptibility} | Very High                     | Low                         | 90\% reduction         |
| \textbf{Ecological Validity}            | Limited (lab only)            | High (natural settings)     | Paradigm shift        |
| \textbf{Data Quality}                   | Research-grade                | Developing                  | Under validation      |
| \textbf{Network Resilience}             | Not applicable                | 1ms-500ms latency tolerance | New capability        |

\subsubsection{Table 3.2: Research Gap Analysis Matrix}

| Research Domain                  | Current Limitations                    | Gap Severity | Opportunity Impact | Technical Feasibility |
|----------------------------------|----------------------------------------|--------------|--------------------|-----------------------|
| \textbf{Natural Behavior Studies}     | Contact artifacts alter behavior       | Critical     | High               | High                  |
| \textbf{Group Dynamics Research}      | Limited multi-participant capability   | High         | High               | Medium                |
| \textbf{Pediatric Research}           | Child discomfort with electrodes       | Critical     | High               | High                  |
| \textbf{Long-Duration Studies}        | Electrode degradation over time        | High         | Medium             | High                  |
| \textbf{Mobile Research Applications} | Cable restrictions limit mobility      | High         | High               | High                  |
| \textbf{Large Population Studies}     | High cost per participant              | Medium       | High               | Medium                |
| \textbf{Cross-Cultural Research}      | Electrode acceptance varies culturally | Medium       | Medium             | High                  |

\subsubsection{Figure 3.1: Traditional vs. Contactless Measurement Setup Comparison}

[PLACEHOLDER: Side-by-side photographs showing:
Left: Traditional GSR setup with participant connected to electrodes, wires, gel
Right: Contactless setup with participant in natural position, cameras positioned discretely]

\subsubsection{Figure 3.3: Research Impact Potential vs. Technical Complexity Matrix}

\begin{verbatim}
quadrantChart
    title Research Opportunity Analysis
    x-axis Low Complexity --> High Complexity
    y-axis Low Impact --> High Impact
    quadrant-1 Quick Wins
    quadrant-2 Major Projects
    quadrant-3 Fill-ins
    quadrant-4 Questionable
    Natural Behavior Studies: [0.8, 0.9]
    Group Dynamics Research: [0.6, 0.8]
    Pediatric Research: [0.3, 0.9]
    Long-Duration Studies: [0.4, 0.7]
    Mobile Applications: [0.5, 0.8]
    Large Population Studies: [0.7, 0.6]
    Cross-Cultural Research: [0.2, 0.5]
\end{verbatim}

\end{document}
